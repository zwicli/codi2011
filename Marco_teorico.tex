\section{Marco teórico}

\begin{instrucciones}
 CODI: ¿Es la teoría actualizada y acertada con respecto al problema que se va a estudiar? ¿Su formulación es coherente? ¿Es clara la perspectiva teórica desde donde se ubica el problema?
\end{instrucciones}

%comentar sobre una simetría Z_2
La partícula de materia oscura puede ser estable, o inestable con un
tiempo de vida mucho más grande que la edad del Universo. En el primer
caso usualmente se impone una simetría adicional $Z_2$ bajo la cual
las partículas del modelo estándar son pares, mientras que las
partículas nuevas son impares. Este tipo de simetría garantiza que la
partícula impar más liviana es estable y, de ser neutra, constituye un
buen candidato de materia oscura. En el programa computacional
microMEGAS~\cite{Belanger:2006is} se puede implementar cualquier
extensión del modelo estándar que posea una simetría $Z_2$ para
calcular numéricamente la densidad de reliquia de materia oscura y la
sección eficaz WIMP--nucleón, relevante para los experimentos de de
detección directa como XENON100. Cuando la partícula de materia oscura
es inestable se requiere de alguna simetría para garantizar que su
acoplamiento a las partículas del modelo estándar esté suficientemente
suprimido.

En el caso de supersimetría por ejemplo, todas las interacciones del
modelo están especificadas por la simetría gauge y el superpotencial,
que debe ser una función holomórfica de los campos escalares del
modelo. En el modelo estándar supersimétrico, además de los términos
del superpotencial que constituyen la semilla del potencial escalar y
del Lagrangiano de Yukawa, la invarianza gauge permite los siguientes
términos
\begin{equation}
  \label{eq:5}
  W_{\cancel{R_p}} = \mu_i\widehat{L}_i\widehat{H}_u + 
  \lambda_{i j k}\widehat{L}_i\widehat{L}_j\widehat{l}_k +
  \lambda'_{i j k}\widehat{L}_i\widehat{Q}_j\widehat{d}_k + 
  \lambda''_{ijk}\widehat{u}_i\widehat{d}_j\widehat{d}_k\,
\end{equation}
La simetría $Z_2$, conocida en éste caso como paridad R, $R_p$,
prohíbe todos estos términos, y cuando la partícula impar bajo paridad
R es el neutralino o el gravitino, ésta puede ser una candidato de
materia oscura. Aunque se rompa paridad R y alguno de estos términos
queden permitidos, el gravitino puede seguir siendo un candidato de
materia oscura pues sus decaimientos están suprimidos por la escala de
Planck. Para que el neutralino pueda ser materia oscura inestable
cuando se rompe paridad R, se requiere que los acoplamientos sean
extremadamente pequeños: del orden de $10^{-24}$ para los
acoplamientos trilineales ${\lambda^{(\ }}'\;''{}^)$. 

En modelos supersimétricos extendidos para incluir una simetría
horizontal anómala $U(1)_H$ a la Froggatt-Nielsen (FN)
\cite{Froggatt:1978nt}, las partículas del modelo estándar y sus
supercompañeros no llevan un número cuántico de paridad R, y en su
lugar llevan una carga horizontal (carga~$H$). Para una artículo de
revisión ver \cite{Dreiner:2003hw}.  Además, este tipo de modelos
involucran nuevos campos pesados de FN y, en la realización más
simple, un supercampo singlete electrodébil $\Phi$ de carga $H$
$-1$. Términos efectivos invariantes bajo $SU(3)_c\times SU(2)_L\times
U(1)_Y\times U(1)_H$ que conservan y también que violan paridad R,
surgen una vez los grados de libertad pesados son integrados debajo la
escala de los campos FN ($M$), donde $M$ corresponde a la masa de
Planck. Estos términos involucran factores del tipo $(\Phi/M)^n$,
donde $n$ esta fijado por las cargas horizontales de los campos
involucrados, y determina si un término en particular puede o no estar
presente en el superpotencial. La holomorfía del superpotencial
prohíbe todos los términos para los cuales $n<0$, y aunque ellos se
generaran después la ruptura de supersimetría a través el potencial de
K\"ahler \cite{Giudice:1988yz} estos términos son en general mucho más
suprimidos que aquellos para los cuales $n\ge0$.  Términos con $n$
fraccionario están completamente prohibidos. Finalmente, una vez
$U(1)_H$ se rompe los términos con $n$ positivo mantienen
acoplamientos de Yukawa determinados, hasta factores de orden uno, por
$\theta^n=(\langle\phi\rangle/M)^n$. Los acoplamientos de Yukawa del
modelo estándar provienen típicamente de términos de esta clase.

En este tipo de modelos, los ángulos de mezcla de los quarks, las masa
de los leptones cargados, y las condiciones de cancelación de
anomalías restringen los posibles asignamientos de cargas $H$
\cite{Leurer:1992wg,Binetruy:1996xk}. Ya que el número de
restricciones es siempre menor que el número de cargas $H$ algunas de
ellas quedan necesariamente sin restricciones, y aparte del límite
superior en sus valores pueden considerarse como parámetros libres que
se puede determinar a partir de información fenomenológica
adicional. 


En el caso de modelos supersimétricos basados una simetría de sabor
anómala $U(1)_H$, con un solo flavón de carga $-1$, una vez se
introducen las condiciones teóricas y fenomenológicas, la solución más
óptima se puede expresar en términos de 4 cargas $H$ libres que se
puede usar para explicar el valor de los 45 parámetros del
superpotencial que violan paridad R en la ec.~(\ref{eq:5})
\cite{Mira:2000gg,Dreiner:2003hw,Dreiner:2003yr,Dreiner:2007vp,Dreiner:2006xw,Sierra:2009zq}.

De las diferentes posibilidades se pueden construir modelos: con 
\begin{enumerate}
\item el modelo con ruptura bilineal de paridad R, es decir, con
  acoplamientos $\mu_i$ que explican las masas y mezclas de los
  neutrinos \cite{Mira:2000gg,Dreiner:2003hw,Dreiner:2006xw}.
\label{item:1}
\item modelos donde la simetría discreta de conservación de
  paridad R, o una simetría equivalente, queda como remanente de la
  ruptura espontánea de la simetría $U(1)_H$
  \cite{Dreiner:2003hw,Dreiner:2003yr,Dreiner:2007vp}. Las masas y
  mezclas de neutrinos se pueden explicar con la introducción de
  neutrinos derechos de carga $H$ semientera.
\label{item:2}
\item Modelos con violación de número leptónico a través de términos
  trilineales con ruptura de paridad R. Se pueden construir modelos
  con hasta dos términos del tipo $\lambda_{ijk}$, con los tres
  índices diferentes \cite{Sierra:2009zq}.  En este caso las cargas
  $H$ se pueden escoger de manera que los acoplamientos trilineales
  que violan paridad R queden muy suprimidos, del orden de $10^{-23}$,
  de modo que si los neutralinos son la partícula supersimétrica más
  livina (LSP de sus siglas en inglés), éstos pueden ser candidatos a
  materia oscura inestable con un tiempo de vida media del orden de
  $10^{26}\,$~sec. Dichos modelos se pueden usar para explicar las
  anomalías en rayos cósmicos \cite{Sierra:2009zq} recientemente
  detectadas por experimentos como PAMELA \cite{Adriani:2008zr}, ATIC
  \cite{:2008zzr} y Fermi~LAT \cite{Abdo:2009zk}. Estos datos sugieren
  que el decaimiento del neutralino sea principalmente leptónico, lo
  cual hace especialmente interesante que la única posibilidad
  consistente con simetrías horizontales de lugar precisamente a
  decaimientos leptónicos.
\label{item:3}
\item Modelos con violación de número bariónico a través de términos
  trilineales que violan $R$--paridad del tipo $\lambda''_{ijk}$. La
  fenomenología de este tipo de modelos aún no ha sido estudiada en la
  literatura
\label{item:4}
\end{enumerate}
En este proyecto pretendemos estudiar de manera sistemática la
fenomenología de los modelos \ref{item:1}. y \ref{item:3}. El caso
\ref{item:3}. sin embargo, nos enfocaremos en la posibilidad en la que
el LSP es el gravitino, donde creemos que también podemos explicar las
masas de los neutrinos, y además como en este caso los acoplamientos
trilineales pueden ser mucho más grandes, tenemos la posibilidad de
comprobar el modelo en el LHC.

\subsection{Justificación del problema}

En el Grupo nos hemos enfocado en extensiones del modelo estándar que
dan cuenta de las masas y mezclas de neutrinos. Hemos estudiado
exhaustivamente las predicciones del modelo con ruptura bilineal de
paridad R que incluye sólo los tres términos con $\mu_i$ (caso
\ref{item:1}.), tanto para el Tevatron como para el LHC, asumiendo que
el neutralino es la LSP
\cite{Magro:2003zb,deCampos:2005ri,deCampos:2007bn,deCampos:2008ic,deCampos:2008re,DeCampos:2010yu}. En
el último trabajo al respecto hemos determinado el nivel de precisión
con el que se puede llegar a medir en el LHC la correlación entre
decaimientos de neutralinos a muón y tau con el ángulo de mezcla
atmosférico de neutrinos: una predicción muy concreta que de no
observarse en el LHC en los próximos años descartaría completamente el
modelo como el mecanismo de generación de masas para neutrinos.
\begin{proyecto}
  En este proyecto pretendemos seguir explorando más correlaciones de
  de observables en el LHC con física de neutrinos para determinar con
  que nivel de precisión se podrían llegar a medir en el LHC. Cuando
  el neutralino es la LSP, los decaimientos a tres cuerpos mediados
  por sfermiones con muones y electrones en los estados finales, están
  correlaciones con el ángulo de mezcla solar, y la longitud de
  decaimiento del neutralino está correlacionada con la diferencia de
  masa atmosférica.
\end{proyecto}

Cuando el gravitino es la LSP, se constituye en un candidato viable de
materia oscura inestable. Éste caso ha sido estudiado exhaustivamente
en la literatura por sus implicaciones en experimentos de rayos
cósmicos, especialmente en el caso en que las masas de neutrinos son
explicadas a través del mecanismo de seesaw. En tal caso el modelo
también explica bariogénisis a través de leptogénesis. De hecho, como
hemos mostramos en \cite{Choi:2010jt}, y ratificado por otro grupo con
datos más recientes en \cite{Garny:2010eg}, los últimos datos de
líneas de rayos gamma publicado por FERMI-LAT excluyen la posibilidad
de que las masas de neutrinos puedan ser generadas por términos
bilineales de ruptura de paridad R si la temperatura de
recalentamiento está sobre $10^9\ $GeV como sugiere leptogénesis. En
esta región del espacio de parámetros el gravitino es mayor de unos 10
GeV y los acoplamientos bilineales son tan pequeños que el decaimiento
de la partícula siguiente a la LSP, la NLSP (de sus siglas en inglés),
ocurre fuera del detector en el LHC, por lo que el modelo es
básicamente indistinguible del MSSM. En éste caso los datos de
detección indirecta de materia oscura a través de rayos cósmicos
serían la única forma de diferenciar el modelo del MSSM (donde el
neutralino es el candidato de materia oscura). Sin embargo, si
consideremos masas de gravitinos más pequeños, aunque ya no podríamos
explicar bariognesis a través de leptogénesis, recuperamos la
posibilidad explicar las masas y mezclas de los neutrinos a través de
los términos bilineales que rompen paridad R~\cite{Hirsch:2005ag}, un
mecanismo, que a diferencia del seesaw, si se puede verificar en el
LHC.

\begin{proyecto}
  En este proyecto realizaremos las simulaciones de las señales del
  NSLP para el LHC  en las regiones del
  espacio de parámetros donde el gravitino es un buen candidato a
  materia oscura y los términos bilineales explican los datos de
  oscilaciones de neutrinos. 
\end{proyecto}

\begin{darkmatter}
  La metodología desarrollada en el estudio exhaustivo del modelo con
  ruptura bilineal de paridad R la hemos logrado aplicar a otros
  modelos de generación radiativa de masas de neutrinos con nuevas
  partículas a la escala del TeV asequibles en el LHC
  \cite{Sierra:2008wj,AristizabalSierra:2006ri}. Recientemente hemos
  comenzado a explorar modelos que puedan dar cuenta simultáneamente
  de masas de neutrinos y candidatos de materia oscura
  \cite{Hirsch:2005ag,Choi:2010jt,Sierra:2008wj}. En el caso del
  seesaw radiativo por ejemplo, se implementa una simetría $Z_2$ de
  manera que la masa de neutrinos sólo pueda ser generada
  radiativamente con los neutrinos derechos a la escala del TeV
  garantizando que el modelo pueda comprobarse en el LHC. La misma
  simetría garantiza que el escalar o a un neutrino derecho puedan ser
  buenos candidatos de materia oscura. Finalmente, la estructura
  escalar del modelo permite tener el Higgs mucho más pesado que en el
  modelo estándar de forma compatible con las correcciones
  electrodébiles.
% Las regiones del modelo en la cual la partícula de materia oscura es
% escalar ha sido bastante estudiada en la literatura en el contexto
% del modelo conocido como doblete inerte \cite{LopezHonorez:2010tb}. Sin
% embargo, las regiones en las cuales el neutrino derecho es el
% candidato de materia oscura todavía requieren de


\begin{proyecto}
  Todo esto hace muy llamativo el estudio de las señales del seesaw
  radiativo para el LHC, y en los experimentos de detección directa de
  materia oscura. En este proyecto pretendemos establecer todas las
  regiones del espacio de parámetros donde se puede tener la densidad
  de reliquia de materia oscura y las masas y mezclas de neutrinos
  adecuadas. En cada región se establecerán las señales que se esperan
  en el LHC y en experimentos de detección directa de materia oscura,
  y para las señales más representativas hacer simulaciones para el
  detector ATLAS del LHC.
\end{proyecto}
\end{darkmatter}

La otra posibilidad de tener términos de violación de número leptónico
en el superpotencial compatible con la simetría horizontal, es el caso
\ref{item:3}. donde tenemos un modelo con ruptura trilineal de paridad
R a través de términos $\lambda$. Esta posibilidad es muy llamativa
porque da lugar a un candidato de materia oscura que decae sólo
leptónicamente. Nuestro grupo \cite{Nardi:2008ix} fue uno de los
primeros en proponer una explicación en términos de materia oscura
inestable para explicar el exceso de positrones observado por el
satelite PAMELA en el 2008 \cite{Adriani:2008zr}. Luego hemos
construido un modelo basado en supersimetría con ruptura de paridad R
a través de términos trilineales del tipo $\lambda$, caso
\ref{item:3}, para explicar la preferencia por decaimientos leptónicos
de la partícula de materia oscura, que en este caso es el
neutralino \cite{Sierra:2009zq}.
\begin{proyecto}
  En este proyecto queremos explorar la posibilidad de explicar
  también las masas de neutrinos, y el gravitino como materia oscura
  en el marco del modelo con ruptura de paridad R a través de términos
  trilineales del tipo $\lambda$.
\end{proyecto}
En un proyecto en marcha estamos explorando la posibilidad opuesta
\ref{item:4}., de un candidato de materia oscura que decae sólo
hadrónicamente. La simetría horizontal garantiza que sólo los
acoplamientos trilineales de quarks derechos estén permitidos. En
dicho proyecto hemos extendido el modelo para incluir masas de
neutrinos. Ésta experiencia nos servirá para hacer las modificaciones
necesarias al modelo con acoplamientos trilineales leptónicos del tipo
$\lambda$ para dar cuenta de masas y mezclas para los neutrinos.


\begin{proyecto}
  En términos generales, en este proyecto queremos explorar modelos
  motivados por falencias fenomenológicas del modelo estándar,
  especialmente aquellos modelos que no sólo tienen implicaciones en
  el LHC, sino también en experimentos de detección de rayos cósmicos,
  y experimentos de detección directa de materia oscura.
\end{proyecto}



\begin{proyecto}
  Con base en lo planteado anteriormente, lo que proponemos en este
  proyecto es tratar de responder la siguiente pregunta: ¿Cuáles
  serían las restricciones impuestas por los recientes y futuros datos
  de los experimentos de física de partículas sobre modelos que
  presenten una partícula candidata a materia oscura o que generen
  masas para los neutrinos?
\end{proyecto}

%%% Local Variables: 
%%% mode: latex
%%% TeX-master: "Ficha-2011_bsm"
%%% End: 
