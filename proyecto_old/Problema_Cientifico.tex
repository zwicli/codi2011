\section{Problema Cientifico}	
\begin{instrucciones}
  en este ítem usted deberá describir de forma
precisa y completa la naturaleza y magnitud del problema de investigación que se quiere
abordar. Formule claramente las preguntas concretas a las cuales se quiere responder en el
contexto del problema planteado.
\end{instrucciones}


%%Final
Con la entrada en operación del LHC, se espera descubrir señales de materia oscura y de modelos que tengan algún mecanismo radiativo para generar masas de neutrinos.

Por otro lado, los recientes datos reportados por las colaboraciones PAMELA y Fermi han impuesto restricciones muy fuertes sobre el flujo de rayos cósmicos provenientes de partículas de materia oscura. Los futuros experimentos de rayos cósmicos, tales como el experimento AMS-02, Planck, IceCube, etc, esperan mejorar la sensibilidad a este tipo de señales. Así mismo, los experimentos de detección directa de materia oscura como DAMA, CDMS, CoGeNT, XENON han creado grandes expectativas por los resultados presentes y los que se espera obtener en los próximos meses. Los experimentos de física de neutrinos y de radiación cósmica de fondo han entrado ya en una era de medidas de precisión.

Con base en lo planteado anteriormente, lo que proponemos en este proyecto es tratar de responder las siguiente pregunta: ¿Cuáles serían las restricciones impuestas por los recientes y futuros datos de todos estos experimentos sobre  modelos que presenten una partícula candidata a materia oscura o que generen masas para los neutrinos?

\begin{ideas}
  \newpage{}
%oscar
Los recientes datos reportados por las colaboraciones PAMELA y Fermi han impuesto restricciones muy fuertes sobre el flujo de rayos cósmicos proveniente de partículas de materia oscura. Estas restricciones afectarían directamente el espacio de parámetros permitido por los experimentos hasta ahora. Adicional a estas nuevas restricciones, también se debe tener en cuenta la sensibilidad de los futuros experimentos de rayos cósmicos, tales como el experimento AMS-02, Planck, IceCube, etc.

Por otro lado, con la entrada en operación del LHC junto con el tevatrón, se esperan descubrir señales del bosón de Higgs, de la materia oscura, dimensiones extra, supersimetría, etc. En algunos modelos más allá del Modelo Estándar se tienen predicciones concretas sobre los posibles productos de aniquilación y desintegración del candidato a materia oscura que podrían ser detectados en el LHC o tevatrón.

Con base en lo planteado anteriormente, lo que proponemos en este proyecto es tratar de responder la siguiente pregunta: ¿Cuáles serían las restricciones impuestas por los recientes y futuros datos astrofísicos y de aceleradores sobre  modelos que presenten una partícula candidata a materia oscura?

%%Nuevo
Actualmente se encuentra funcionado el acelerador Tevatron en Fermilab en Estados Unidos. Este acelera protones y antiprotones a una energía de 1 TeV, y se espera que siga funcionado por 2 años más. Ya ha entrado en funcionamiento el acelerador LHC en el CERN en la frontera entre Francia y Suiza. Aunque de momento se encuentra en estado de pruebas funcionando a energías que alcanzarán los 7TeV, en un futuro cercano este  acelerará protones en ambos sentidos a una energía que alcanzará hasta los 14 TeV. La mayoría de los experimentos conducidos son hechos en grandes instrumentos llamados detectores. Por ejemplo el LHC tiene 4 detectores. En dos de ellos ATLAS y CMS hay grupos colombianos participando. En los últimos años el Grupo ha tenido una estrecha vinculación con el Grupo de Marta Losada que esta participando en la colaboración ATLAS.

A nivel de detección de radiación cósmica tenemos el satélite PAMELA para detección de rayos cósmicos de electrones, positrones, antiprotones, y núcleos livianos. En agosto del año pasado PAMELA reporto una abundancia anómala de positrones %\cite{pamela}, 
que ha dado lugar a un sinnúmero de publicaciones tratando de explicar su origen.  Fermi-LAT es otro satélite para la detección de rayos cósmicos que también estará funcionando  durante los próximos años. En sus primeras medidas %\cite{fermi} 
también ha reportado una abundancia anómala de positrones y electrones. El detector de rayos cósmicos AMS-02~%\cite{ams:2009},  
que será instalado en la Estación Espacial  internacional, medirá el espectro de antiprotones y positrones en un rango de energía mucho más amplio y con una estadística mucho mejor. En cuanto a detección directa de materia oscura grandes expectivas por los resultados presentes y los que se espera obtener en los próximos meses ... \url{http://resonaances.blogspot.com/2010/03/another-experiment-sees-dark-matter.html}

En los próximos años entrará en El satélite Planck fue lanzado en mayo, y desde julio  se encuentra en pleno funcionamiento. Con los datos obtenidos de este experimento se espera medir la radiación cósmica de fondo con mucha más precisión que su antecesor WMAP. Además esta el experimento Auger en Argentina que detecta rayos cósmicos de ultra alta energía que alcanzan a llegar a la superficie terrestre. 

También hay varios experimentos de física de neutrinos en marcha que esperan mejorar los datos actuales de oscilaciones de neutrinos, determinar el ángulo de mezcla $\theta_{13}$, del cual actualmente solo hay cotas, y si este resulta ser diferente de cero, buscar evidencia de violación de CP en el sector de neutrinos, un ingrediente crucial para generar bariogénesis a través de leptogénesis.

Esto representa un escenario muy dinámico donde el Grupo debe estar preparado para interpretar cualquier evidencia que resulte del Higgs o nueva física a la luz de los modelos que viene desarrollando. 
	
La pregunta que queremos abordar en el presento proyecto es si: 
¿se podrá establecer en los aceleradores futuros cual es modelo que
explica la generación de masas y mezclas para los neutrinos?

%%Jose david
Por medio del estudio del modelo de seesaw radiativo se predicen algunos canales de decaimiento para la materia oscura que podrían ser observados en el LHC. Si se llegan a detectar dichos canales sabremos que es el modelo que la generación de masas de los neutrinos se da a través del mecanismo propuesto en este modelo.

Aunque se ha encontrado que los neutrinos tienen masas a la escala de
sub-eV y se mezclan %\cite{hep-ph/0405172}, 
con experimentos sólo sobre
física de neutrinos no se puede establece cual el modelo que contiene
el mecanismo apropiado de generación de masas y mezcla de neutrinos.
En particular, será muy difícil establecer la naturaleza de los
neutrinos, es decir si tienen masa de Dirac o Majorana.

Los aceleradores de partículas tienen la posibilidad de
establecer cual es el mecanismo de generación de masas y mezclas de
neutrinos y de forma indirecta establecer entonces la naturaleza de
los mismos. El mecanismo seesaw %\cite{Gell-Mann:vs} 
ha sido la
explicación más popular para la peque nez de las masas de neutrinos. Es
elegante y simple, y depende sólo de análisis dimensional para la
nueva física que se requiere. Los datos actuales de neutrinos
apuntan a una escala de seesaw del orden $10^{10}$ - $10^{15}$ GeV,
donde la violación de número leptónico ocurre a través de masas de
Majorana de los neutrinos derechos. Con una escala tan alta, lo
efectos de violación de sabor leptónico en procesos diferentes a la
oscilación de neutrinos son extremadamente peque nos. Por ejemplo el
decaimiento del muon a electrón fotón en extensiones seesaw del Modelo
Estándar, es decenas de ordenes de magnitud menor a la cota actual

Una alternativa al mecanismo seesaw que también explica naturalmente
la peque nez de la masa de neutrinos son los mecanismos de generación
radiativa, ver por ejemplo los siguientes artículos recientes y
sus referencias
%\cite{hep-ph/0503059,AristizabalSierra:2006ri,Babu:2002uu,Dong:2006vk}.
En este aproximación las masas de neutrinos son cero a nivel árbol y
son inducidas únicamente como correcciones radiativas finitas.  Estas
correcciones radiativas son proporcionales a la raíz cuadrada de
las masas del leptón cargado (o quark) dividas por la escala de nueva
física $\Lambda$. Las masas de los neutrinos pueden estar en el rango
de sub-eV aún cuando la escala de nueva física sea del orden de
TeV. En este caso, los procesos de violación de número leptónico en
otros procesos diferentes a la física de neutrinos pueden llegar
a ser accesibles a los aceleradores.

En este proyecto proponemos analizar en detalle las consecuencias
experimentales de modelos específicos donde las masas de
neutrinos se generan radiativamente. En particular estamos interesados
en calcular los decaimientos de las partículas del modelo que
estén controlados por los mismos parámetros que inducen las masas y
las mezclas de los neutrinos. Estos decaimientos podrían no sólo
ayudar a determinar el mecanismo de generación de masas de neutrinos
con aceleradores de partículas, sino también ayudar a determinar
experimentalmente los parámetros del modelo que aparecen en la matriz
de masa de neutrinos.


%%%%%%%%%%%%        Inicio  OSCAR
%Inicio Oscar.

Los recientes datos reportados por las colaboraciones PAMELA y Fermi-LAT
han señalado un exceso en el flujo de positrones y en el flujo total de 
electrones y positrones respectivamente, mientras para el flujo de antiprotones no se 
ha observado exceso alguno. El exceso en el flujo de antimateria puede ser 
interpretado como evidencia de nuevas fuentes astrofísicas de rayos cósmicos de 
alta energía en nuestra galaxia, tales como pulsares o remanentes de supernovas. 
En este escenario, el flujo de positrones y rayos gamma astrofísicos constituirían 
un nuevo background para las posibles señales de detección materia oscura o para las posibles zonas 
de exclusión de los diferentes modelos que predicen la existencia de materia oscura. En particular, 
los background de los flujos de positrones, antiprotones y rayos gamma afectarían directamente 
el espacio de parámetros permitido por los experimentos hasta ahora de los modelos 
supesimétricos con violación de paridad R extendidos con una simetría horizontal 
$U(1)_H$ donde el neutralino es candidato a materia oscura inestable. Adicional a 
estas nuevas restricciones, también se debe tener en cuenta la sensibilidad de 
los futuros experimentos de rayos cósmicos, tales como el experimento AMS-02.

Por otro lado, con la entrada en operación del LHC junto con el tevatrón, 
se esperan descubrir señales del bosón de Higgs, de la materia oscura, dimensiones extra, 
supersimetría, entre muchas otras señales.  En modelos supersimétricos con violación de paridad R 
extendidos con una simetría horizontal $U(1)_H$ donde el neutralino decae rápidamente vía operadores 
que violan número bariónico, se tendrían predicciones concretas sobre los posibles productos de 
desintegración del neutralino que podrían ser detectados en el LHC o tevatrón.

Con base en lo planteado anteriormente, lo que proponemos en este proyecto es 
tratar de responder la siguiente pregunta: ¿Cuáles serían las restricciones 
impuestas por los recientes y futuros datos astrofísicos y de aceleradores sobre 
los modelos supersimétricos con violación de paridad R extendidos con una simetría $U(1)_H$ que presentan un neutralino inestable?

Fin Oscar
%%%%%%%%%%%%        Fin  OSCAR

\end{ideas}
%%% Local Variables: 
%%% mode: latex
%%% TeX-master: "main"
%%% End: 
