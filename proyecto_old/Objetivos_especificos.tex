\section{Objetivos especificos}
%%%

%Final
Para el modelo supersimétrico con violación bilineal de paridad R (VBPR) [1],  establecer a través de la simulación de eventos con neutralinos en el LHC, con que nivel de precisión se pueden llegar a medir las correlaciones establecidas entre señales de neutralinos y diferentes observables de física de neutrinos.

Calcular el flujo de rayos gamas y confrontarlo con lo observado en experimentos de rayos cósmicos en modelos supersimétricos donde el gravitino es el candidato de materia oscura inestable (CMOI) en la región de masa de hasta 80 GeV. Estudios recientes  muestran que en está región son importantes canales de decaimiento que no se habían considerado previamente [4]. 

Explicar los resultados de PAMELA y ATIC en modelos supersimétricos con violación trilineal de paridad R (VTPR) y una simetría horizontal con un mecanismo see-saw para la generación de masas de neutrinos y con el gravitino como CMOI.

Obtener las restricciones que los diferentes experimentos de rayos cósmicos presentes y futuros imponen sobre el neutralino o el gravitino como CMOI.

Establecer las señales en aceleradores de modelos con VTPR y violación de número bariónico inducida por una simetría Abeliana anómala U(1). 

Determinar las regiones del espacio de parámetros compatible con física de neutrinos y con la densidad de materia oscura del seesaw radiativo, para estudiar sus implicaciones experimentales.

Continuar con el estudio sistemático [6] de modelos 3-3-1 de tres y cuatro familias.

\begin{ideas}
  \newpage{}

%%Oscar
Para el modelo del doblete inerte, estudiar la producción y propagación de positrones y antiprotones  provenientes de  la aniquilación de partículas de materia oscura escalar. Para las regiones consistentes con WMAP, determinar cuáles serían las restricciones astrofísicas sobre la sección eficaz y la masa de la materia oscura escalar.  

Calcular las implicaciones en aceleradores de modelos con violación de número bariónico inducida por una simetría abeliana anómala U(1)H. En este tipo de escenarios se espera obtener predicciones muy específicas en el cociente entre acoplamientos, que se verían reflejados entre cocientes de amplitudes de decaimiento a diferentes canales hadrónicos, que permitan comprobar la hipótesis de simetrías horizontales para explicar la jerarquía de masas de los fermiones directamente en el LHC. 

En el modelo de seesaw radiativo, determinar la jerarquía de masas de las partículas impares del sector de parámetros compatible con observaciones de WMAP y neutrinos. Estudiar los posibles canales visibles en el LHC que involucren materia oscura del modelo, y analizar el flujo de rayos gamma producto de la aniquilación de materia oscura en la galaxia, en el modelo de seesaw radiativo.

Obtener las restricciones que los diferentes experimentos de rayos cósmicos 
presentes y futuros imponen sobre la vida media y la masa del neutralino y del gravitino como 
candidato a materia oscura que decae, en modelos que violan paridad R.


%%%Jose David
\begin{itemize}
 \item Determinar la jerarquía de masas de las particulas impares del sector de parámetros compatible con observaciones de WMAP y neutrinos en el modleo de seesaw radiativo.
 \item Estudiar los posibles canales visibles en el LHC que involucren materia oscura del modelo de seesaw radiativo.
 \item Analizar el flujo de rayos gamma producto de la aniquilacion de materia oscura en la galaxia, en el modelo de seesaw radiativo.
\end{itemize}

%%%Proyecto viejo
\begin{itemize}
\item En modelos con VBPR se ha encontrado que la matriz de masa de
  neutrinos puede ser reconstruida en aceleradores como el LHC o el
  ILC para varias posibilidades de partícula supersimétrica más
  ligera. En esté trabajo pretendemos establecer a través de la
  simulación de eventos con neutralinos en el LHC, con que nivel de
  precisión se puede llegar a medir una de las correlaciones
  establecidas entre cocientes de decaimientos de neutralinos y el
  ángulo de mezcla atmosférico.
\item En el modelo de Zee se ha evaluado recientemente la posibilidad
  de reconstruir su matriz de masa de neutrinos en aceleradores
  futuros para un caso especial de dicha matriz. En el presente
  proyecto pretendemos evaluar a que nivel se puede reconstruir la
  matriz de masa en el caso general
\item Para el modelo de Babu estableceremos el conjunto de observables
  necesarios para reconstruir su matriz de masa de neutrinos en
  aceleradores futuros
\item Para modelos 3-3-1 con neutrinos derechos y con simetrías
  discretas que prohiban masas de neutrinos a nivel árbol, queremos
  evaluar su viabilidad para explicar los datos actuales de
  física de neutrinos
\item Producir cuatro artículos internacionales con los resultados
  obtenidos para cada uno de los modelos estudiados
\item Formar un estudiante de doctorado y uno de maestría
\end{itemize}
%%%%%%%%%%%%        Inicio  OSCAR
Inicio Oscar.\\
\begin{itemize}
\item Obtener las restricciones que los diferentes experimentos de rayos cósmicos 
presentes y futuros imponen sobre la vida media y la masa del neutralino como 
candidato a materia oscura que decae, en modelos con términos trilineales 
que violan paridad R con carga $n$ negativa. Los resultados que se obtenga se 
espera que sean validos para modelos supersimétricos generales con violación trilineal de paridad R.

\item Calcular las implicaciones en aceleradores de modelos con violación 
de número bariónico a través de términos trilineales $\lambda''$ inducidos por la simetría 
abeliana anómala $U(1)_H$. En este tipo de escenarios se espera obtener 
predicciones muy específicas en el cociente entre acoplamientos, que se verían reflejados entre 
cocientes de amplitudes de decaimiento a diferentes canales hadrónicos, que permitan comprobar 
la hipótesis de simetrías horizontales para explicar la jerarquía de masas de los fermiones directamente en el LHC. 
\end{itemize}
Fin Oscar
%%%%%%%%%%%%        Fin  OSCAR
\end{ideas}

%%% Local Variables: 
%%% mode: latex
%%% TeX-master: "main"
%%% End: 

