\section{Resultados esperados}	
\begin{instrucciones}
  Formule los resultados directos verificables que se
alcanzarán con el desarrollo de los objetivos específicos del proyecto. Estos deben ser coherentes
con los objetivos y con la metodología planteada.


  \begin{enumerate}
  \item \textbf{Relacionados con la generación de conocimiento y/o nuevos desarrollos
 tecnológicos:} Incluye resultados/productos que corresponden a nuevo
 conocimiento científico o tecnológico o a nuevos desarrollos o adaptaciones de
 tecnología que puedan verificarse a través de publicaciones científicas,
 productos o procesos tecnológicos, patentes, normas, mapas, bases de datos,
 colecciones de referencia, secuencias de macromoléculas en bases de datos de
 referencia, registros de nuevas variedades vegetales, etc.
\item \textbf{Conducentes al fortalecimiento de la capacidad científica
  nacional:} Incluye resultados/productos tales como formación de
  recurso humano a nivel profesional o de posgrado (trabajos de grado
  o tesis de maestría o doctorado sustentadas y aprobadas),
  realización de cursos relacionados con las temáticas de los
  proyectos (deberá anexarse documentación soporte que certifique su
  realización), formación y consolidación de redes de investigación
  (anexar documentación de soporte y verificación) y la construcción
  de cooperación científica internacional (anexar documentación de
  soporte y verificación).
\item \textbf{Dirigidos a la apropiación social del conocimiento:}
  Incluye aquellos resultados/productos que son estrategias o medios
  para divulgar o transferir el conocimiento o tecnologías generadas
  en el proyecto a los beneficiarios potenciales y a la sociedad en
  general. Incluye tanto las acciones conjuntas entre investigadores y
  beneficiarios como artículos o libros divulgativos, cartillas,
  videos, programas de radio, presentación de ponencias en eventos,
  entre otros.
  \end{enumerate}

  Para cada uno de los resultados/productos esperados identifique (en
  los cuadros a continuación) indicadores de verificación (ej:
  publicaciones, patentes, registros, videos, certificaciones, etc.)
  as\'\i{} como las instituciones, gremios y comunidades beneficiarias,
  nacionales o internacionales, que podrán utilizar los resultados de
  la investigación para el desarrollo de sus objetivos, políticas,
  planes o programas:

\end{instrucciones}

%Final
Publicar al menos 4 artículos internacionales con algunos de los siguientes resultados:

Determinación de la factibilidad del LHC para determinar las correlaciones de VBPR entre el cociente de branchings de neutralino a W mu y W tau con el ángulo de mezcla atmosférico de neutrinos, así como de la correlación entre la longitud de decaimiento del neutralino y la diferencia de masa al cuadrado atmosférica.

Determinación del flujo de rayos gamas y su confrontación con lo observado en experimentos de rayos cósmicos en modelos supersimétricos donde el gravitino es el CMOI con masa de hasta 80 GeV. 

Determinación de acoplamientos, vida media y masa del gravitino como CMOI en modelos de ruptura trilineal de paridad R (RTPR) con violación de número leptónico, que pueda explicar el exceso de positrones en rayos cósmicos. Predicciones de flujo de positrones y rayos gama del modelo para experimentos futuros de rayos cósmicos. 

Determinación de las restricciones que los diferentes experimentos de rayos cósmicos 
presentes y futuros imponen sobre el neutralino como CMOI. 

Determinación de las señales en el LHC para 1) Modelos de RTPR con violación de número bariónico y mecanismo see-saw para masas de neutrinos, inducida por una simetría Abeliana anómala U(1). 2) Regiones del espacio de parámetros compatible con física de neutrinos y densidad de materia oscura del seesaw radiativo. 3) Modelos 3-3-1 con cuatro familias

Formación 1 estudiante de maestría y 1 de doctorado.


\begin{ideas}
\newpage{}

%Oscar
Para el modelo del seesaw radiativo y del doblete escalar inerte, obtener una zona del espacio de parámetros lo suficientemente estrecha como para que sea predictiva, obtener algunos canales visibles que involucren materia oscura en el LHC, y predecir el flujo de rayos gamma en la galaxia producto de la aniquilacion de materia oscura.
Así mismo se espera determinar el rango de tiempos de vida media del candidato a materia oscura inestable que se podrá excluir en experimentos futuros como el AMS-02, IceCube, etc.
De otro lado, esperamos construir el primer modelo estándar supersimétrico con una simetría horizontal anómala U(1) que se pueda contrastar directamente con resultados experimentales de aceleradores. 
Formular el primer modelo con violación de número bariónico a través de términos lambda'' que a la vez de cuenta de las masas y mezclas de neutrinos.
Se espera que los resultados de está investigación clarifiquen hasta que punto se pueden acotar los valores de los parámetros trilineales y bilineales que violan R--paridad en el caso de que las anomalías actuales en rayos cósmicos puedan ser explicadas por fuentes astrofísicas. Se espera mejorar las cotas existentes en varios ordenes de magnitud. 


%%Jose David
\begin{itemize}
 \item Obtener una zona del espacio de parámetros lo suficientemente estrecha como para que sea predictiva, del modelo de seesaw radiativo.
 \item Obtener algunos canales visibles que involucren materia oscura en el LHC, del modelo de seesaw radiativo.
 \item Predecir el flujo de rayos gamma en la galaxia producto de la aniquilacion de materia oscura para el modelo de seesaw radiativo.
\end{itemize}

%%Proyecto viejo
4 artículos en revistas internacionales, formación de estudiante de
maestría, formación de estudiante de doctorado.

\textbf{Tabla 3.5.1 Generación de nuevo conocimiento}\\
\begin{tabular}{|c|c|c|}\hline
   \textbf{Resultado/Producto}&\textbf{Indicador} & \textbf{Beneficiario}\\
   \textbf{esperado}& & \\\hline
   \parbox[t]{4cm}{Desarrollar todos los detalles de un procedimiento de simulación de se nales de modelo con VBpR en el LHC y el ILC, para extraer con la mayor precisión posible la correlación que se puede llegar a obtener entre experimentos en aceleradores y experimentos de neutrinos}& \parbox[t]{4cm}{Publicaciones en revistas internacionales (mínimo tres artículos), por lo menos una conferencia internacional, seminarios en institutos del país y el exterior}& \parbox[t]{4cm}{Comunidad científica de físicos de partículas elementales y experimentales de altas energías.}\\\hline 
   & & \\\hline
 \end{tabular}

\textbf{Tabla 3.5.2 Fortalecimiento de la comunidad científica}\\
\begin{tabular}{|c|c|c|}\hline
   \textbf{Resultado/Producto}&\textbf{Indicador} & \textbf{Beneficiario}\\
   \textbf{esperado}& & \\\hline
 \parbox[t]{4cm}{Formación de estudiante a nivel de maestría} &\parbox[t]{4cm}{Defensa exitosa de una tesis de Maestría. Publicaciones internacionales firmada por el estudiante} & \\\hline 
\parbox[t]{4cm}{Formación de estudiante a nivel de Doctorado}  & \parbox[t]{4cm}{Defensa exitosa de una tesis de Doctorado a corto plazo, iniciando su formación doctoral en el marco de este proyecto}& \parbox[t]{4cm}{Estudiantes, el grupo de investigación, el Instituto de Física.} \\\hline
\parbox[t]{4cm}{}  & \parbox[t]{4cm}{}& \parbox[t]{4cm}{} \\\hline
 \end{tabular}

\textbf{Tabla 3.5.3 Apropiación social del conocimiento}\\
\begin{tabular}{|c|c|c|}\hline
   \textbf{Resultado/Producto}&\textbf{Indicador} & \textbf{Beneficiario}\\
   \textbf{esperado}& & \\\hline
\parbox[t]{4cm}{}  & \parbox[t]{4cm}{}& \parbox[t]{4cm}{} \\\hline
   & & \\\hline
 \end{tabular}


\subsection{Impactos esperados a partir del uso de los resultados:}

\begin{instrucciones}
  Los impactos no necesariamente se logran al finalizar el proyecto, ni
con la sola consecución de los resultados/productos. Los impactos
esperados son una descripción de la posible incidencia del uso de los
resultados del proyecto en función de la solución de los asuntos o
problemas estratégicos, nacionales o globales, abordados. Generalmente
se logran en el mediano y largo plazo, como resultado de la aplicación
de los conocimientos o tecnologías generadas a través del desarrollo
de una o varias líneas de investigación en las cuales se inscribe el
proyecto. Los impactos pueden agruparse, entre otras, en las
siguientes categorías: sociales, económicos, ambientales, de
productividad y competitividad. Para cada uno de los impactos
esperados se deben identificar indicadores cualitativos o
cuantitativos verificables as\'\i:
\end{instrucciones}

\begin{tabular}{|c|c|c|c|}\hline
   \textbf{\parbox[t]{2cm}{\textbf{Impacto esperado}}}&\parbox[c]{4cm}{\textbf{Plazo (a nos) después de finalizado el proyecto: corto (1-4 ), mediano (5-9), largo (10 o más)}} & \textbf{\parbox[t]{2cm}{\textbf{Indicador verificable}}} &\textbf{Supuestos*}\\
   & & &\\\hline 
   & & &\\\hline
 \end{tabular}\\
*Los supuestos indican los acontecimientos, las condiciones o las decisiones,
necesarios para que se logre el impacto esperado.


Consultar proyecto viejo
\end{ideas}



%%% Local Variables: 
%%% mode: latex
%%% TeX-master: "main"
%%% End: 

