\section{ Objetivos del Proyecto }

Proponer modelos que tengan consecuencias medibles en experimentos de aceleradores, de rayos cósmicos, y de detección directa de materia oscura, así como encontrar modelos que expliquen los resultados en cualquiera de ellos.

En modelos específicos para la generación radiativa de masas de neutrinos, establecer si la matriz de masa de neutrinos resultante puede llegar a ser reconstruida en los aceleradores futuros. Se considerarán modelos de supersimetría con Violación Bilineal de Paridad R (VBPR), el modelo de Zee para generación radiactiva de masas
de neutrinos a un loop, el modelo de Babu para generación radiativa de
masas de neutrinos a dos loops, y modelos 3-3-1 con neutrinos
derechos.

Analizar las restricciones astrofísicas y de aceleradores sobre los modelos supersimétricos
con violación de paridad R extendidos con una simetría $U(1)_H$ que presentan un neutralino inestable.


%%% Local Variables: 
%%% mode: latex
%%% TeX-master: Ficha-2011_bsm
%%% End: 

