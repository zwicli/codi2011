\section{Aproximación metodológica}
\begin{instrucciones}
  Esta debe reflejar la estructura lógica y el
rigor científico del proceso de investigación, empezando por la elección de un enfoque
metodológico específico y finalizando con la forma como se van a analizar, interpretar y presentar
los resultados.
\end{instrucciones}
%%%

%Final
Para un modelo que incluya mecanismos de generación radiativa de masas de neutrinos se determina el espacio de parámetros compatible con los datos experimentales sobre física de neutrinos, se calculan los branchings de decaimiento y las secciones eficaces y se establecen correlaciones entre estos observables y los datos de física de neutrinos [7]. Para el caso de VBPR todo esto ya ha sido implementado en el programa computacional SPheno [8]. Finalmente se hace una simulación con PYTHIA [9] de la factibilidad de descubrir el modelo en aceleradores, como hemos venido haciendo con VBPR en [1], y de la viabilidad de medir las correlaciones establecidas, que es uno de los objetivos del proyecto.

Como hemos ya hecho en [3], obtendremos resultados numéricos para los flujos de rayos cósmicos  producidos en el decaimiento del CMOI usando PYTHIA. A partir de los resultados se analizará el tiempo de vida del CMOI  en términos de su masa con el fin de obtener y analizar cuáles serían las regiones del espacio de parámetros restringidas o excluidas por los experimentos de rayos cósmicos.

En este proyecto haremos una implementación del modelo de seesaw radiativo en Micromegas  con el cual se pueden hacer los cálculos de densidad de reliquia, secciones eficaces, amplitudes de decaimiento, etc. Una vez establecidas las regiones del espacio de parámetros relevantes, se procederá con el método para mecanismos de generación radiativa de masas de neutrinos descrito anteriormente.

\newpage{}
%%Oscar
Para un modelo que presente un candidato a materia oscura, proponemos los siguientes pasos para determinar las restricciones de los recientes y futuros experimentos:
1. Obtener expresiones analíticas o numéricas para los flujos de rayos cósmicos producidos en el decaimiento o aniquilación del candidato a materia oscura.
2. Al mismo tiempo, calcular las expresiones analíticas para los decaimientos del candidato a materia oscura. Simular la producción de materia oscura en aceleradores  y su posterior decaimiento o aniquilación. Con base en lo anterior, determinar los branching ratios para diferentes canales de decaimiento de la materia oscura, analizar cuál es la dependencia con los acoplamientos y determinar cuáles serían las restricciones más fuertes provenientes de aceleradores sobre dichos acoplamientos. 
3. Los modelos de seesaw raditivo y del doblete escalar inerte se estudiarán a través del paquete computacional micrOMEGAs, con el cual se hacen actualmente a nivel mundial los cálculos de densidad de reliquia, secciones eficaces, amplitudes de decaimiento, etc.
4. A partir de los resultados construir figuras donde se muestre el tiempo de vida del candidato a materia oscura en términos de su masa con el fin de obtener y analizar cuáles serían las regiones del espacio de parámetros que serían excluidas por los resultados experimentales astrofísicos. 

%%%Jose David
El modelo de seesaw raditivo se estudiara a traves del paquete computacinal micrOMEGAs, con el cual se hacen actualmente a nivel mundial los calculos de densidad de reliquia, secciones eficaces, amplitudes de decaimiento, entre otros en modelos genéricos que pretendan tener canddatos para materia oscura. Además, desarrollaremos scripts que nos permitan hacer el estudio del comportamiento del sector de neutrinos del modelo, para poder compararlo con las observaciones.

%%%Proyecto viejo
Para un modelo que incluya mecanismos de generación radiativa de masas
de neutrinos se propone el siguiente método para determinar si es
posible verificar sus predicciones en aceleradores futuros, en
particular si es posible la reconstrucción de la matriz de masa a
través de observables asociados a los decaimientos de partículas del
modelo:
\begin{enumerate}
\item Cálculo de todos los términos de masas de neutrinos a nivel
  árbol y radiativos como función de los acoplamientos relevantes en
  el Lagrangiano del modelo.

\item Cálculo de autoestados de masas y ángulos de mezcla de neutrinos
  como función de los acoplamientos relevantes en el Lagrangiano del
  modelo.

\item Determinación del espacio de parámetros compatible con los datos
  experimentales sobre física de neutrinos a 3-sigma y de las cotas
  experimentales sobre los observables del modelo. Esto normalmente
  requiere la realización de programas numéricos de cálculo intensivo.

\item Los parámetros que inducen masas y mezclas de neutrinos pueden
  también ser los responsables de los decaimientos de las nuevas
  partículas predichas por el modelo. Como ocurre, por ejemplo, con la
  partícula supersimétrica más ligera en los modelos con violación
  bilineal de paridad R (VBPR). Por consiguiente es necesario calcular
  los branchings de decaimiento de esas nuevas partículas.

\item Cálculo de las secciones eficaces de producción de las nuevas
  partículas del modelo para comprobar si puede llegar a generarse la
  suficiente estadística en los aceleradores para que los decaimientos
  de las partículas nuevas lleguen a ser medibles.

\item Establecimiento de correlaciones entre cocientes de branchings
  de decaimientos de las nuevas partículas con la física de neutrinos.
  Esto permitiría por un lado descartar el modelo si alguna
  correlación no es respetada, o la posibilidad de reconstruir la
  matriz de masa de neutrinos si se pueden medir suficientes
  correlaciones.

\item Simulación de las se nales del modelo en aceleradores para
  determinar la viabilidad de medir los observables asociados con las
  correlaciones de física de neutrinos.
\end{enumerate}
Este método ya ha sido usado parcialmente en la literatura para
diferentes modelos.  En el modelo supersimétrico con VBPRr los pasos 1
a 6 ya han sido desarrollados en la literatura para diferentes
posibilidades de partícula supersimétrica más liviana.


Todos estos decaimientos están implementadas en una versión que aún no
es pública del programa SPHENO %\cite{hep-ph/0301101}, 
llamada
SPhenoRP, y de la cual disponemos gracias a su autor, y colaborador de
éste proyecto, Werner Porod. En este proyecto queremos abordar la
parte 7 del análisis para el decaimiento del neutralino, en
colaboración con los Grupos del profesor Jose Valle en Valencia, al
que pertenece el Dr Werner Porod y del Grupo de Oscar Eboli en Brasil.
En el modelo con VBPR se predice que el cociente entre el branching
del neutralino decayendo a $qq'\mu$ con el del neutralino decayendo a
$qq'\tau$ debe ser aproximadamente la tangente al cuadrado del ángulo de
mezcla atmosférico. Dicha correlación se sigue manteniendo
aproximadamente si el par de quarks provienen del $W$. De esta manera,
una desviación significativa de 1 para el cociente entre los
branchings del neutralino decayendo a $W\mu$ y $W\tau$, podría excluir al
modelo con VBPR como el único responsable de la generación de masas y
mezclas para los neutrinos.

Se calcularán los decaimientos del neutralino, usando el programa
SPhenoRP para un punto en el espacio de parámetros supersimétrico
representativo %\cite{Aguilar-Saavedra:2005pw} 
donde el neutralino es
la partícula supersimétrica más liviana. Se usará la salida del
programa en formato SLHA %\cite{hep-ph/0407067} 
con la información
sobre el espectro y los branchings supersimétricos, como archivo de
entrada de un programa que usa PYTHIA %\cite{hep-ph/0010017} para
simular el evento en el detector ATLAS del LHC. Cuando se genera un
evento con este programa las partículas supersimétricas producidas
sufren decaimientos en cadena hasta decaer a dos neutralinos y a
partículas del Modelo Estándar. Estos neutralinos a su vez decaen a
partículas del modelo estándar. En el programa se debe entonces
introducir una serie de cortes para comprobar si el evento contribuye
al numero de eventos de neutralinos decayendo a $W\mu$, $N_\mu$ , o a
$N_\tau$ . Se generan nuevos eventos aleatorios hasta alcanzar una
estadística similar a la esperada en el LHC, es decir de millones de
neutralinos producidos al alcanzar luminosidad de 100/fb. Se espera
que $N_\mu$ y $N_\tau$ sean del orden de miles para que el error en la
medida de sea suficientemente peque no. Esto podría garantizar que
dicho cociente pueda ser medido con suficiente significancia en el
LHC. Con el siguiente conjunto de cortes, que explota el hecho de que
el neutralino decae exhibiendo vértices desplazados
%\cite{hep-ph/0501153}, 
esperamos garantizar que los números de eventos
a $W\mu$ y a $W\tau$ sean suficientemente grandes. Se propone que un evento
contribuye a $N_\mu$ o $N_\tau$ s\'\i: el neutralino decae dentro de la región
de aceptancia del detector y exhibe un vértice desplazado a una
distancia suficientemente alejada del centro de interacción; Al menos
uno de los neutralinos decae a un $W$ acompa nado de un muon o un tau;
El $W$ decae a un par $qq'$; Si el $W$ está acompa nado por un tau, se
exige que el tau decaiga a través del canal a 1-prong; Los quarks
provenientes de $W$, as\'\i{} como el muon o el tau deben impactar
suficientes detectores de traza de modo que el vértice desplazado en
efecto pueda reconstruirse; Para reducir el background debido a
vértices desplazados provenientes de partículas del Modelo Estándar se
debe exigir que la masa invariante de las trazas que convergen al
vértice desplazado sea suficientemente grande.  Una vez obtenido el
número de eventos a $W \mu$, $N_\mu$, y el numero de eventos a $W \tau$, $N_\tau$, se
estimará el error en el cociente $N_\mu / N_\tau$ a partir de la raíz
cuadrada de la suma de los inversos de los números de eventos. Si este
error resulta ser suficientemente peque no como para que se pueda
establecer con suficiente significancia que es diferente de cero, se
podría establecer que dicho observable puede llegar a ser medible en
el LHC.

En el artículo internacional que se escribirá al respecto, se
mostrarán gráficos que ilustren el efecto de los cortes más
relevantes, y se hará énfasis en las ventajas que puede traer la
reconstrucción completa del neutralino en la determinación del
espectro completo supersimétrico. Algo que sólo se puede hacer en
casos muy específicos en el Modelo Estándar Mínimo Supersimétrico (con
conservación de paridad R).

Para el refinamiento y cuantificación de los cortes propuestos se
solicita una visita del investigador principal al Grupo de Valencia
donde se puede contar con la asesoría de miembros de la colaboración
ATLAS del LHC. Para la implementación de los cortes en el programa se
solicita la visita del Profesor Oscar Eboli de Brasil a la Universidad
de Antioquia.

De otro lado, la generación de millones de eventos requiere el uso de
técnicas de programación en paralelo que serán implementadas en los
cluster del Grupo y del Instituto de Física de la Universidad de
Antioquia.

Se solicita un viaje del investigador principal para presentar los
resultados en la conferencia SUSY 2007, la cual será muy importante
pues coincidirá con el inicio de operaciones del LHC previsto para ese
a no.

Para el modelo de Zee de generación radiativa de masas de neutrinos a
un loop, los pasos 1,2 ya han sido realizados en la literatura
%\cite{Balaji:2001ex}.  
Este modelo contiene un doblete de Higgs
adicional al del Modelo Estándar, y un Higgs cargado singlete de
SU(2). La versión que es consistente con física de neutrinos
corresponde al caso en el cual los dos dobletes de Higgs tienen
acoplamientos de Yukawa con todos los fermiones del Modelo Estándar.
De ésta forma el doblete Higgs adicional da lugar a nueve
acoplamientos tipo Yukawa en el Lagrangiano, mientras que el Higgs
cargado singlete de SU(2) da lugar a tres parámetros adicionales. El
potencial escalar contiene un acoplamiento entre el Higgs cargado
singlete y los dos dobletes que viola explícitamente número leptónico
y es la fuente para la generación de masas de Majorana de neutrinos a
un loop. Recientemente, en colaboración con el Grupo de Valencia,
hemos realizado los pasos 3 a 6, para una textura de la matriz de masa
de neutrinos en la cual la primera entrada diagonal es diferente de
cero, pero las otras dos son cero %\cite{AristizabalSierra:2006ri}. 
La
matriz de masa queda dependiendo de un parámetro global que se
determina a partir de las masas y mezclas de los higgses cargados del
modelo, y de siete acoplamientos tipo Yukawa del Lagrangiano. Esos
siete acoplamientos también determinan los decaimientos leptónicos de
los higgses cargados del modelo. En ese trabajo se considera la
producción y el decaimiento de los dos higgses cargados y se
establecen las correlaciones entre cocientes de branchings de
decaimiento de esos higgses cargados a leptones con los ángulos de
mezcla neutrinos. Se ha mostrado que para el caso en el cual el Higss
cargado más pesado es principalmente singlete, la matriz de masa se
puede reconstruir. Además se ha mostrado que la jerarquía de
decaimientos del Higgs cargado más liviano a leptones, puede ser muy
diferente a la de los modelos con dos dobletes de Higgs usuales.  En
el modelo estudiado el decaimiento del Higgs cargado a muon neutrino
puede dominar al decaimiento usual a tau neutrino.

En el presente proyecto pretendemos realizar los pasos 3 a 6 para el
caso general con todas las entradas de la matriz de masa de neutrinos
diferente de cero. En esta parte contaremos con la ayuda del
estudiante de doctorado de la Universidad de Antioquia Oscar Zapata, y
de la asesoría del Grupo de Valencia, en especial del estudiante de
doctorado de dicho Grupo Diego Aristizabal con quien hemos
desarrollado el trabajo anteriormente mencionado. Inicialmente se
realizará un estudio analítico similar al del trabajo anterior
considerando los otros dos casos posibles con dos entradas diagonales
iguales a cero, identificando las nuevas correlaciones entre
decaimientos de higgses cargados y los ángulos de mezcla de neutrinos
en esos casos. Esta parte requiere el uso del software Mathematica
para establecer los resultados analíticos. Luego se realizará un
análisis numérico de la matriz de masas de neutrino con todos las
entradas diferentes de cero para confrontar el rango de validez de las
correlaciones obtenidas. Cómo el espacio de parámetros es tan grande
en este caso, se requiere el uso de técnicas de programación en
paralelo que serán implementadas en el cluster del Grupo y el del
Instituto de Física de la Universidad de Antioquia.

Cómo los decaimientos leptónicos del Higgs neutro también están
controlados por los mismos parámetros que inducen las masas y mezclas
de neutrinos, se intentará establecer las posibles correlaciones entre
los decaimientos del Higgs neutro y los ángulos de mezcla de
neutrinos. Se solicita un computador para el estudiante de doctorado
para los cálculos analíticos y numéricos relacionados con ésta parte
del trabajo, as\'\i{} como la compra de una licencia de Mathematica para
instalarse en ese computador.

En el artículo internacional que se escriba al respecto se
confrontarán numéricamente las correlaciones entre los decaimientos de
partículas y la física de neutrinos más relevantes a través de
gráficos que muestren como la dispersión de puntos del espacio de
parámetros se acomodan a los resultados analíticos obtenidos. Además
se hará énfasis en las diferencias de las predicciones del modelo con
respecto a los modelos de dos dobletes de Higgs convencionales.

En el modelo de Babu, de generación radiativa de masas de neutrinos a
dos loops, los pasos 1 a 3 ya han sido realizado en la
literatura %\cite{Babu:2002uu}.  
Este modelo introduce un Higgs cargado
y un Higgs doblemente cargado adicional al contenido de partículas del
Modelo Estándar, los cuales son singletes de SU(2). El Higgs cargado
adicional da lugar a 3 acoplamientos tipo Yukawa en el Lagrangiano,
mientras que el Higgs doblemente cargado da lugar a seis parámetros
más. Además el potencial escalar contiene un término que viola
explícitamente número leptónico entre los nuevos higgses cargados y
que es la fuente para la generación de masas de Majorana para los
neutrinos a dos loops. En este proyecto pretendemos abordar los pasos
4 a 6 para el decaimiento del Higss doblemente cargado que se
introduce en dicho modelo. Un análisis preliminar parece mostrar que
la matriz de masa de neutrinos se puede reconstruir completamente a
partir de los observables asociados con el decaimiento de esa
partícula. Es decir que los nueve parámetros del sector de Yukawa y el
parámetro adicional con dimensiones de masa del término que viola
número leptónico en el potencial escalar se podrían determinar a
partir de los observables del modelo y sus correlaciones con la física
de neutrinos. Además se predice una jerarquía muy específica del
decaimiento del Higgs doblemente cargado a leptones cargados, lo que
podría resultar en una predicción de una se nal inequívoca del modelo
en aceleradores que tengan una producción suficiente para esta
partícula. El principal atractivo de este modelo es que predice un
valor del decaimiento muon a electrón fotón cercano a la cota
experimental. Pretendemos establecer si el modelo puede llegar a
excluirse debido a las mejoras esperadas para la cota sobre ese
decaimiento en los próximos a nos. En esta parte también se contará con
los dos estudiantes de doctorado Oscar Zapata y Diego Aristizabal.

En el artículo internacional que se escriba al respecto se procederá
de forma similar al del caso anterior. Se solicita una pasantía del
estudiante de doctorado al Grupo de Valencia para seleccionar los
resultados más importantes que aparecerán en ambos artículos.

Los modelos 3-3-1, contienen los ingredientes para los mecanismos más
comunes de generación de masas de neutrinos: el mecanismo see-saw,
(masas de Dirac y Majorana a nivel árbol), y mecanismos de generación
radiativa a uno y dos loops %\cite{Dong:2006vk}. 
En este proyecto nos enfocaremos en
versiones de modelos 3-3-1 donde, gracias a simetrías discretas
apropiadas %\cite{Fanchiotti:2006jq}., 
las masas de neutrinos sólo se generen a dos loops, y
por consiguiente resulten naturalmente suprimidas respecto a las masas
de los fermiones cargados del modelo. Para este caso abordaremos los
pasos 1 a 6, con énfasis en los pasos iniciales, pues creemos que hay
contribuciones que aún no se han considerado en la literatura. En
modelos 3-3-1 con neutrinos derechos la matriz de neutrinos a baja
energía resultante podría ser 6x6 si la escala de los neutrinos de
Majorana es suficientemente peque na. En este caso es necesario
reinterpretar los datos sobre física de neutrinos para los seis
neutrinos ligeros resultantes, incluyendo los efectos de materia en el
ángulo de mezcla solar.

Para abordar la parte 1 es necesario establecer los acoplamientos
relevantes del Lagrangiano que contribuirán a la generación de masas
radiativas para los neutrinos, as\'\i{} como la identificación de los
decaimientos de partículas nuevas que también estén controlados por
esos parámetros. En este parte aprovecharemos la experiencia del
profesores William Ponce y Luis Alberto Sánchez en este tipo de
modelos, ver por ejemplo %\cite{Ponce:2002sg}.
 Contaremos también con
el apoyo del estudiante de Maestría de la Universidad de Antioquia
Richard Hamilton a quien se le darán cursos electivos de la Maestría,
sobre las técnicas necesarias para el cálculo de loops en Teorías
Cuántica de Campos. Para ello se solicitan los libros detallados en la
Bibliografía. También se solicita un computador para el profesor Luis
Alberto Sánchez.

En el artículo internacional inicial que se escriba al respecto se
hará énfasis en la generalidad y los detalles relacionados con el
cálculo de las masas radiativas de neutrinos, as\'\i{} como en la validez
de la interpretación de las oscilaciones de neutrinos en términos de
neutrinos tipo pseudodirac. Se solicita la visita del estudiante de
doctorado Diego Aristizabal que pertenece al Grupo de Valencia para
aprovechar su experiencia en el calculo de los loops que dan lugar a
las masas de neutrinos en los modelos 3-3-1, as\'\i{} como su ayuda en el
establecimiento de las correlaciones necesarias en el caso del Modelo
de Zee, de Babu, y de los modelos 3-3-1.

Para administrar el cluster del Grupo, mantener los servidores del
Grupo que permitan una comunicación fluida con nuestros asesores
internacionales y asesorar en las tareas de programación, se requiere
un administrador de sistemas durante la duración del proyecto. As\'\i{}
mismo y como se detalla en la bibliografía se solicitan libros de
programación en Fortran y Fortran 90, que nos sirvan como referencia
en la implementación de los programas numéricos.


Todos estos decaimientos están implementadas en una versión que aún no
es pública del programa SPHENO %\cite{hep-ph/0301101}, 
llamada
SPhenoRP, y de la cual disponemos gracias a su autor, y colaborador de
éste proyecto, Werner Porod. En este proyecto queremos abordar la
parte 7 del análisis para el decaimiento del neutralino, en
colaboración con los Grupos del profesor Jose Valle en Valencia, al
que pertence el Dr Werner Porod y del Grupo de Oscar Eboli en Brasil.
En el modelo con VBPR se predice que el cociente entre el branching
del neutralino decayendo a $qq'\mu$ con el del neutralino decayendo a
$qq'\tau$ debe ser aproximadamente la tangente al cuadrado del ángulo de
mezcla atmosférico. Dicha relación se sigue manteniendo
aproximadamente si el par de quarks provienen del $W$. De
esta manera, una desviación significativa de 1 para el cociente entre
los branchings del neutralino decayendo a $W\mu$ y $W\tau$, podría
excluir al modelo con VBPR como el único responsable de la generación
de masas y mezclas para los neutrinos.

Se calcularán los decaimientos del neutralino, usando el programa
SPhenoRP para un punto en el espacio de parámetros supersimétrico
representativo %\cite{Aguilar-Saavedra:2005pw} 
donde el neutralino es
la partícula supersimétrica más liviana. Se usará la salida del
programa en formato SLHA %\cite{hep-ph/0407067} 
con la información
sobre el espectro y los branchings supersimétricos, como archivo de
entrada de un programa que usa PYTHIA %\cite{hep-ph/0010017} 
para
simular el evento en el detector ATLAS del LHC. Cuando se genera un
evento con este programa las partículas supersimétricas
producidas sufren decaimientos en cadena hasta decaer a dos
neutralinos y a partículas del Modelo Estándar. Estos neutralinos
a su vez decaen a partículas del modelo estándar.  En el programa
se debe entonces introducir una serie de cortes para comprobar si el
evento contribuye al numero de eventos de neutralinos decayendo a
$W\mu$, $N_\mu$, o a $N_\tau$. Se generan nuevos eventos aleatorios hasta
alcanzar una estadística similar a la esperada en el LHC, es
decir de millones de neutralinos producidos al alcanzar luminosidad de
100/fb.  Se espera que $N_\mu$ y $N_\tau$ sean del orden de miles para que
el error en la medida de $N_\mu/N_\tau$ sea suficientemente peque no. Esto
podría garantizar que dicho cociente puede ser medido con
suficiente significancia en el LHC. Con el siguiente conjunto de
cortes, que explota el hecho de que el neutralino decae exhibiendo
vértices desplazados %\cite{hep-ph/0501153},
 esperamos garantizar que
los números de eventos a $W\mu$ y a $W\mu$ sean suficientemente grandes.
Se propone que un evento contribuye a $N_\mu$ o $N_\tau$ s\'\i:
\begin{itemize}
\item el neutralino decae dentro de la región de aceptancia del
  detector y exhibe un vertice desplazado a una distancia
  suficientemente alejada del centro de interacción;
\item Al menos uno de los neutralinos decae a un $W$ acompa nado de un
  muon o un tau;
\item El $W$ decae a un par q-q'
\item Si esta acompa nado por un tau que a su vez decae a través del
  canal a 1-prong.
\item Los quarks provenientes de W, as\'\i{} como el muon o el tau deben
  impactar suficientes detectores de traza de modo que el vértice
  desplazado en efecto pueda reconstruirse
\item Para reducir el background debido a vértices desplazados
  provenientes de partículas del Modelo Estándar se debe exigir que la
  masa invariante de las trazas que convergen al vértice desplazado
  sea suficientemente grande
\end{itemize}
Una vez obtenido el número de eventos a $W$ muon, $N_\mu$, y el numero
de eventos a $W$ tau, $N_\tau$, se estimará el error en el cociente
$N_\mu/N_\tau$ a partir de la raíz cuadrada de la suma de los inversos de
los números de eventos. Si este error resulta ser suficientemente
peque no como para que $N_\mu/N_\tau$ se pueda establecer con suficiente
significancia que es diferente de cero, se podría establecer que dicho
observable puede llegar a ser medible en el LHC.

En el artículo internacional que se escribirá al respecto, se
mostrarán gráficos que ilustren el efecto de los cortes más
relevantes, y se hará énfasis en las ventajas que puede traer la
reconstrucción completa del neutralino en la determinación del
espectro completo supersimétricos. Algo que sólo se puede hacer en
casos muy específicos en el Modelo Estándar Mínimo Supersimétrico
(con conservación de paridad R)

Para el refinamiento y cuantificación de los cortes propuestos se
solicita una visita del investigador principal al Grupo de Valencia
donde se puede contar con la asesoría de miembros de la colaboración
ATLAS del LHC.  Para la implementación de los cortes en el programa se
solicita la visita del Profesor Oscar Eboli de Brasil a al Universidad
de Antioquia.

De otro lado, la generación de millones de eventos requiere el uso de
técnicas de programación en paralelo que serán implementadas en los
cluster del Grupo y del Instituto de Física de la Universidad de
Antioquia

Se solicita una viaje del investigador principal para presentar los
resultados en la conferencia SUSY 2007, la cual será muy importante
pues coincidirá con el inicio de operaciones del LHC previsto para ese
a no.

Para el modelo de Zee de generación radiativa de masas de neutrinos a
un loop, los pasos 1,2 ya han sido realizados en la literatura
%\cite{Balaji:2001ex}. 
Este modelo contiene un doblete de Higgs
adicional al del Modelo Estándar, y un Higgs cargado singlete de
SU(2). La versión que es consistente con física de neutrinos
corresponde al caso en el cual los dos dobletes de Higgs tienen
acoplamientos de Yukawa con todos los fermiones del Modelo Estándar.
De ésta forma el doblete Higgs adicional da lugar a nueve
acoplamientos tipo Yukawa en el Lagrangiano, mientras que el Higgs
cargado singlete de SU(2) da lugar a tres parámetros adicionales. El
potencial escalar contiene un acoplamiento entre el Higgs cargado
singlete y los dos dobletes que viola explícitamente número leptónico
y es la fuente para la generación de masas de Majorana de neutrinos a
un loop. Recientemente, en colaboración con el Grupo de Valencia,
hemos realizado los pasos 3 a 6, para una textura de la matriz de masa
de neutrinos en la cual la primera entrada diagonal es diferente de
cero, pero las otras dos son cero %\cite{AristizabalSierra:2006ri}. 
La
matriz de masa queda dependiendo de un parámetro global que se
determina a partir de las masas y mezclas de los higges cargados del
modelo, y de siete acoplamientos tipo Yukawa del Lagrangiano. Esos
siete acoplamientos también determinan los decaimientos leptónicos de
los higgses cargados del modelo. En ese trabajo se considera la
producción y el decaimiento de los dos higgses cargados y se
establecen las correlaciones entre cocientes de branchings de
decaimiento de esos higgses cargados a leptones con los ángulos de
mezcla neutrinos. Se ha mostrado que para el caso en el cual el Higss
cargado más pesado es principalmente singlete, la matriz de masa se
puede reconstruir. Además se ha mostrado que la jerarquía de
decaimientos del Higgs cargado más liviano a leptones, puede ser muy
diferente a la de los modelos con dos dobletes de Higgs usuales.  En
el modelo estudiado el decaimiento a muon neutrino puede dominar al
decaimiento usual a tau neutrino.


En el presente proyecto pretendemos realizar los pasos 3 a 6 para el
caso general con todas las entradas de la matriz de masa de neutrinos
diferente de cero. En esta parte contaremos con la ayuda del
estudiante de doctorado de la Universidad de Antioquia Oscar Zapata, y
de la asesoría del Grupo de Valencia, en especial del estudiante de
doctorado de dicho Grupo Diego Aristizabal con quien hemos
desarrollado el trabajo anteriormente mencionado.  Inicialmente se
realizará un estudio analítico similar al del trabajo anterior
considerando los otros dos casos posibles con dos entradas diagonales
iguales a cero, identificando las nuevas correlaciones entre
decaimientos de higgses cargados y los ángulos de mezcla de neutrinos
en esos casos. Esta parte requiere el uso del software Mathematica
para establecer los resultados analíticos.  Luego se realizará un
análisis numérico de la matriz de masas de neutrino con todos las
entradas diferentes de cero para confrontar el rango de validez de las
correlaciones obtenidas. Cómo el espacio de parámetros es tan grande
en este caso, se requiere el uso de técnicas de programación en
paralelo que serán implementadas en el cluster del Grupo y del
Instituto de Física de la Universidad de Antioquia.

Cómo los decaimientos leptónicos del Higgs neutro también están
controlados por los mismos parámetros que inducen las masas y mezclas
de neutrinos, se intentará establecer las posibles correlaciones entre
los decaimientos del Higss neutro y los ángulos de mezcla de
neutrinos. Se solicita un computador para el estudiante de doctorado
para los cálculos análiticos y numéricos relacionados con ésta parte
del trabajo, as\'\i{} como la compra de una licencia de Mathematica para
instalarse en ese computador.

En el artículo internacional que se escriba al respecto se
confrontarán numéricamente las correlaciones entre los decaimientos de
partículas y la física de neutrinos más relevantes a través de
gráficos que muestren como la dispersión de puntos del espacio de
parámetros se acomodan a los resultados analíticos obtenidos. Además
se hará énfasis en las diferencias de las predicciones del modelo con
respecto a los modelos de dos dobletes de Higgs convencionales. 


En el modelo de Babu, de generación radiativa de masas de neutrinos a
dos loops, los pasos 1 a 3 ya han sido realizado en la literatura
%\cite{Babu:2002uu}. 
Este modelo introduce un Higgs cargado y un Higgs
doblemente cargado adicional al contenido de partículas del
Modelo Estándar, los cuales son singletes de SU(2). El Higgs cargado
adicional da lugar a 3 acoplamientos tipo Yukawa en el Lagrangiano,
mientras que el Higgs doblemente cargado da lugar a seis parámetros
más.  Además el potencial escalar contiene un término que viola
explícitamente número leptónico entre los nuevos higgses cargados
y que es la fuente para la generación de masas de Majorana para los
neutrinos a dos loops.  En este proyecto pretendemos abordar los pasos
4 a 6 para el decaimiento del Higss doblemente cargado que se
introduce en dicho modelo. Un análisis preliminar parece mostrar que
la matriz de masa de neutrinos se puede reconstruir completamente a
partir de los observables asociados con el decaimiento de esa
partícula. Es decir que los nueve parámetros del sector de Yukawa
y el parámetro adicional con dimensiones de masa del término que viola
número leptónico en el potencial escalar se podrían determinar a
partir de los observables del modelo y sus correlaciones con la
física de neutrinos. Además se predice una jerarquía muy
específica del decaimiento del Higgs doblemente cargado a
leptones cargados, lo que podría resultar en una predicción de
una se nal inequívoca del modelo en aceleradores que tengan una
producción suficiente para esta partícula. El principal atractivo
de este modelo es que predice un valor del decaimiento electrón a muon
gamma cercano a la cota experimental. Pretendemos establecer si el
modelo puede llegar a excluirse debido a las mejoras esperadas para la
cota sobre ese decaimiento en los próximos a nos. En esta parte también
se contará con los dos estudiantes de doctorado Oscar Zapata y Diego
Aristizabal.

En el artículo internacional que se escriba al respecto se procederá
de forma similar al del caso anterior. Se solicita una pasantía del
estudiante de doctorado al Grupo de Valencia para seleccionar los
resultados más importantes que aparecerán en ambos artículos.


Los modelos 3-3-1, contienen los ingredientes para los mecanismos más
comunes de generación de masas de neutrinos: el mecanismo see-saw,
(masas de Dirac y Majorana a nivel árbol), y mecanismos de generación
radiativa a uno y dos loops %\cite{Dong:2006vk}. 
En este proyecto nos
enfocaremos en versiones de modelos 3-3-1 donde, gracias a simetrías
discretas apropiadas %\cite{Fanchiotti:2006jq}, 
las masas de neutrinos
sólo se generen a dos loops, y por consiguiente resulten naturalmente
suprimidas respecto a las masas de los fermiones cargados del modelo.
Para este caso abordaremos los pasos 1 a 6, con énfasis en los pasos
iniciales, pues creemos que hay contribuciones que aún no se han
considerado en la literatura. En modelos 3-3-1 con neutrinos derechos
la matriz de neutrinos a baja energía resultante podría ser 6x6 si la
escala de los neutrinos de Majorana es suficientemente peque na. En
este caso es necesario reinterpretar los datos sobre física de
neutrinos para los seis neutrinos ligeros resultantes, incluyendo los
efectos de materia en el ángulo de mezcla solar. 

Para abordar la parte 1 es necesario establecer los acoplamientos
relevantes del Lagrangiano que contribuirán al generación de masas
radiativas para los neutrinos, as\'\i{} como la identificación de los
decaimientos de partículas nuevas que también estén controlados
por esos parámetros. En este parte aprovecharemos la experiencia del
profesores William Ponce y Luis Alberto Sánchez en este tipo de
modelos, ver por ejemplo %\cite{Ponce:2002sg}.
Contaremos también con
el apoyo del estudiante de Maestría de la Universidad de
Antioquia Richard Hamilton a quien se le darán cursos electivos de la
Maestría, sobre las técnicas necesarias para el cálculo de loops
en Teorías Cuántica de Campos.  Para ello se solicitan los libros
detallados en la Bibliografía.  También se solicita un computador
para el profesor Luis Alberto Sánchez


En el artículo internacional inicial que se escriba al respecto se
hará énfasis en la generalidad y los detalles relacionados con el
cálculo de las masas radiativas de neutrinos, as\'\i{} como en la validez
de la interpretación de las oscilaciones de neutrinos en términos de
neutrinos tipo pseudodirac.


Se solicita la visita del estudiante de doctorado Diego Aristizabal
que pertenece al Grupo de Valencia para aprovechar su experiencia en
el calculo de los loops que dan lugar a las masas de neutrinos en los
modelos 3-3-1, as\'\i{} como su ayuda en el establecimiento de las
correlaciones necesarias en el caso del Modelo de Zee, de Babu, y de
los modelos 3-3-1.



Para administrar el cluster del Grupo, mantener los servidores del
Grupo que permitan una comunicación fluida con nuestros asesores
internacionales y asesorar en las tareas de programación se requiere
un administrador de sistemas durante la duración del proyecto.

As\'\i{} mismo y como se detalla en la bibliografía se solicitan libros de
programación en Fortran y Fortran 90, que nos sirvan como referencia
en la implementación de los programas numéricos.



%%% Local Variables: 
%%% mode: latex
%%% TeX-master: "main"
%%% End: 
