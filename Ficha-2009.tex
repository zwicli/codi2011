%Solicitud contrapartida de proyecto de COLCIENCIAS
\documentclass[11pt]{article}
\usepackage[spanish]{babel}
%Maldito arial 12
\usepackage[utf8]{inputenc}
%changing the deafult fonts
\usepackage{textcomp}
\renewcommand{\rmdefault}{phv}
\renewcommand{\sfdefault}{pag}
\renewcommand{\ttdefault}{pcr}

\usepackage{amsmath}
\usepackage{amssymb}
%\usepackage{bm}
\usepackage{graphicx}
\usepackage{colortbl}
\usepackage{array}
\usepackage{xcolor}

\usepackage{multirow}
\usepackage{tabularx}


\usepackage[nomessages]{fp}
\usepackage{cancel}
\usepackage[colorlinks=true]{hyperref}
%\usepackage[square,numbers]{natbib}
%\setcitestyle{notesep={:}}
\usepackage[punctsep]{collref}

\setlength{\oddsidemargin}{0pt}
\setlength{\evensidemargin}{0pt}
\setlength{\topmargin}{-1cm}
\setlength{\textheight}{21.5cm}
\setlength{\textwidth}{17cm}

\newcolumntype{Y}{>{\centering\arraybackslash}X}

\begin{document}

\vglue -2cm
\renewcommand{\arraystretch}{1.28}
\noindent
\hspace{-0.5cm}\begin{tabularx}{1.059\linewidth}{|c|Y|c|c|c|}\hline
 \multirow{4}{*}{\parbox[b]{3cm}{\vspace{0.1cm}\centering\includegraphics[scale=0.3]{udea}\\
\scriptsize{{\fontfamily{ptm}\selectfont Universidad de Antioquia}}\\
\tiny{{\fontfamily{ptm}\selectfont Vicerrectoría de Investigación}}
}
}  & \textbf{\large VICERRECTORÍA DE INVESTIGACIÓN} & \multicolumn{3}{c|}{{\large Actualización}}\\
\cline{2-5}
 &\textbf{FICHA TÉCNICA}  & {\large Día} & {\large Mes} & {\large Año}\\
\cline{3-5}
 &  &  &  & \\[-13pt]
 & \textbf{\small PROYECTOS DE INVESTIGACIÓN} &  &  & \\[-5pt]
 & \textbf{2009} &  &  & \\
\hline
\end{tabularx}
%
\renewcommand{\arraystretch}{1}

\bigskip{}
\noindent{}
La ficha técnica es el resumen del proyecto, con una extensión máxima de seis páginas sin contar anexos. El investigador debe tener especial cuidado para que la ficha técnica contenga la información básica necesaria acerca del proyecto, puesto que el jurado final solo recibe la mencionada ficha. El proyecto de investigación quedará en archivo en el centro de investigaciones.



\begin{center}
  \textbf{\large INFORMACIÓN GENERAL}
\end{center}

\renewcommand{\arraystretch}{1.71}
\noindent
\hspace{-0.5cm}\begin{tabularx}{1.059\linewidth}{|l|l|}\hline
\multicolumn{2}{|Y|}{\cellcolor[gray]{.8}\textbf{Implicaciones de una simetría $U(1)_H$ en experimentos de rayos cósmicos y de aceleradores}}\\\hline
\multicolumn{2}{|Y|}{}\\\hline
\cellcolor[gray]{.8}\textbf{Facultad, Escuela o Instituto}&Facultad de Ciencias Exactas y Naturales\\\hline
\cellcolor[gray]{.8}\textbf{Grupo que avala el proyecto}& Fenomenología de Interacciones Fundamentales (GFIF)\\\hline
\cellcolor[gray]{.8}\textbf{Código Colciencias del Grupo si aplica}&COL0008423\\\hline
\cellcolor[gray]{.8}\textbf{Nombre del Investigador Principal}&Diego A. Restrepo Qunitero\\\hline
\cellcolor[gray]{.8}\textbf{Nombre del Coinvestigador}&William Ponce\\\hline
\cellcolor[gray]{.8}\textbf{Documento de Identidad N${}^{\text{\textbf{o}}}$}&98554575\\\hline
\cellcolor[gray]{.8}\textbf{Convocatoria}&Sostenibilidad 2009-2010 de GFIF\\\hline
\cellcolor[gray]{.8}\textbf{Duración del proyecto (en meses)}&18 meses\\\hline
\end{tabularx}

\vspace{-0.02cm}
\renewcommand{\arraystretch}{1.038}
\noindent
\hspace{-0.5cm}\begin{tabularx}{1.059\linewidth}{|l|r|}
\multicolumn{2}{|Y|}{\cellcolor[gray]{.8}\textbf{DATOS DEL PRESUPUESTO}}\\[7pt]\hline
\textbf{Recursos propios del Grupo}\hspace{6.2cm}\quad&{\$ 14,000,000,\footnotesize{00}}\\\hline
\textbf{Universidad} \footnotesize{(recursos en especie y recursos  frescos)}&{\$ 43,750,000\footnotesize{00}}\\\hline
\textbf{Otras entidades} \footnotesize{(si aplica):}&{\$ 0,\footnotesize{00}}\\\hline
\textbf{Valor total del proyecto:}&{\$ 57,750,000,\footnotesize{00}}\\\hline
\end{tabularx}
\renewcommand{\arraystretch}{1}

\begin{center}
  \textbf{\large DESCRIPCIÓN DEL PROYECTO}
\end{center}

% \bigskip
% \begin{itemize}

%    \item
% {\bf Título del Proyecto } \\
% Computing  Yukawa couplings for Up quarks and neutrinos in $SU(5)\times U(1)$

%    \item
% {\bf Nombre Investigador Principal }\\  Diego Restrepo.

%    \item
% {\bf Teléfono y Dirección Electrónica  }\\ Tel.: +57-4-2105635 \quad
% E-mail: {\tt restrepo@udea.edu.co }

%    \item
% {\bf Nombre del Grupo de Investigación }\\
% Grupo de  Fenomenología de Interacciones Fundamentales. \\
% El Grupo apoya los programas de posgrado de Maestría y de Doctorado
% del Instituto de Física.

%    \item
% {\bf Categoría del grupo en la convocatoria Colciencias}\\
% Grupo A.


% \item \textbf{Indicar si el grupo aplicó a estrategia de sostenibilidad 2005-2006 y cual fue el resultado}\\
%  El grupo aplicó a la sostenibilidad 2005-2006 con los siguientes resultados:
%  \begin{itemize}
%  \item Número de publicaciones en revistas internacionales: 12
%  \item Financiación Externa: 2 proyectos de Colciencias por valor de 220 millones
%  \item Estudiantes de Maestría Graduados: 4
%  \item Estudiantes de Doctorado Graduados: 1
%  \end{itemize}


% \item
%   {\bf Nombres de quienes desarrollarán el Proyecto }\\
%   Diego Restrepo (Investigador Principal)
%   Enrico Nardi (Co-investigador); \\
%   N.N (Administrador de Sistemas). % y Auxiliar de Administración).


% \item \textbf{Programa de maestría o doctorado que el  grupo apoya}\\
% Maestría y Doctorado en Física.
%    \item
% {\bf Estudiante(s) Nombre, Carné, Programa de Estudios  }\\
% N.N. Maestría en Física. Corresponde a un estudiante nuevo de la maestría que ingrese para el semestre 2008-1 en la cohorte 27. \\


% % \smallskip\noindent (Este proyecto involucra oficialmente el estudiante de
% % Doctorado Jorge I. Zuluaga, el cual ya tiene los recursos economicos de una
% % beca de Doctorado de la Universidad de Antioquia.  Sin embargo, por el
% % desarrollo del Proyecto se involucrará otro estudiante de Doctorado al
% % cual se brindará una aiuda economica con los recursos del proyecto.)

%    \item
% {\bf Duración del Proyecto: } \qquad \quad \  12 meses.

%    \item
% {\bf Valor total del Proyecto: } \qquad  49,000,000 de pesos.

%    \item {\bf Valor aprobado por otra entidad: } \qquad  No Aplica.

%    \item
% {\bf Valor Solicitado al CODI: } \qquad   25,000,000 de pesos.

% \end{itemize}

% %%%%%%%%%%%%%%%%%%%%%%%%%%%%%%%%%%%%%%%%%%%%%%%%%%%%%%%%

% %\newpage

% % \pagenumbering{arabic}

% % \small



\bigskip

\section{ Planteamiento del Problema }

\subsection{Marco teórico}

Con la entrada en operación del LHC, se espera descubrir señales de materia oscura y de modelos que tengan algún mecanismo radiactivo para generar masas de neutrinos.

Los futuros experimentos de rayos cósmicos, tales como el experimento AMS-02, Planck, IceCube, etc, esperan mejorar la sensibilidad a este tipo de señales. Así mismo, los experimentos de detección directa de materia oscura como DAMA, CDMS, CoGeNT, XENON han creado grandes expectativas por los resultados presentes y los que se espera obtener en los próximos meses. Los experimentos de física de neutrinos y de radiación cósmica de fondo han entrado ya en una era de medidas de precisión.
Lo que proponemos en este proyecto es tratar de responder las siguiente pregunta: ¿Cuáles serán las restricciones impuestas por los recientes y futuros datos de todos estos experimentos sobre  modelos que presenten una partícula candidata a materia oscura o que generen masas para los neutrinos?
Los recientes datos reportados por las colaboraciones PAMELA y Fermi han impuesto restricciones muy fuertes sobre el flujo de rayos cósmicos provenientes de partículas de materia oscura. Estas restricciones afectarán directamente el espacio de parámetros permitido por los experimentos actuales. Adicional a estas nuevas restricciones, también se debe tener en cuenta la sensibilidad de los futuros experimentos de rayos cósmicos, tales como el experimento AMS-02, Planck, IceCube, etc.

Por otro lado, con la entrada en operación del LHC junto con el tevatrón, se esperan descubrir señales del bosón de Higgs, materia oscura, dimensiones extra, supersimetría, etc. En algunos modelos más allá del Modelo Estándar se tienen predicciones concretas sobre los posibles productos de aniquilación y desintegración del candidato a materia oscura que podrán ser detectados.

Actualmente se encuentra funcionado el acelerador Tevatron en Fermilab en Estados Unidos. Este acelera protones y antiprotones a una energía de 1 TeV, y se espera que siga funcionado por 2 años más. Ya ha entrado en funcionamiento el acelerador LHC en el CERN en la frontera entre Francia y Suiza. Aunque de momento se encuentra en estado de pruebas funcionando a energias que alcanzarán los 7TeV, en un futuro cercano este  acelerará protones en ambos sentidos a una energía que alcanzará hasta los 14 TeV. La mayoría de los experimentos conducidos son hechos en grandes instrumentos llamados detectores. Por ejemplo el LHC tiene 4 detectores. En dos de ellos, ATLAS y CMS, hay grupos colombianos participando. En los últimos años el Grupo ha tenido una estrecha vinculación con el Grupo de Marta Losada que esta participando en la colaboración ATLAS.

A nivel de detección de radiación cósmica tenemos el satélite PAMELA para detección de rayos cósmicos de electrones, positrones, antiprotones, y núcleos livianos. En agosto del año pasado PAMELA reporto una abundancia anómala de positrones %\cite{pamela},
que ha dado lugar a un sinnúmero de publicaciones tratando de explicar su origen.  Fermi-LAT es otro satélite para la detección de rayos cósmicos que también estará funcionando  durante los próximos años. En sus primeras medidas %\cite{fermi}
también ha reportado una abundancia anómala de positrones y electrones. El detector de rayos cósmicos AMS-02~%\cite{ams:2009},
que será instalado en la Estación Espacial  internacional, medirá el espectro de antiprotones y positrones en un rango de energía mucho más amplio y con una estadística mucho mejor. En cuanto a detección directa de materia oscura existen grandes expectivas por los resultados presentes y los que se espera obtener en los próximos meses %... \url{http://resonaances.blogspot.com/2010/03/another-experiment-sees-dark-matter.html}

En los próximos años entrará el satélite Planck que fue lanzado en mayo, y desde julio  se encuentra en pleno funcionamiento. Con los datos obtenidos de este experimento se espera medir la radiación cósmica de fondo con mucha más precisión que su antecesor WMAP. Además esta el experimento Auger en Argentina que detecta rayos cósmicos de ultra alta energía que alcanzan a llegar a la superficie terrestre.

También hay varios experimentos de física de neutrinos en marcha que esperan mejorar los datos actuales de oscilaciones de neutrinos, determinar el ángulo de mezcla $\theta_{13}$, del cual actualmente solo hay cotas, y si este resulta ser diferente de cero, buscar evidencia de violación de CP en el sector de neutrinos, un ingrediente crucial para generar bariogénesis a través de leptogénesis.

Esto representa un escenario muy dinámico donde el Grupo debe estar preparado para interpretar cualquier evidencia que resulte del Higgs o nueva física a la luz de los modelos que viene desarrollando.
	
La pregunta que queremos abordar en el presento proyecto es si:
¿se podrá establecer en los aceleradores futuros cual es el modelo que
explica la generación de masas y mezclas para los neutrinos?

%%Jose david
Por medio del estudio del modelo de seesaw radiactivo se predicen algunos canales de decaimiento para la materia oscura que podrán ser observados en el LHC. Si se llegan a detectar dichos canales sabremos en que modelo la generación de masas de los neutrinos se da, y a través de que mecanismo propuesto.

Aunque se ha encontrado que los neutrinos tienen masas a la escala de
sub-eV y se mezclan %\cite{hep-ph/0405172},
con experimentos sólo sobre
física de neutrinos no se puede establece cual es el modelo que contiene
el mecanismo apropiado de generación de masas y mezcla de neutrinos.
En particular, será muy difícil establecer la naturaleza de los
neutrinos, es decir si tienen masa de Dirac o Majorana.

Los aceleradores de partículas tienen la posibilidad de
establecer cual es el mecanismo de generación de masas y mezclas de
neutrinos y de forma indirecta establecer entonces la naturaleza de
los mismos. El mecanismo seesaw %\cite{Gell-Mann:vs}
ha sido la
explicación más popular para la pequeñez de las masas de neutrinos. Es
elegante y simple, y depende sólo de análisis dimensional para la
nueva física que se requiere. Los datos actuales de neutrinos
apuntan a una escala de seesaw del orden $10^{10}$ - $10^{15}$ GeV,
donde la violación de número leptónico ocurre a través de masas de
Majorana de los neutrinos derechos. Con una escala tan alta, los
efectos de violación de sabor leptónico en procesos diferentes a la
oscilación de neutrinos son extremadamente pequeños. Por ejemplo el
decaimiento del muon a electrón fotón en extensiones seesaw del Modelo
Estándar, es decenas de ordenes de magnitud menor a la cota actual.

Una alternativa al mecanismo seesaw que también explica naturalmente
la pequeñez de la masa de neutrinos son los mecanismos de generación
radiativa, ver por ejemplo los siguientes artículos recientes y
sus referencias
%\cite{hep-ph/0503059,AristizabalSierra:2006ri,Babu:2002uu,Dong:2006vk}.
En esta aproximación las masas de neutrinos son cero a nivel árbol y
son inducidas únicamente como correcciones radiativas finitas.  Estas
correcciones radiativas son proporcionales a la raíz cuadrada de
las masas del leptón cargado (o quark) divididas por la escala de nueva
física $\Lambda$. Las masas de los neutrinos pueden estar en el rango
de sub-eV aún cuando la escala de nueva física sea del orden de
TeV. En este caso, los procesos de violación de número leptónico en
otros procesos diferentes a la física de neutrinos pueden llegar
a ser accesibles a los aceleradores.

En este proyecto proponemos analizar en detalle las consecuencias
experimentales de modelos específicos donde las masas de
neutrinos se generan radiativamente. En particular estamos interesados
en calcular los decaimientos de las partículas del modelo que
están controlados por los mismos parámetros que inducen las masas y
las mezclas de los neutrinos. Estos decaimientos podrán no sólo
ayudar a determinar el mecanismo de generación de masas de neutrinos
con aceleradores de partículas, sino también ayudar a determinar
experimentalmente los parámetros del modelo que aparecen en la matriz
de masa de neutrinos.

Los recientes datos reportados por las colaboraciones PAMELA y Fermi-LAT
han señalado un exceso en el flujo de positrones y en el flujo total de
electrones y positrones respectivamente, mientras para el flujo de antiprotones no se
ha observado exceso alguno. El exceso en el flujo de antimateria puede ser
interpretado como evidencia de nuevas fuentes astrofísicas de rayos cósmicos de
alta energía en nuestra galaxia, tales como pulsares o remanentes de supernovas.
En este escenario, el flujo de positrones y rayos gamma astrofísicos constituirán
un nuevo background para las posibles señales de detección de materia oscura o para las posibles zonas
de exclusión de los diferentes modelos que predicen la existencia de materia oscura. En particular,
los background de los flujos de positrones, antiprotones y rayos gamma afectarán directamente
el espacio de parámetros permitido por los experimentos hasta ahora de los modelos
supesimétricos con violación de paridad R extendidos con una simetría horizontal
$U(1)_H$ donde el neutralino es candidato a materia oscura inestable. Adicional a
estas nuevas restricciones, también se debe tener en cuenta la sensibilidad de
los futuros experimentos de rayos cósmicos, tales como el experimento AMS-02.

Por otro lado, con la entrada en operación del LHC junto con el tevatrón,
se esperan descubrir señales del bosón de Higgs, materia oscura, dimensiones extra,
supersimetría, entre muchas otras señales.  En modelos supersimétricos con violación de paridad R
extendidos con una simetría horizontal $U(1)_H$ donde el neutralino decae rápidamente vía operadores
que violan número bariónico, se tendrán predicciones concretas sobre los posibles productos de
desintegración del neutralino que podrán ser detectados en el LHC o tevatrón.

Con base en lo planteado anteriormente, lo que proponemos en este proyecto es
tratar de responder la siguiente pregunta: ¿Cuáles serán las restricciones
impuestas por los recientes y futuros datos astrofísicos y de aceleradores sobre
los modelos supersimétricos con violación de paridad R extendidos con una simetría $U(1)_H$ que presentan un neutralino inestable?

\subsection{Planteamiento del problema}


En el caso de modelos supersimétricos basados una simetría de sabor anómala $U(1)_H$, con un solo flavón de carga $-1$, una vez se introducen las condiciones teóricas y fenomenológicas, la solución más óptima se puede expresar en términos de 4 cargas $H$ libres que se puede usar para explicar el valor de los 45 parámetros del superpotencia que violan $R$--paridad \cite{Mira:2000gg,Dreiner:2003hw,Dreiner:2003yr,Dreiner:2007vp,Dreiner:2006xw,Sierra:2009zq}.
\begin{equation}
  \label{eq:5}
  W_{\cancel{R_p}} = \mu_i\widehat{L}_i\widehat{H}_u +
  \lambda_{i j k}\widehat{L}_i\widehat{L}_j\widehat{l}_k +
  \lambda'_{i j k}\widehat{L}_i\widehat{Q}_j\widehat{d}_k +
  \lambda''_{ijk}\widehat{u}_i\widehat{d}_j\widehat{d}_k\,
\end{equation}
De las diferentes posibilidades se pueden construir modelos con
\begin{enumerate}
\item solamente términos bilineales que violan $R$--paridad, con acoplamientos $\mu_i$ que explican las masas y mezclas de los neutrinos \cite{Mira:2000gg,Dreiner:2003hw,Dreiner:2006xw}
\label{item:1}
\item modelos donde la simetría discreta de conservación de $R$--paridad, o una simetría equivalente, queda como remanente de la ruptura espontánea de la simetría $U(1)_H$ \cite{Dreiner:2003hw,Dreiner:2003yr,Dreiner:2007vp}. Las masas y mezclas de neutrinos se pueden explicar con la introducción de neutrinos derechos de carga $H$ semientera.
\label{item:2}
\item Modelos con violación de número leptónico a través de términos trilineales con ruptura de $R$--paridad. Se pueden construir modelos con hasta dos términos del tipo $\lambda_{ijk}$, con los tres índices diferentes \cite{Sierra:2009zq}.  En este caso las cargas $H$ se pueden escoger de manera que los acoplamientos trilineales que violan $R$--paridad queden muy suprimidos, del orden de $10^{-23}$, de modo que los neutralinos pueden ser  candidatos a materia oscura inestables con un tiempo de vida media del orden de $10^{26}\,$~sec. Dichos modelos se pueden usar para explicar las anomalías en rayos cósmicos \cite{Sierra:2009zq} recientemente detectadas por experimentos como PAMELA \cite{Adriani:2008zr}, ATIC \cite{:2008zzr} y Fermi~LAT \cite{Abdo:2009zk}. Estos datos sugieren que el decaimiento del neutralino sea principalmente leptónico, lo cual hace especialmente interesante que la única posibilidad consistente con simetrías horizontales de lugar precisamente a decaimientos leptónicos.
\label{item:3}
\item Modelos con violación de número bariónico a través de términos trilineales que violan $R$--paridad del tipo $\lambda''_{ijk}$. La fenomenología de este tipo de modelos aún no ha sido estudiada en la literatura
\label{item:4}
\end{enumerate}
En este proyecto pretendemos estudiar de manera sistemática la fenomenología de los modelos \ref{item:3}., \ref{item:4}. a luz de los resultados experimentales presentes y futuros.

En el caso \ref{item:3}., en lugar de asumir que los términos trilineales pueden explicar las anomalías de rayos cósmicos, se puede asumir que estas anomalías tienen un origen astrofísico, e intentar usar los datos presentes para restringir los posibles tiempo de vida media y masa del neutralino. En esta línea, también se planea estudiar las predicciones de modelos específicos de materia oscura inestable para los experimentos de rayos cósmicos en marcha, y los que comenzarán a funcionar en un futuro cercano.  En particular queremos  investigar el flujo de los rayos gamma que generan en modelos donde la materia oscura decae leptónica  o hadrónicamente (caso \ref{item:4}) y determinar las regiones del espacio de parámetros que puedan ser confrontadas con las medidas de rayos gamma que se esperan de Fermi--LAT en los próximos meses.

En el caso \ref{item:4}, en este proyecto pretendemos investigar posibles predicciones de señales para el LHC en modelos donde la simetría horizontal se use para formular modelos consistentes de violación de número bariónico a través de términos trilineales de ruputura $R$--paridad de tipo $\lambda''$. Después de tener en cuenta las restricciones experimentales sobre acoplamientos individuales
%, de las cuales la más importante es \cite{Barbier:2004ez}
%\begin{align}
%   \lambda_{11k}''<  10^{-9}   \left( \frac{m_{\tilde g}}{100\,\text{GeV}}  \right)^{1/2}
%\left( \frac{m_{\tilde q_k}}{100\,\text{GeV}}  \right)^2
%\end{align}
esperamos tener una predicción muy concreta para la jerarquía en los acoplamientos $\lambda''$, la cual podría dar lugar a señales muy específicas en el LHC. De este modo, en caso de que se descubriera en el LHC un modelo de ruptura de $R$--paridada través términos $\lambda''$, se podría determinar si la simetría Abeliana  $U(1)_H$ con un sólo flavón es la simetría que explica la jerarquía en la masas de los fermiones.

En el modelo usual para violación de número bariónico a través de términos $\lambda''$, donde se impone  una simetría discreta del tipo paridad leptónica, los términos no renormalizables asociados a la masa de neutrino quedan prohibidos, por lo que el mecanismo seesaw no puede ser implementado. En el caso de que la simetría horizontal se use para formular un modelo de este tipo, se puede introducir un nuevo flavón con carga $H$ fraccionaria [del tipo usado en \cite{Chen:2008tc}] que puede dar lugar a un mecanismo de seesaw apropiado para explicar los datos de oscilaciones de neutrinos. En este proyecto pretendemos explorar esta posibilidad.


\section{ Objetivos del Proyecto }

Proponer modelos que tengan consecuencias medibles en experimentos de aceleradores, de rayos cósmicos, y de detección directa de materia oscura, así como encontrar modelos que expliquen los resultados en cualquiera de ellos.

En modelos específicos para la generación radiativa de masas de neutrinos, establecer si la matriz de masa de neutrinos resultante puede llegar a ser reconstruida en los aceleradores futuros. Se considerarán modelos de supersimetría con Violación Bilineal de Paridad R (VBPR), el modelo de Zee para generación radiactiva de masas
de neutrinos a un loop, el modelo de Babu para generación radiativa de
masas de neutrinos a dos loops, y modelos 3-3-1 con neutrinos
derechos.

Analizar las restricciones astrofísicas y de aceleradores sobre los modelos supersimétricos
con violación de paridad R extendidos con una simetría $U(1)_H$ que presentan un neutralino inestable.

\section{ Metodología Propuesta }

Los elementos teóricos fundamentales del proyecto que queremos desarrollar
son los siguientes:

\begin{enumerate}
\item Se asume supersimetría,   y una simetría Abeliana de sabor $U(1)_H$ con un sólo flavón, $\Phi$, de carga $H$ $-1$.

\item Para un determinado término del superpotencial (\ref{eq:5}) con carga horizontal $n$ tenemos que, para $\theta=\langle\Phi\rangle$
  \begin{enumerate}
  \item Si $n\ge 0$ el acoplamiento se genera con una supresión $\theta^n$.
  \item Si $n<0$ el acoplamiento se genera con una supresión $(m_{3/2}/M)\theta^{|n|}\sim 10^{-17}\theta^{|n|}$ para un operador trilineal, y con una supresión $m_{3/2}\theta^{|n|}$ para un operador bilineal.
  \item Si $n$ es fraccionario el operador queda prohibido.
  \end{enumerate}
Usando estás posibilidades debemos fijar las cargas $H$ libres para obtener el modelo supersimétrico con las propiedades adecuadas. Las condiciones para obtener modelos con solo términos trilineales de ruptura de paridad $R$ con violación de número leptónico ya han sido establecidas en \cite{Sierra:2009zq}.

En este proyecto debemos establecer las condiciones para obtener modelos con únicamente acoplamientos de $\lambda''$. En el caso de $n<0$ nos interesa analizar las restricciones que los experimentos sobre rayos cósmicos actuales y futuros imponen sobre dicho modelo, mientras que en el caso en que los operadores con acoplamientos $\lambda''$ tengan cargas con $n\ge 0$, nos interesa explorar las predicciones que dicho modelo tenga para el LHC.
\end{enumerate}

Para el caso de modelos con un candidato a materia oscura inestable la metodología es la siguiente:
En este escenario, asumiremos que el flujo de positrones y rayos gamma astrofísicos constituirían un nuevo background para las posibles señales de detección de materia oscura \cite{Choi:2009qc}. En colaboración con el Dr. Carlos Yaguna, uno de los autores de la referencia anterior, se compararán los perfiles de dicho background  con las simulaciones hechas en \texttt{Pythia} \cite{Sjostrand:2006za} del flujo rayos cósmicos para modelos con ruptura trilineal de $R$--paridad donde los neutralinos son candidatos de materia oscura inestable. A partir de los flujos que superen de forma significativa el nuevo background se obtendrán las regiones de exclusión del  tiempo de vida media, y de masa para la partícula de materia oscura. Con el nuevo background  astrofísico, y con una metodología similar a la usada en \cite{Choi:2009qc} para  el caso de aniquilación de materia oscura, también se estudiará la posibilidad de detectar positrones que se originen a partir de materia oscura inestable, en futuros experimentos de medidas de flujos de positrones tales como el experimento AMS-02~\cite{ams:2009}. AMS-02 tomará datos durante tres años y espera medir el espectro de positrones hasta los $300\,$GeV. De esta comparación esperamos obtener el rango de tiempos de vida media y masas que se podrán explorar en AMS-02, para modelos con materia oscura inestable.


Para el caso de modelos con ruptura de paridad $R$ del tipo $\lambda''$ y carga $n\ge 0$ debemos encontrar una forma analítica que nos permita predecir el tamaño de los acoplamientos, y entonces establecer observables para el LHC que sean independientes del valor específico de $n$ y que nos permitan hacer predicciones que sean compatibles con las hipótesis de la simetría horizontal usada. Así mismo deberemos determinar la carga fraccionaria más óptima de un nuevo flavón que nos permita explicar de la mejor forma posible los resultados experimentales de oscilaciones de neutrinos.


\section{ Resultados Esperados }

Se espera que los resultados de está investigación clarifiquen hasta que punto se pueden acotar los valores de los parámetros trilineales que violan $R$--paridad en el caso de que las anomalías actuales en rayos cósmicos puedan ser explicadas por fuentes astrofísicas, como por ejemplo pulsares cercanos \cite[y referencias internas]{Chowdhury:2009jd}. Se espera mejorar las cotas existentes \cite{Barbieri:1988mw,Berezinsky:1991sp,Diwan:1996gw,Berezinsky:1996pb,Baltz:1997ar,Gupta:2004dz,Huber:2005iz} en varios ordenes de magnitud. También esperamos obtener señales específicas que se puedan contrastar con los resultados experimentales de rayos gamma que se esperan en los próximos meses.
%En particular del flujo de rayos gamma que se espera para un neutralino decayendo a $\tau^\pm\,\mu^\mp\,\nu_e$ a través de un acoplamiento trilineal del tipo $\lambda_{123}$, o de un neutralino decayendo hadrónicamente a través de acoplamientos $\lambda''$.
Así mismo se espera determinar el rango de tiempos de vida y media del neutralino como materia oscura inestable que se podrá excluir en experimentos futuros como el AMS-02 \cite{ams:2009}

De otro lado, esperamos construir el primer modelo estándar supersimétrico con una simetría horizontal anómala $U(1)_H$ que se pueda constrastar directamente con resultados experimentales de aceleradores. Así mismo como formular el primer modelo con violación de número bariónico a través de términos $\lambda''$ que a la vez de cuenta de las masas y mezclas de neutrinos.

En conclusión esperemos obtener predicciones específicas para modelos supersimétricos con contenido mínimo de partículas y una simetría horizontal anómala $U(1)_H$ que puedan comprobarse directamente en experimentos de rayos cósmicos o de aceleradores de partículas



\section{ Cronograma}



\begin{itemize}

\item {\bf meses 1-3 :} Se revisará la literatura existente sobre
  simetrías horizontales, además de revisar la literatura de  las diferentes colaboraciones de experimentos de rayos cósmicos.
  %  En esta fase también se formará el
  % estudiante de Maestría en los tópicos específicos
  % relativos a la investigación, así como permitirle tomar parte activa
  % en todas las fases de desarrollo del proyecto.

\item {\bf meses 4-11 : }   Se establecerá el background de rayos cósmicos, asumiendo que las anomalías actuales se pueden explicar a partir de fenómenos astrofísicos. Se hará una programa de simulación en \texttt{Pythia} \cite{Sjostrand:2006za} para los modelos con ruptura trilineal de $R$--paridad.
%a través de términos $\lambda$, así como para el modelo con ruptura de paridad $R$ a través de términos $\lambda''$.
Con este programa se calculará el flujo de positrones, electrones, y antiprotones, en cada modelo y se comparará con el background astrofísico para establecer las regiones de tiempo de vida media y masas de neutralinos que se pueden excluir a partir del los datos actuales. También se evaluará el impacto de las futuras medidas de positrones y electrones por parte PAMELA~\cite{Adriani:2008zr}, Fermi--LAT\cite{Abdo:2009zk} y AMS-02~\cite{ams:2009} en los modelos estudiados. Así mismo se evaluara el background de rayos gamma y se hará un programa de simulación en \texttt{Pythia} para el espectro de rayos gamas provenientes de desintegración de neutralinos en taus y hadrones.
%los cuales generan fotones en sus cadenas de decaimiento.
Estos resultados se compararán con las medidas preliminares\footnote{T. Porter (Fermi-LAT) (2009), Talk given at TeV Particle Astrophysics (TeVPA), July 13-17, 2009.} y con las medidas que serán reportadas por la colaboración Femi--LAT en los próximos meses.

Paralelamente se establecerán las condiciones sobre los cuatro cargas $H$ libres del modelo estándar supersimétrico con una simetría $U(1)_H$ con violación de número bariónico a través de términos trilineales $\lambda''$. Usando las restricciones experimentales sobre los acoplamientos individuales $\lambda''$~\cite{Barbier:2004ez}, se establecerán los cocientes entre acoplamientos $\lambda''$ que predice el modelo. A partir de dichos cocientes se establecerán las posibles señales de decaimiento de la partícula supersimétrica más liviana,  que permitan comprobar el modelo propuesto en el LHC. Adicionalmente se intentará implementar el mecanismo seesaw para explicar las masas y mezclas de neutrinos, introduciendo un nuevo flavón de carga fraccionaria.

\item {\bf meses 11-18 : } Se analizarán los resultados importantes y se escribirán dos artículos, uno con las restricciones de experimentos de rayos cósmicos en modelos supersimétricos de ruptura de $R$--paridad con materia oscura inestable, y otro sobre las predicciones para el LHC del modelo estándar supersimétrico con una simetría $U(1)_H$ con violación de número bariónico a través de términos trilineales $\lambda''$.

\end{itemize}



\section{ Compromisos y estrategia de comunicación }


Se asume los siguientes compromisos:

\begin{itemize}

\item Por lo menos dos artículos en revista internacional indexada
  A1.

\item Ponencia en evento internacional donde se presentaran los resultados de le
  investigación.

\item
Por lo menos un seminario en el Instituto de Física de la Universidad
de Antioquia, entregado por parte del estudiante de Doctorado.

\end{itemize}






\section{ Funciones de los Estudiantes Formación }



En el proyecto se incluye un estudiantes de Maestría y uno de doctorado.

Además de participar en todos los pasos del desarrollo de la
investigación, los estudiantes  tendrán las siguientes
tareas especificas.

Tareas específicas  estudiante de doctorado:
\begin{itemize}
\item Determinar los perfiles de background para electrones, positrones, antiprotones y rayos gamma
\item Elaborar los programas en \texttt{Pythia} para calculo del flujo de electrones, positrones, antiprotones y rayos gamma
\item Establecer las regiones de tiempo de vida media y masa del neutralino excluidas para los modelos bajo consideración
\item Elaborar un borrador con los principales resultados
\end{itemize}

Tareas específicas del estudiante de maestría:
\begin{itemize}
\item Establecer los diferentes conjuntos de parámetros $\lambda''$ para valores diferentes del la carga $H$ de los operadores correspondientes. En caso de que alguno de ellos resulte con carga $n$ negativa, calcular el acoplamiento diferente de cero que surge después de hacer la rotación del potencial de K\"ahler
\item Encontrar la carga fraccionaria óptima del nuevo flavón para explicar las masas y mezclas de los neutrinos a través del mecanismo seesaw en el modelo estándar supersimétrico son la simetría anómala $U(1)_H$ y violación de número bariónico a través de términos trilineales $\lambda''$
\item Elaborar un borrador con los principales resultados
\end{itemize}



\section{ Presupuesto}

Ver documento adjunto

%\newpage
\bibliographystyle{apsrev4-1long}
\bibliography{susy}%%,soko,snova,nu-rev06,parke-ref}

\end{document}
