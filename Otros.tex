\begin{instrucciones}
  Establecer la importancia y aporte de la investigación
propuesta en función de la generación de conocimiento, el desarrollo tecnológico, la innovación y
la solución a problemas nacionales.
\end{instrucciones}


\begin{instrucciones}
  Estos deben ser coherentes con los objetivos específicos y con la
  metodología planteada.\\
  Los resultados/productos pueden clasificarse en tres categorías:
  \begin{enumerate}
  \item \textbf{Relacionados con la generación de conocimiento y/o nuevos desarrollos
 tecnológicos:} Incluye resultados/productos que corresponden a nuevo
 conocimiento científico o tecnológico o a nuevos desarrollos o adaptaciones de
 tecnología que puedan verificarse a través de publicaciones científicas,
 productos o procesos tecnológicos, patentes, normas, mapas, bases de datos,
 colecciones de referencia, secuencias de macromoléculas en bases de datos de
 referencia, registros de nuevas variedades vegetales, etc.
\item \textbf{Conducentes al fortalecimiento de la capacidad científica
  nacional:} Incluye resultados/productos tales como formación de
  recurso humano a nivel profesional o de posgrado (trabajos de grado
  o tesis de maestría o doctorado sustentadas y aprobadas),
  realización de cursos relacionados con las temáticas de los
  proyectos (deberá anexarse documentación soporte que certifique su
  realización), formación y consolidación de redes de investigación
  (anexar documentación de soporte y verificación) y la construcción
  de cooperación científica internacional (anexar documentación de
  soporte y verificación).
\item \textbf{Dirigidos a la apropiación social del conocimiento:}
  Incluye aquellos resultados/productos que son estrategias o medios
  para divulgar o transferir el conocimiento o tecnologías generadas
  en el proyecto a los beneficiarios potenciales y a la sociedad en
  general. Incluye tanto las acciones conjuntas entre investigadores y
  beneficiarios como artículos o libros divulgativos, cartillas,
  videos, programas de radio, presentación de ponencias en eventos,
  entre otros.
  \end{enumerate}

  Para cada uno de los resultados/productos esperados identifique (en
  los cuadros a continuación) indicadores de verificación (ej:
  publicaciones, patentes, registros, videos, certificaciones, etc.)
  as\'\i{} como las instituciones, gremios y comunidades beneficiarias,
  nacionales o internacionales, que podrán utilizar los resultados de
  la investigación para el desarrollo de sus objetivos, políticas,
  planes o programas:

\end{instrucciones}

%Final
Las evidencias más sólidas de nueva física son las medidas de masas y mezclas de neutrinos y de materia oscura.

La enorme actividad experimental actual en física de partículas intenta determinar los modelos que incorporan esa nueva física. Esto requiere un esfuerzo teórico enorme al que esperamos aportar con este proyecto. En los últimos meses se ha reportando un exceso de rayos cósmicos de electrones y positrones.  Esto ha dado lugar a un sinnúmero de publicaciones tratando de explicar su origen [2]. Los diferentes experimentos de detección de materia oscura han comenzado a obtener resultados prometedores. El LHC ya ha comenzado a producir colisiones a una energía de 7 TeV.

El presente proyecto se centra en proponer modelos que tengan consecuencias medibles en todos estos experimentos, y encontrar modelos que expliquen los resultados de cualquiera de ellos. Pretendemos establecer observables que no sólo permitan descubrir [1] sino también  comprobar si el VBPR es el mecanismo de generación de masas de neutrinos, así como estudiar las implicaciones de modelos con un candidato de materia oscura. 

%%% Local Variables: 
%%% mode: latex
%%% TeX-master: Ficha-2011_bsm
%%% End: 
