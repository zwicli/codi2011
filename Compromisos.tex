\section{ Compromisos y estrategia de comunicación }
\begin{instrucciones}
  Formule los resultados directos verificables que se
alcanzarán con el desarrollo de los objetivos específicos del proyecto. Estos deben ser coherentes
con los objetivos y con la metodología planteada.


  \begin{enumerate}
  \item \textbf{Relacionados con la generación de conocimiento y/o nuevos desarrollos
 tecnológicos:} Incluye resultados/productos que corresponden a nuevo
 conocimiento científico o tecnológico o a nuevos desarrollos o adaptaciones de
 tecnología que puedan verificarse a través de publicaciones científicas,
 productos o procesos tecnológicos, patentes, normas, mapas, bases de datos,
 colecciones de referencia, secuencias de macromoléculas en bases de datos de
 referencia, registros de nuevas variedades vegetales, etc.
\item \textbf{Conducentes al fortalecimiento de la capacidad científica
  nacional:} Incluye resultados/productos tales como formación de
  recurso humano a nivel profesional o de posgrado (trabajos de grado
  o tesis de maestría o doctorado sustentadas y aprobadas),
  realización de cursos relacionados con las temáticas de los
  proyectos (deberá anexarse documentación soporte que certifique su
  realización), formación y consolidación de redes de investigación
  (anexar documentación de soporte y verificación) y la construcción
  de cooperación científica internacional (anexar documentación de
  soporte y verificación).
\item \textbf{Dirigidos a la apropiación social del conocimiento:}
  Incluye aquellos resultados/productos que son estrategias o medios
  para divulgar o transferir el conocimiento o tecnologías generadas
  en el proyecto a los beneficiarios potenciales y a la sociedad en
  general. Incluye tanto las acciones conjuntas entre investigadores y
  beneficiarios como artículos o libros divulgativos, cartillas,
  videos, programas de radio, presentación de ponencias en eventos,
  entre otros.
  \end{enumerate}

  Para cada uno de los resultados/productos esperados identifique (en
  los cuadros a continuación) indicadores de verificación (ej:
  publicaciones, patentes, registros, videos, certificaciones, etc.)
  as\'\i{} como las instituciones, gremios y comunidades beneficiarias,
  nacionales o internacionales, que podrán utilizar los resultados de
  la investigación para el desarrollo de sus objetivos, políticas,
  planes o programas:

\end{instrucciones}

%Final
%\begin{itemize}
 Artículo en revistas ISI (Web of Science) o Scopus. 
 Otro producto de nuevo conocimiento, de autoría o en coautoría
  del estudiante: artículo en revista ISI (Web of Science) o Scopus.
 Un artículo en revista de divulgación.
 Proyecto de investigación presentado a una entidad o
  convocatoria externa.
 Pasantía de investigación para el estudiante de Doctorado.
 Participaciòn con ponencia en evento académico nacional o
  internacional.
%\end{itemize}

\begin{ideas}
  


Dirección de un trabajo de investigación para optar al título de Magíster.

Opcional: Participación con ponencia en evento académico nacional :
    13° Encuentro Colombiano de Matemática Educativa 2012 - ASCOLMEPublicar al menos 4 artículos internacionales con algunos de los siguientes resultados:

Determinación de la factibilidad del LHC para determinar las correlaciones de VBPR entre el cociente de branchings de neutralino a W mu y W tau con el ángulo de mezcla atmosférico de neutrinos, así como de la correlación entre la longitud de decaimiento del neutralino y la diferencia de masa al cuadrado atmosférica.

Determinación del flujo de rayos gamas y su confrontación con lo observado en experimentos de rayos cósmicos en modelos supersimétricos donde el gravitino es el CMOI con masa de hasta 80 GeV. 

Determinación de acoplamientos, vida media y masa del gravitino como CMOI en modelos de ruptura trilineal de paridad R (RTPR) con violación de número leptónico, que pueda explicar el exceso de positrones en rayos cósmicos. Predicciones de flujo de positrones y rayos gama del modelo para experimentos futuros de rayos cósmicos. 

Determinación de las restricciones que los diferentes experimentos de rayos cósmicos 
presentes y futuros imponen sobre el neutralino como CMOI. 

Determinación de las señales en el LHC para 1) Modelos de RTPR con violación de número bariónico y mecanismo see-saw para masas de neutrinos, inducida por una simetría Abeliana anómala U(1). 2) Regiones del espacio de parámetros compatible con física de neutrinos y densidad de materia oscura del seesaw radiativo. 3) Modelos 3-3-1 con cuatro familias

Formación 1 estudiante de maestría y 1 de doctorado.
\end{ideas}

\begin{ideas}

Se asume los siguientes compromisos:

\begin{itemize}

\item Por lo menos tres artículos en revista internacional indexada
  A1. Y un artículo divulgativo.

\item Ponencia en evento internacional donde se presentaran los resultados de le
  investigación.

\item
Por lo menos un seminario en el Instituto de Física de la Universidad
de Antioquia, entregado por parte del estudiante de Doctorado.

\end{itemize}
\end{ideas}


%%% Local Variables: 
%%% mode: latex
%%% TeX-master: t
%%% End: 
