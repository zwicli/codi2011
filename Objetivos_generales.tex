\section{Objetivos generales}
\begin{instrucciones}


  \textbf{(máximo 500 palabras)}: Deben mostrar una relación clara y
  consistente con la descripción del problema y, específicamente, con
  las preguntas o hipótesis que se quieren resolver. La formulación de
  objetivos claros y viables constituye una base importante para
  juzgar el resto de la propuesta y, además, facilita la
  estructuración de la metodología. Se recomienda formular un solo
  objetivo general, coherente con el problema planteado, y los
  objetivos específicos necesarios para lograr el objetivo general.
  Estos últimos deben ser alcanzables con la metodología propuesta.
  Con el logro de los objetivos se espera, entre otras, encontrar
  respuestas a una o más de las siguientes preguntas: \textquestiondown Cuál será el
  conocimiento generado si el trabajo se realiza? \textquestiondown Qué solución
  tecnológica se espera desarrollar? Recuerde que la generación de
  conocimiento es más que la producción de datos nuevos y que no se
  deben confundir objetivos con actividades o procedimientos
  metodológicos.
\end{instrucciones}
\begin{evaluacion}
  A tener en cuenta: 20/100
  \begin{enumerate}
  \item Calidad y consistencia: 10\\
    Los objetivos son precisos y coherentes con el planteamiento del
    problema, con el título y específicamente con las preguntas y/o
    hipótesis que se quieren resolver?
  \item Viabilidad: 5\\
    Los objetivos son alcanzables con el enfoque teórico, con la
    metodología y el tiempo de ejecución propuestos?
  \item Impacto esperado y usuarios 5
  \end{enumerate}
\end{evaluacion}

\begin{evaluacion}
  Calidad y consistencia:
\end{evaluacion}

%%Final
Proponer modelos que tengan consecuencias medibles en experimentos de aceleradores, de rayos cósmicos, y de detección directa de materia oscura, así como encontrar modelos que expliquen los resultados que puedan ser obtenidos en cualquiera de ellos. En modelos específicos para la generación radiativa de masas de neutrinos, establecer si la matriz de masa de neutrinos resultante puede llegar a ser reconstruida en el LHC. Se considerarán especialmente modelos que a su vez contengan candidatos de materia oscura [1,2]. Para algunos modelos específicos que posean un candidato a materia oscura, estudiar de manera sistemática la fenomenología de estos modelos a luz de los resultados experimentales presentes y futuros. Para este tipo de modelos, establecer cuáles son las restricciones sobre el espacio de parámetros, impuestas por los aceleradores, y por los actuales y futuros experimentos de rayos cósmicos. Se considerarán modelos de violación de paridad R con el neutralino [3] o gravitino [4] como candidatos a materia oscura inestable (CMOI), además del seesaw radiativo [2]. Dichos modelos serán implementados en el paquete computacional Micromegas [5] para establecer las regiones del espacio de parámetros donde se obtiene la densidad de reliquia apropiada y las regiones donde se esperan señales observables en los experimentos de detección directa de materia oscura. Las referencias citadas se encuentran en el Anexo.

\newpage{}

%Oscar
Para algunos modelos específicos que posean un candidato a materia oscura, estudiar de manera sistemática la fenomenología de estos modelos a luz de los resultados experimentales presentes y futuros. Establecer cuáles son las restricciones sobre el espacio de parámetros, impuestas por los aceleradores y por los actuales y futuros experimentos de rayos cósmicos. Se considerarán los modelos del doblete escalar inerte,  seesaw radiativo, modelos supersimétricos con el gravitino como materia oscura inestable,  y modelos supersimétricos con violación de paridad R extendidos con una simetría horizontal U(1).



%%Jose David
Determinar el o los sectores del espacio de parámetros del modelo de seesaw radiativo que arrojen una correcta densidad 
de reliquia y estén de acuerdo a los datos observacionales de ángulos de mezcla y masas de los neutrinos.	

%%Proyecto viejo


\subsubsection{Objetivo general}
Proponer modelos que tengan consecuencias medibles en experimentos de aceleradores, de rayos cósmicos, y de detección directa de materia oscura, así como encontrar modelos que expliquen los resultados en cualquiera de ellos. 

En modelos específicos para la generación radiativa de masas de neutrinos, establecer si la matriz de masa de neutrinos resultante puede llegar a ser reconstruida en los aceleradores futuros. Se considerarán modelos de supersimetría con Violación Bilineal de Paridad R (VBPR), el modelo de Zee para generación radiativa de masas
de neutrinos a un loop, el modelo de Babu para generación radiativa de
masas de neutrinos a dos loops, y modelos 3-3-1 con neutrinos
derechos.
\vspace{4cm}
\\
%%%%%%%%%%%%        Inicio  OSCAR
Inicio Oscar.\\
Analizar las restricciones astrofísicas y de aceleradores sobre los modelos supersimétricos 
con violación de paridad R extendidos con una simetría $U(1)_H$ que presentan un neutralino inestable.

Fin Oscar
%%%%%%%%%%%%        Fin  OSCAR
%%% Local Variables: 
%%% mode: latex
%%% TeX-master: "main"
%%% End: 



