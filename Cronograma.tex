\section{ Cronograma}
\begin{instrucciones}
  ¿La secuencia de actividades se adecúa a las fases de desarrollo del proyecto? ¿La duración en cada una de las etapas es apropiada y garantiza el cumplimiento del objetivo?
\end{instrucciones}


\begin{itemize}
\item \textbf{meses 1-3} Revisión bibliográfica sobre cálculos de
  secciones eficaces promediadas térmicamente , programas
  computacionales para cálculos de secciones eficaces y amplitudes de
  decaimientos, propiedades de los detectores y filtros de selección
  de datos en el LHC, propagación de los diferentes tipos de rayos
  cósmicos a través de la galaxia.  En esta fase también se formará el
  estudiante de Doctorado en los tópicos específicos relativos a la
  investigación, así como permitirle tomar parte activa en todas las
  fases de desarrollo del proyecto.

\item \textbf{meses 3-12}
  Durante está fase desarrollaremos las herramientas computacionales necesarias para el proyecto, y los cálculos analíticos necesarios:

\begin{proyecto}
  Con los colaboradores de Brasil y Valencia hemos desarrollado un
  programa computacional en PYTHIA para calcular el nivel de precisión
  con el que se puede llegar a medir en el LHC la correlación entre el
  ángulo de mezcla atmosférico de neutrinos y el cociente decaimientos
  a dos cuerpos: $\tilde\chi_1^0\to W^\pm\mu^\mp$ sobre
  $\tilde\chi_1^0\to W^\pm\tau^\mp$, cuando el $W^\pm$ decae
  hadrónicamente. Por lo que es necesario reconstruir la masa
  invariante del $W^\pm$ a partir de los jets a los que se desintegra
  y de usar cortas para seleccionar el $\mu$ o el $\tau$ asociado a
  cada decaimiento del neutralino. En ésta fase se modificará el
  programa para establecer el nivel de precisión con la que se puede
  medir la correlación entre la longitud de decaimiento del neutralino
  con la diferencia de masa atmosférica de masa al cuadrado.  El
  programa de calcular longitud de decaimiento $L_0$ del neutralino y
  su error $\Delta L_0$, donde $L_0$ está dada por
  \begin{align}
    L_0=\frac{L^{\text{lab}}}{p_{\tilde\chi_1^0}^{\text{reco}}}m_{\tilde\chi_1^0}^{inv}
  \end{align}
  donde $L^{\text{lab}}$ es la distancia desde el punto de interacción
  hasta el punto en el espacio donde decae el neutralino, $p$ y $m$
  son el cuadri-momentum y la masa invariante del neutralino. Para la
  determinación de $\Delta L_0$ debemos hacer un programa para hacer
  ajustas a las distribuciones gaussianas del cocientes
  $\frac{L^{\text{lab}}}{p_{\tilde\chi_1^0}^{\text{reco}}}$ y de la
  masa invariante $m_{\tilde\chi_1^0}^{inv}$. Similarmente, para calcular el nivel de precisión
  con el que se puede llegar a medir la correlación entre el ángulo de mezcla solar con 
  los decaimientos a tres cuerpos mediados por sfermiones con muones y
  electrones en los estados finales, debemos modificar el programa
  para reconstruir estos decaimientos a tres cuerpos. 
\end{proyecto}

\begin{proyecto}
  Para el caso en que el gravitino es la LSP debemos modificar el programa para calcular las correlaciones con base en la NLSP teniendo en cuenta los nuevos canales en las cuales la NLSP puede decaer al gravitino 
\end{proyecto}


\begin{proyecto}
  En ésta parte del proceso se establecerán las condiciones sobre los
  cuatro cargas $H$ libres del modelo estándar supersimétrico con una
  simetría $U(1)_H$ con violación de número leptónico a través de
  términos trilineales $\lambda$. Usando las restricciones
  experimentales sobre los acoplamientos individuales
  $\lambda$~\cite{Barbier:2004ez}, se establecerán las posibles señales de decaimiento
  de la partícula supersimétrica más liviana, que permitan comprobar
  el modelo propuesto en el LHC. Adicionalmente se 
  implementará el mecanismo seesaw para explicar las masas y mezclas de
  neutrinos, introduciendo nuevos flavones de carga fraccionaria.
  En ésta parte se
  analizará de forma sistemática las asignaciones de carga que
  permitan la generación de texturas adecuadas de masas de neutrinos y
  a la vez mantengan prohibidos los otros términos de ruptura de
  paridad R. De acuerdo a exploraciones iniciales esperamos que en
  este caso la LSP deba ser un gravitino si se quiere mantener un
  candidato a materia oscura inestable. Para explorar las
  restricciones que los datos de detección indirecta, se establecerá
  el background de rayos cósmicos, asumiendo que las anomalías
  actuales se pueden explicar a partir de fenómenos astrofísicos. Se
  hará una programa de simulación en \texttt{Pythia}
  \cite{Sjostrand:2006za} para calcular el flujo de positrones,
  electrones, y se comparará con el background astrofísico para
  establecer las regiones de tiempo de vida media y masas de
  neutralinos que se pueden excluir a partir del los datos
  actuales. También se evaluará el impacto de las futuras medidas de
  positrones y electrones por parte PAMELA~\cite{Adriani:2008zr},
  Fermi--LAT\cite{Abdo:2009zk} y AMS-02~\cite{ams:2009} en los modelos
  estudiados. Es de resaltar que la simetría horizontal sólo permite
  la aparición de términos $\lambda$ con los tres índices diferentes
  por los decaimientos del gravitino no pueden generar rayos gamma.
\end{proyecto}


\begin{proyecto}
  Para el seesaw radiativo, escribiremos el Lagrangiano del modelo el
  lenguaje de programación LanHEP~\cite{0805.0555}. Con este programa
  podremos generar las cuatro tablas en el formato de
  CalcHEP~\cite{hep-ph/0412191}, con lo cual podremos usar calcular
  las secciones eficaces y las amplitudes de decaimiento del modelo, y
  como MicrOMEGAS usas internamente CalcHEP, calcular también la
  densidad de reliquia y la sección eficaz WIMP-nucleón. Debemos crear
  también una interfase en Python para hacer los llamados a los
  diferentes cálculos y para implementar las regiones del espacio de
  parámetros que son compatibles con los datos de oscilaciones de
  neutrinos. Paralelamente debemos hacer cálculos analíticos de las
  secciones eficaces promediadas térmicamente para estar seguros que
  los resultados de MicroMEGAS son correctos.
\end{proyecto}


\item \textbf{meses 12-18}
En esta parte se obtendrán y analizarán los resultados numéricos de los modelos bajo consideración. El estudiante de doctorado realizará su pasantía doctoral en un Grupo de Investigación europeo.

\item \textbf{meses 18-24} 
  En esta parte se prepararán los artículos
  para publicación con base en los resultados obtenidos. Uno sobre ...
  Se analizarán los resultados importantes y se escribirán dos
  artículos, uno con las restricciones de experimentos de rayos
  cósmicos en modelos supersimétricos de ruptura de $R$--paridad con
  materia oscura inestable, y otro sobre las predicciones para el LHC
  del modelo estándar supersimétrico con una simetría $U(1)_H$ con
  violación de número bariónico a través de términos trilineales
  $\lambda''$.

  Se escribirá el artículo de divulgación sobre resultados recientes
  del LHC, que para la época estará brindando resultados definitivos
  de la búsqueda del Higgs hasta una masa de 600 GeV.

  Se presentarán los resultados en una conferencia internacional y se
  preparará el informe final del proyecto.

\end{itemize}

%%% Local Variables: 
%%% mode: latex
%%% TeX-master: "Ficha-2011_bsm"
%%% End: 

