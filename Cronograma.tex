\section{ Cronograma}



\begin{itemize}

\item {\bf meses 1-3 :} Se revisará la literatura existente sobre
  simetrías horizontales, además de revisar la literatura de  las diferentes colaboraciones de experimentos de rayos cósmicos.
  %  En esta fase también se formará el
  % estudiante de Maestría en los tópicos específicos
  % relativos a la investigación, así como permitirle tomar parte activa
  % en todas las fases de desarrollo del proyecto.

\item {\bf meses 4-11 : }   Se establecerá el background de rayos cósmicos, asumiendo que las anomalías actuales se pueden explicar a partir de fenómenos astrofísicos. Se hará una programa de simulación en \texttt{Pythia} \cite{Sjostrand:2006za} para los modelos con ruptura trilineal de $R$--paridad.
%a través de términos $\lambda$, así como para el modelo con ruptura de paridad $R$ a través de términos $\lambda''$.
Con este programa se calculará el flujo de positrones, electrones, y antiprotones, en cada modelo y se comparará con el background astrofísico para establecer las regiones de tiempo de vida media y masas de neutralinos que se pueden excluir a partir del los datos actuales. También se evaluará el impacto de las futuras medidas de positrones y electrones por parte PAMELA~\cite{Adriani:2008zr}, Fermi--LAT\cite{Abdo:2009zk} y AMS-02~\cite{ams:2009} en los modelos estudiados. Así mismo se evaluara el background de rayos gamma y se hará un programa de simulación en \texttt{Pythia} para el espectro de rayos gamas provenientes de desintegración de neutralinos en taus y hadrones.
%los cuales generan fotones en sus cadenas de decaimiento.
Estos resultados se compararán con las medidas preliminares\footnote{T. Porter (Fermi-LAT) (2009), Talk given at TeV Particle Astrophysics (TeVPA), July 13-17, 2009.} y con las medidas que serán reportadas por la colaboración Femi--LAT en los próximos meses.

Paralelamente se establecerán las condiciones sobre los cuatro cargas $H$ libres del modelo estándar supersimétrico con una simetría $U(1)_H$ con violación de número bariónico a través de términos trilineales $\lambda''$. Usando las restricciones experimentales sobre los acoplamientos individuales $\lambda''$~\cite{Barbier:2004ez}, se establecerán los cocientes entre acoplamientos $\lambda''$ que predice el modelo. A partir de dichos cocientes se establecerán las posibles señales de decaimiento de la partícula supersimétrica más liviana,  que permitan comprobar el modelo propuesto en el LHC. Adicionalmente se intentará implementar el mecanismo seesaw para explicar las masas y mezclas de neutrinos, introduciendo un nuevo flavón de carga fraccionaria.

\item {\bf meses 11-18 : } Se analizarán los resultados importantes y se escribirán dos artículos, uno con las restricciones de experimentos de rayos cósmicos en modelos supersimétricos de ruptura de $R$--paridad con materia oscura inestable, y otro sobre las predicciones para el LHC del modelo estándar supersimétrico con una simetría $U(1)_H$ con violación de número bariónico a través de términos trilineales $\lambda''$.

\end{itemize}

%%% Local Variables: 
%%% mode: latex
%%% TeX-master: "Ficha-2011_bsm"
%%% End: 

