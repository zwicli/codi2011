%Solicitud contrapartida de proyecto de COLCIENCIAS
\documentclass[11pt]{article}
\usepackage[spanish]{babel}
%Maldito arial 12
\usepackage[utf8]{inputenc}
%changing the deafult fonts
\usepackage{textcomp}
\renewcommand{\rmdefault}{phv}
\renewcommand{\sfdefault}{pag}
\renewcommand{\ttdefault}{pcr}

\usepackage{amsmath}
\usepackage{amssymb}
%\usepackage{bm}
\usepackage{graphicx}
\usepackage{colortbl}
\usepackage{array}
\usepackage{xcolor}

\usepackage{multirow}
\usepackage{tabularx}


\usepackage[nomessages]{fp}
\usepackage{cancel}
\usepackage[colorlinks=true]{hyperref}
%\usepackage[square,numbers]{natbib}
%\setcitestyle{notesep={:}}
\usepackage[punctsep]{collref}

\setlength{\oddsidemargin}{0pt}
\setlength{\evensidemargin}{0pt}
\setlength{\topmargin}{-1cm}
\setlength{\textheight}{21.5cm}
\setlength{\textwidth}{17cm}

\newcolumntype{Y}{>{\centering\arraybackslash}X}

%%new comments enviorement
\usepackage{comment}
\includecomment{instrucciones}
\specialcomment{instrucciones}
{\begingroup\itshape }{ \medskip \endgroup}
\excludecomment{instrucciones}

\includecomment{ideas}
\specialcomment{ideas}
{\begingroup}{ \medskip \endgroup}
\excludecomment{ideas}

\includecomment{evaluacion}
\specialcomment{evaluacion}
{\begingroup}{ \medskip \endgroup}
\excludecomment{evaluacion}

\includecomment{gammalines}
\specialcomment{gammalines}
{\begingroup}{ \medskip \endgroup}
\excludecomment{gammalines}


\includecomment{proyecto}
\specialcomment{proyecto}
{\begingroup}{\endgroup}



\begin{document}

\vglue -2cm
\renewcommand{\arraystretch}{1.28}
\noindent
\hspace{-0.5cm}\begin{tabularx}{1.059\linewidth}{|c|Y|c|c|c|}\hline
 \multirow{4}{*}{\parbox[b]{3cm}{\vspace{0.1cm}\centering\includegraphics[scale=0.3]{udea}\\
\scriptsize{{\fontfamily{ptm}\selectfont Universidad de Antioquia}}\\
\tiny{{\fontfamily{ptm}\selectfont Vicerrectoría de Investigación}}
}
}  & \textbf{\large VICERRECTORÍA DE INVESTIGACIÓN} & \multicolumn{3}{c|}{{\large Actualización}}\\
\cline{2-5}
 &\textbf{FICHA TÉCNICA}  & {\large Día} & {\large Mes} & {\large Año}\\
\cline{3-5}
 &  &  &  & \\[-13pt]
 & \textbf{\small PROYECTOS DE INVESTIGACIÓN} &  &  & \\[-5pt]
 & \textbf{2009} &  &  & \\
\hline
\end{tabularx}
%
\renewcommand{\arraystretch}{1}

\bigskip{}
\noindent{}
La ficha técnica es el resumen del proyecto, con una extensión máxima de seis páginas sin contar anexos. El investigador debe tener especial cuidado para que la ficha técnica contenga la información básica necesaria acerca del proyecto, puesto que el jurado final solo recibe la mencionada ficha. El proyecto de investigación quedará en archivo en el centro de investigaciones.



\begin{center}
  \textbf{\large INFORMACIÓN GENERAL}
\end{center}

\renewcommand{\arraystretch}{1.71}
\noindent
\hspace{-0.5cm}\begin{tabularx}{1.059\linewidth}{|l|l|}\hline
\multicolumn{2}{|Y|}{\cellcolor[gray]{.8}\textbf{Implicaciones de una simetría $U(1)_H$ en experimentos de rayos cósmicos y de aceleradores}}\\\hline
\multicolumn{2}{|Y|}{}\\\hline
\cellcolor[gray]{.8}\textbf{Facultad, Escuela o Instituto}&Facultad de Ciencias Exactas y Naturales\\\hline
\cellcolor[gray]{.8}\textbf{Grupo que avala el proyecto}& Fenomenología de Interacciones Fundamentales (GFIF)\\\hline
\cellcolor[gray]{.8}\textbf{Código Colciencias del Grupo si aplica}&COL0008423\\\hline
\cellcolor[gray]{.8}\textbf{Nombre del Investigador Principal}&Diego A. Restrepo Qunitero\\\hline
\cellcolor[gray]{.8}\textbf{Nombre del Coinvestigador}&William Ponce\\\hline
\cellcolor[gray]{.8}\textbf{Documento de Identidad N${}^{\text{\textbf{o}}}$}&98554575\\\hline
\cellcolor[gray]{.8}\textbf{Convocatoria}&Sostenibilidad 2009-2010 de GFIF\\\hline
\cellcolor[gray]{.8}\textbf{Duración del proyecto (en meses)}&18 meses\\\hline
\end{tabularx}

\vspace{-0.02cm}
\renewcommand{\arraystretch}{1.038}
\noindent
\hspace{-0.5cm}\begin{tabularx}{1.059\linewidth}{|l|r|}
\multicolumn{2}{|Y|}{\cellcolor[gray]{.8}\textbf{DATOS DEL PRESUPUESTO}}\\[7pt]\hline
\textbf{Recursos propios del Grupo}\hspace{6.2cm}\quad&{\$ 14,000,000,\footnotesize{00}}\\\hline
\textbf{Universidad} \footnotesize{(recursos en especie y recursos  frescos)}&{\$ 43,750,000\footnotesize{00}}\\\hline
\textbf{Otras entidades} \footnotesize{(si aplica):}&{\$ 0,\footnotesize{00}}\\\hline
\textbf{Valor total del proyecto:}&{\$ 57,750,000,\footnotesize{00}}\\\hline
\end{tabularx}
\renewcommand{\arraystretch}{1}

\begin{center}
  \textbf{\large DESCRIPCIÓN DEL PROYECTO}
\end{center}

\begin{instrucciones}
  Del investigador para ser asociado:
  \begin{itemize}
  \item  Haber estado vinculado al grupo de investigación durante el último año. El Coordinador del grupo debe certificar las actividades investigativas en las que se ha participado.

  \item La vinculación del investigador asociado al grupo no puede exceder los tres años.

  \item Ser un investigador joven que represente un claro prospecto para la generación de relevo profesoral. Edad máxima: 35 años.

  \item No poseer la calidad de profesor o estudiante (de la Universidad de Antioquia). Se exceptúan de esta última condición los profesores de cátedra.
  \end{itemize}
\end{instrucciones}


\section{Línea, programa o agenda de investigación en la que se inscribe el proyecto}

\begin{instrucciones}
planteamiento del problema, marco teórico, objetivos, 
metodología, resultados esperados, cronograma, compromisos y estrategia de comunicación, funciones del 
estudiante, presupuesto (según formato, incorporado en el texto y la versión en Excel) y  bibliografía.  Es 
importante tener en cuenta que el proyecto de investigación es el documento sobre el cual se realizará la 
evaluación, constituye un medio de comunicación científica por el cual el investigador interactúa con pares 
anónimos que darán un concepto sobre la calidad de su proyecto. El proyecto no puede tener una 
extensión mayor de  30 páginas, incluidos anexos y bibliografía
\end{instrucciones}

% replace \include by \input at the end to check the pages
% In branch release the full document will be generated
\section{Planteamiento del Problema }
\begin{instrucciones}
  CODI: ¿Está bien definido el problema que se quiere investigar? ¿Es clara su justificación desde el punto de vista académico, científico, tecnológico, social, económico y legal? (15).

  COLCI: en este ítem usted deberá describir de forma precisa y completa la
  naturaleza y magnitud del problema de investigación que se quiere
  abordar. Formule claramente las preguntas concretas a las cuales se
  quiere responder en el contexto del problema planteado.
\end{instrucciones}
%Tesis
El modelo estándar de las interacciones fundamentales ha sido muy exitoso durante las últimas décadas para explicar la mayoría de los resultados experimentales de física de altas energías. Sin embargo, simultáneamente se ha venido acumulando evidencia experimental que requiere extender el modelo estándar con nuevas partículas.  En éste proyecto se explorarán diferentes extensiones del modelo estándar que explican estos datos y se obtendrán predicciones concretas en experimentos presentes y futuros.

%evidencia 1
La principal evidencia de necesidad de física más allá del modelo estándar se halla en las medidas de las oscilaciones de neuiontrinos, que a través de muchos experimentos diferentes han logrado establecer que los neutrinos tienen masa y se mezclan entre sí. Si la masa de neutrinos es total o parcialmente explicada por métodos radiativos, no sólo es posible dar cuenta de la pequeñez de las masas de los neutrinos con respecto a la de los otros fermiones, sino también, que modelos de este tipo pueden hacer predicciones muy concretas en aceleradores de partículas, como el LHC. Aún en el caso de modelos tan estudiados de este tipo, como es el caso del modelo supersimétrico con violación bilineal de paridad R, quedan muchos aspectos que explorar. 

%evidencia 2
En los últimos tiempos, sobre todos después de las medidas de precisión de WMAP, el satelite de la NASA que ha medido con mucha precisión la radiación cósmica de fondo, se ha venido estableciendo la necesidad de una forma de materia compuesta por partículas  débilmente interactuantes (WIMPs de sus siglas en inglés) que se conoce como materia oscura. Esta forma el 22\% de la materia del Universo. Sin embargo, aún no se ha encontrado evidencia directa de la existencia de dicha partícula, y existen modelos incluso donde dicho tipo de materia podría ser indetectable.

%evidencia 3
La materia oscura puede ser estable o con un tiempo de vida mucho mayor que la edad del Universo. En el caso de materia oscura inestable se esperan señales fuertes en los detectores de rayos cósmicos instalados en satélites artificiales que orbitan la tierra como Fermi-LAT y PAMELA, y en los detectores futuros como AMS-02~\cite{ams:2009}, el cual será instalado en la estación espacial internacional.  De hecho, recientemente se ha encontrado un exceso en positrones que se podría explicar con una partícula de materia oscura que decae principalmente a leptones con una vida media de unos $10^{26}$ seg.

%evidencia 4
Las observaciones astronómicas sugieren que el Universo está compuesto en su mayor parte de materia. En el contexto del big-bang, esto implica que en algún momento grandes cantidades de materia y antimateria se aniquilaron dejando el pequeño exceso de materia que constituye el Universo observable actual. El problema de explicar el exceso inicial de materia sobre antimateria se conoce con el nombre de bariogenesis. Con el modelo estándar no es posible explicar bariogénesis.  

%conclusion
Un modelo ideal sería uno que de cuenta de las masas y mezclas de neutrinos, tenga un candidato a materia oscura que sirva para explicar el exceso de positrones en experimentos de rayos cósmicos y a la vez contenga los ingredientes para explicar bariogenesis. En este proyecto pretendemos formular un modelo de éstas características, además queremos continuar explorando otros posibles modelos que puedan dar cuenta de alguna de las observaciones que requieren un extensión del modelo estándar.


 

\subsection{Marco teórico}

\begin{instrucciones}
 CODI: ¿Es la teoría actualizada y acertada con respecto al problema que se va a estudiar? ¿Su formulación es coherente? ¿Es clara la perspectiva teórica desde donde se ubica el problema?
\end{instrucciones}

En el grupo nos hemos enfocado en extensiones del modelo estándar que dan cuenta de las masas y mezclas de neutrinos. Hemos estudiado exhaustivamente las predicciones del modelo bilineal con ruptura de paridad R en el Tevatron y el LHC \cite{hep-ph/0304232,hep-ph/0501153,DeCampos:2010yu,deCampos:2008ic,deCampos:2008re,deCampos:2007bn}. La metodología desarrollada la hemos aplicado a otros modelos de generación radiativa de masas de neutrinos con nuevas partículas a la escala del TeV asequibles en el LHC \cite{Sierra:2008wj,AristizabalSierra:2006ri}. Recientemente hemos comenzado a explorar modelos que puedan dar cuenta simultáneamente de masas de neutrinos y candidatos de materia oscura \cite{Hirsch:2005ag,Choi:2010jt,Sierra:2008wj}. Nuestro grupo \cite{Nardi:2008ix} fue uno de los primeros en proponer una explicación en términos de materia oscura inestable para explicar el exceso de positrones observado por el satelite PAMELA en el 2008\cite{Adriani:2008zr}. Luego hemos construido un modelo basado en supersimetría con ruptura de paridad R para explicar la preferencia por decaimientos leptónicos de la partícula de materia oscura, que en este caso es el neutralino. Para ello se ha implementado una simetría horizontal que permite sólo la presencia de acoplamientos trilineales leptónicos \cite{Sierra:2009zq}. En un proyecto en marcha estamos explorando la posibilidad opuesta de un candidato de materia oscura que decae sólo hadrónicamente. La simetría horizontal garantiza que sólo los acoplamientos trilineales de quarks derechos estén permitidos. Finalmente hemos extendido el modelo para incluir masas de neutrinos.


%Actualmente se encuentra funcionado el acelerador Tevatron que acelera protones y antiprotones a una energía de 1 TeV, y se espera que siga funcionado por 2 años más. 
A finales de este año se espera que entre en funcionamiento el acelerador LHC. Este acelerará protones en ambos sentidos a una energía que alcanzará hasta los 14 TeV. 
%La mayoría de los experimentos conducidos son hechos en grandes instrumentos llamados detectores. Por ejemplo el LHC tiene 4 detectores. Dos ellos ATLAS y CMS, son detectores de propósito general donde se espera explorar señales de nueva física.
Desde el año pasado varios experimentos sobre rayos cósmicos como ATIC \cite{:2008zzr} y los satélites PAMELA \cite{Adriani:2008zr} y Fermi--LAT \cite{Abdo:2009zk}, han venido reportando un exceso en el flujo de electrones y positrones en rayos cósmicos. Estos resultados han dado lugar a un sinnúmero de publicaciones tratando de explicar su origen. En los próximos meses se esperan medidas de rayos gamma por parte de Fermi--LAT, las cuales pueden ayudar a discernir si el origen de las anomalías detectadas en electrones y positrones es debida a fuentes astrofísicas como pulsares cercanos, o a la aniquilación o el decaimiento de materia oscura. El detector de rayos cósmicos AMS-02~\cite{ams:2009},  que será instalado en la Estación Espacial  internacional, medirá el espectro de antiprotones y positrones en un rango de energía mucho más amplio y con una estadística mucho mejor.
El satélite Planck fue lanzado en mayo, y desde julio  se encuentra en pleno funcionamiento.  Con los datos obtenidos de este experimento se espera medir la radiación cósmica de fondo con mucha más precisión que su antecesor WMAP. También hay varios experimentos de física de neutrinos en marcha que esperan mejorar los datos actuales de oscilaciones de neutrinos.


\subsection{Justificación}
En este proyecto queremos explorar modelos motivados por falencias 






 



Con la entrada en operación del LHC, se espera descubrir señales de materia oscura y de modelos que tengan algún mecanismo radiativo para generar masas de neutrinos.

Por otro lado, los recientes datos reportados por las colaboraciones PAMELA y Fermi han impuesto restricciones muy fuertes sobre el flujo de rayos cósmicos provenientes de partículas de materia oscura. Los futuros experimentos de rayos cósmicos, tales como el experimento AMS-02, Planck, IceCube, etc, esperan mejorar la sensibilidad a este tipo de señales. Así mismo, los experimentos de detección directa de materia oscura como DAMA, CDMS, CoGeNT, XENON han creado grandes expectativas por los resultados presentes y los que se espera obtener en los próximos meses. Los experimentos de física de neutrinos y de radiación cósmica de fondo han entrado ya en una era de medidas de precisión.

Con base en lo planteado anteriormente, lo que proponemos en este proyecto es tratar de responder las siguiente pregunta: ¿Cuáles serían las restricciones impuestas por los recientes y futuros datos de todos estos experimentos sobre  modelos que presenten una partícula candidata a materia oscura o que generen masas para los neutrinos?


\subsection{Marco teórico}


\begin{ideas}
Se ha calculado el flujo de rayos gamas y confrontarlo con lo observado en experimentos de rayos cósmicos en modelos supersimétricos donde el gravitino es el candidato de materia oscura inestable (CMOI) en la región de masa de hasta 80 GeV. Estudios recientes  muestran que en está región son importantes canales de decaimiento que no se habían considerado previamente [4]. 



  El problema que queremos abordar trata acerca de la fenomenología más alla del modelo estándar.
%Con la entrada en operación del LHC, se espera descubrir señales de materia oscura y de modelos que tengan algún mecanismo radiactivo para generar masas de neutrinos.
%Por otro lado, los recientes datos reportados por las colaboraciones PAMELA y Fermi han impuesto restricciones muy fuertes sobre el flujo de rayos cósmicos provenientes de partículas de materia oscura. Los futuros experimentos de rayos cósmicos, tales como el experimento AMS-02, Planck, IceCube, etc, esperan mejorar la sensibilidad a este tipo de señales. Así mismo, los experimentos de detección directa de materia oscura como DAMA, CDMS, CoGeNT, XENON han creado grandes expectativas por los resultados presentes y los que se esperan obtener en los próximos meses. Los experimentos de física de neutrinos y de radiación cósmica de fondo han entrado ya en una era de medidas de precisión.
Ya que con la entrada en operación del LHC más el tevatrón, se esperan descubrir señales del bosón de Higgs, de la materia oscura, dimensiones extra, supersimetría, etc. En algunos modelos más allá del Modelo Estándar se tienen predicciones concretas sobre los posibles productos de aniquilación y desintegración del candidato a materia oscura que podrían ser detectados en estos aceleradores.La mayoría de los experimentos conducidos son hechos en grandes instrumentos llamados detectores. Por ejemplo el LHC tiene 4 detectores. En dos de ellos ATLAS y CMS hay grupos colombianos participando. En los últimos años el Grupo ha tenido una estrecha vinculación con el Grupo de Marta Losada que esta participando en la colaboración ATLAS.

 Además se tienen otros experimentos como son las colaboraciones PAMELA y Fermi que han impuesto restricciones muy fuertes sobre el flujo de rayos cósmicos provenientes de partículas de materia oscura. Los futuros experimentos de rayos cósmicos, tales como el experimento AMS-02, Planck, IceCube, etc, esperan mejorar la sensibilidad a este tipo de señales. Así mismo, los experimentos de detección directa de materia oscura como DAMA, CDMS, CoGeNT, XENON han creado grandes expectativas por los resultados presentes y los que se esperan obtener en los próximos meses.

%Actualmente se encuentra funcionando el acelerador Tevatron en Fermilab en los Estados Unidos. Este acelera protones y antiprotones a una energía de 1 TeV, y se espera que siga funcionado por 2 años más. Ya ha entrado en funcionamiento el acelerador LHC en el CERN en la frontera entre Francia y Suiza. Aunque de momento se encuentra en estado de pruebas funcionando a energías que alcanzarán los 7TeV, en un futuro cercano éste acelerará protones en ambos sentidos a una energía que alcanzará hasta los 14 TeV. La mayoría de los experimentos conducidos son hechos en grandes instrumentos llamados detectores. Por ejemplo el LHC tiene 4 detectores. En dos de ellos ATLAS y CMS hay grupos colombianos participando. En los últimos años el Grupo ha tenido una estrecha vinculación con el Grupo de Marta Losada que esta participando en la colaboración ATLAS.

%A nivel de detección de radiación cósmica tenemos el satélite PAMELA para detección de rayos cósmicos de electrones, positrones, antiprotones, y núcleos livianos. En agosto del año pasado PAMELA reporto una abundancia anómala de positrones %\cite{pamela}, 
%que ha dado lugar a un sinnúmero de publicaciones tratando de explicar su origen.  Fermi-LAT es otro satélite para la detección de rayos cósmicos que también estará funcionando durante los próximos años. En sus primeras medidas %\cite{fermi} 
%también ha reportado una abundancia anómala de positrones y electrones. El detector de rayos cósmicos AMS-02~%\cite{ams:2009},  
%que será instalado en la Estación Espacial internacional, medirá el espectro de antiprotones y positrones en un rango de energía mucho más amplio y con una estadística mucho mejor. Análogamente está el satélite Planck que fue lanzado en mayo, y desde julio se encuentra en pleno funcionamiento. Con los datos obtenidos de este experimento se espera medir la radiación cósmica de fondo con mucha más precisión que su antecesor WMAP. Además esta el experimento Auger en Argentina que detecta rayos cósmicos de ultra alta energía que alcanzan a llegar a la superficie terrestre.

Igualmente hay varios experimentos de física de neutrinos en marcha que esperan mejorar los datos actuales de oscilaciones de neutrinos, determinar el ángulo de mezcla $\theta_{13}$, del cual actualmente solo hay cotas, y si este resulta ser diferente de cero, buscar evidencia de violación de CP en el sector de neutrinos, un ingrediente crucial para generar bariogénesis a través de leptogénesis.

 Aunque se ha encontrado que los neutrinos tienen masas a la escala de
sub-eV, no se ha podido establecer cual es el modelo que contiene
el mecanismo apropiado de generación de masas y mezcla de neutrinos.
En particular, será muy difícil establecer la naturaleza de los
neutrinos, es decir si tienen masa de Dirac o Majorana. El mecanismo seesaw %\cite{Gell-Mann:vs} 
ha sido la
explicación más popular para la pequeñez de las masas de neutrinos. Es
elegante y simple, y depende sólo de análisis dimensional para la
nueva física que se requiere. Los datos actuales de neutrinos
apuntan a una escala de seesaw del orden $10^{10}$ - $10^{15}$ GeV,
donde la violación de número leptónico ocurre a través de masas de
Majorana de los neutrinos derechos. Con una escala tan alta, los
efectos de violación de sabor leptónico en procesos diferentes a la
oscilación de neutrinos son extremadamente pequeños. Una alternativa al mecanismo seesaw que también explica naturalmente
la pequeñez de la masa de neutrinos son los mecanismos de generación
radiativa.
%\cite{hep-ph/0503059,AristizabalSierra:2006ri,Babu:2002uu,Dong:2006vk}.
%En este aproximación las masas de neutrinos son cero a nivel árbol y
%son inducidas únicamente como correcciones radiativas finitas.
% Estas correcciones radiativas son proporcionales a la raíz cuadrada de
%las masas del leptón cargado (o quark) dividas por la escala de nueva
%física $\Lambda$. Las masas de los neutrinos pueden estar en el rango
%de sub-eV aún cuando la escala de nueva física sea del orden de
%TeV. En este caso, los procesos de violación de número leptónico en
%otros procesos diferentes a la física de neutrinos pueden llegar
%a ser accesibles a los aceleradores.
En este proyecto proponemos analizar en detalle las consecuencias
experimentales de modelos específicos donde las masas de
neutrinos se generan radiativamente.
% En particular estamos interesados
%en calcular los decaimientos de las partículas del modelo que
%estén controlados por los mismos parámetros que inducen las masas y
%las mezclas de los neutrinos. Estos decaimientos podrían no sólo
%ayudar a determinar el mecanismo de generación de masas de neutrinos
%con aceleradores de partículas, sino también ayudar a determinar
%experimentalmente los parámetros del modelo que aparecen en la matriz
%de masa de neutrinos.

Con lo expuesto anteriormente, las preguntas que queremos abordar en el presente proyecto son: 
¿se podrá establecer en los aceleradores futuros cual es el modelo que
explica la generación de masas y mezclas para los neutrinos? ¿Cuáles serían las restricciones 
impuestas por los recientes y futuros datos astrofísicos y de aceleradores sobre los modelos supersimétricos con violación de paridad R extendidos con una simetría $U(1)_H$ que presentan un neutralino ó gravitino inestable? Y ¿Cuáles serían las restricciones impuestas por los recientes y futuros datos de todos estos experimentos sobre  modelos que presenten una partícula candidata a materia oscura?
Todo esto representa un escenario muy dinámico donde el Grupo debe estar preparado para interpretar cualquier evidencia que resulte del Higgs o nueva física a la luz de los modelos que viene desarrollando.




\textbf{Comentar sobre la linea de rayos gamma \cite{1011.3786}}
\subsection{Planteamiento del problema}
\begin{instrucciones}
  ¿Están vinculados con el problema planteado? ¿Son viables, claros, concretos y factibles, de acuerdo con el estudio y los métodos?
\end{instrucciones}

En el caso de modelos supersimétricos basados en una simetría de sabor anómala $U(1)_H$, con un solo flavón de carga $-1$, una vez se introducen las condiciones teóricas y fenomenológicas, la solución más óptima se puede expresar en términos de 4 cargas $H$ libres que se puede usar para explicar el valor de los 45 parámetros del superpotencial que violan $R$--paridad \cite{Mira:2000gg,Dreiner:2003hw,Dreiner:2003yr,Dreiner:2007vp,Dreiner:2006xw,Sierra:2009zq}. 
\begin{equation}
  \label{eq:5}
  W_{\cancel{R_p}} = \mu_i\widehat{L}_i\widehat{H}_u +
  \lambda_{i j k}\widehat{L}_i\widehat{L}_j\widehat{l}_k +
  \lambda'_{i j k}\widehat{L}_i\widehat{Q}_j\widehat{d}_k +
  \lambda''_{ijk}\widehat{u}_i\widehat{d}_j\widehat{d}_k\,
\end{equation}
De las diferentes posibilidades se pueden construir modelos con
\begin{enumerate}
\item Solamente términos bilineales que violan $R$--paridad, con acoplamientos $\mu_i$ que explican las masas y mezclas de los neutrinos \cite{Mira:2000gg,Dreiner:2003hw,Dreiner:2006xw}
\label{item:1}
\item Modelos donde la simetría discreta de conservación de $R$--paridad, o una simetría equivalente, queda como remanente de la ruptura espontánea de la simetría $U(1)_H$ \cite{Dreiner:2003hw,Dreiner:2003yr,Dreiner:2007vp}. Las masas y mezclas de neutrinos se pueden explicar con la introducción de neutrinos derechos de carga $H$ fraccionaria.
\label{item:2}
\item Modelos con violación de número leptónico a través de términos trilineales con ruptura de $R$--paridad. Se pueden construir modelos con hasta dos términos del tipo $\lambda_{ijk}$, con los tres índices diferentes \cite{Sierra:2009zq}.  En este caso las cargas $H$ se pueden escoger de manera que los acoplamientos trilineales que violan $R$--paridad queden muy suprimidos, del orden de $10^{-23}$, de modo que los neutralinos o gravitinos pueden ser candidatos a materia oscura inestables con un tiempo de vida media del orden de $10^{26}\,$~sec. Dichos modelos se pueden usar para explicar las anomalías en rayos cósmicos \cite{Sierra:2009zq} recientemente detectadas por experimentos como PAMELA \cite{Adriani:2008zr}, ATIC \cite{:2008zzr} y Fermi~LAT \cite{Abdo:2009zk}. Estos datos sugieren que el decaimiento del neutralino sea principalmente leptónico, lo cual hace especialmente interesante que la única posibilidad consistente con simetrías horizontales de lugar precisamente a decaimientos leptónicos.
\label{item:3}
\item Modelos con violación de número bariónico a través de términos trilineales que violan $R$--paridad del tipo $\lambda''_{ijk}$. La fenomenología de este tipo de modelos aún no ha sido estudiada en la literatura
\label{item:4}
\end{enumerate}
%En este proyecto pretendemos estudiar de manera sistemática la fenomenología de los modelos \ref{item:3}., \ref{item:4}. a luz de los resultados experimentales presentes y futuros.

%En el caso \ref{item:3}., en lugar de asumir que los términos trilineales pueden explicar las anomalías de rayos cósmicos, se puede asumir que estas anomalías tienen un origen astrofísico, e intentar usar los datos presentes para restringir los posibles tiempo de vida media y masa del neutralino o gravitino. En esta línea, también se planea estudiar las predicciones de modelos específicos de materia oscura inestable para los experimentos de rayos cósmicos en marcha, y los que comenzarán a funcionar en un futuro cercano. En particular queremos  investigar el flujo de los rayos gamma que generan en modelos donde la materia oscura decae leptónica o hadrónicamente (caso \ref{item:3}) y determinar las regiones del espacio de parámetros que puedan ser confrontadas con las medidas de rayos gamma que se esperan de Fermi--LAT en los próximos meses.

%En el caso \ref{item:3}, en este proyecto pretendemos investigar posibles predicciones de señales para el LHC en modelos donde la simetría horizontal se use para formular modelos consistentes de violación de número leptónico a través de términos trilineales de ruputura $R$--paridad de tipo $\lambda$. Después de tener en cuenta las restricciones experimentales sobre acoplamientos individuales
%, de las cuales la más importante es \cite{Barbier:2004ez}
%\begin{align}
%   \lambda_{11k}''<  10^{-9}   \left( \frac{m_{\tilde g}}{100\,\text{GeV}}  \right)^{1/2}
%\left( \frac{m_{\tilde q_k}}{100\,\text{GeV}}  \right)^2
%\end{align}
%esperamos tener una predicción muy concreta para la jerarquía en los acoplamientos $\lambda$, la cual podría dar lugar a señales muy específicas en el LHC. De este modo, en caso de que se descubriera en el LHC un modelo de ruptura de $R$--paridad a través términos $\lambda$, se podría determinar si la simetría Abeliana  $U(1)_H$ con uno ó más flavones es la simetría que explica la jerarquía en la masas de los fermiones.
En el modelo usual para violación de número leptónico a través de términos $\lambda$, donde se impone una simetría discreta del tipo paridad leptónica, los términos no renormalizables asociados a la masa del neutrino quedan prohibidos, por lo que el mecanismo seesaw no puede ser implementado. En el caso de que la simetría horizontal se use para formular un modelo de este tipo, se puede introducir un nuevo flavón con carga $H$ fraccionaria [del tipo usado en \cite{Chen:2008tc}] que puede dar lugar a un mecanismo de seesaw apropiado para explicar los datos de oscilaciones de neutrinos. En este proyecto pretendemos explorar la posibilidad de obtener una ttextura para las masa de los neutrinos en base a un modelo supersimétrico a la Froggatt-Nielsen, y además obtener un candidato consistente para materia oscura.

\end{ideas}


%%% Local Variables: 
%%% mode: latex
%%% TeX-master: "Ficha-2011_bsm"
%%% End: 

\section{Marco teórico}

\begin{instrucciones}
 CODI: ¿Es la teoría actualizada y acertada con respecto al problema que se va a estudiar? ¿Su formulación es coherente? ¿Es clara la perspectiva teórica desde donde se ubica el problema?
\end{instrucciones}

%comentar sobre una simetría Z_2
La partícula de materia oscura puede ser estable, o inestable con un
tiempo de vida mucho más grande que la edad del Universo. En el primer
caso usualmente se impone una simetría adicional $Z_2$ bajo la cual
las partículas del modelo estándar son pares, mientras que las
partículas nuevas son impares. Este tipo de simetría garantiza que la
partícula impar más liviana es estable y, de ser neutra, constituye un
buen candidato de materia oscura. En el programa computacional
microMEGAS~\cite{Belanger:2006is} se puede implementar cualquier
extensión del modelo estándar que posea una simetría $Z_2$ para
calcular numéricamente la densidad de reliquia de materia oscura y la
sección eficaz WIMP--nucleón, relevante para los experimentos de de
detección directa como XENON100. Cuando la partícula de materia oscura
es inestable se requiere de alguna simetría para garantizar que su
acoplamiento a las partículas del modelo estándar esté suficientemente
suprimido.

En el caso de supersimetría por ejemplo, todas las interacciones del
modelo están especificadas por la simetría gauge y el superpotencial,
que debe ser una función holomórfica de los campos escalares del
modelo. En el modelo estándar supersimétrico, además de los términos
del superpotencial que constituyen la semilla del potencial escalar y
del Lagrangiano de Yukawa, la invarianza gauge permite los siguientes
términos
\begin{equation}
  \label{eq:5}
  W_{\cancel{R_p}} = \mu_i\widehat{L}_i\widehat{H}_u + 
  \lambda_{i j k}\widehat{L}_i\widehat{L}_j\widehat{l}_k +
  \lambda'_{i j k}\widehat{L}_i\widehat{Q}_j\widehat{d}_k + 
  \lambda''_{ijk}\widehat{u}_i\widehat{d}_j\widehat{d}_k\,
\end{equation}
La simetría $Z_2$, conocida en éste caso como paridad R, $R_p$,
prohíbe todos estos términos, y cuando la partícula impar bajo paridad
R es el neutralino o el gravitino, ésta puede ser una candidato de
materia oscura. Aunque se rompa paridad R y alguno de estos términos
queden permitidos, el gravitino puede seguir siendo un candidato de
materia oscura pues sus decaimientos están suprimidos por la escala de
Planck. Para que el neutralino pueda ser materia oscura inestable
cuando se rompe paridad R, se requiere que los acoplamientos sean
extremadamente pequeños: del orden de $10^{-24}$ para los
acoplamientos trilineales ${\lambda^{(\ }}'\;''{}^)$. 

En modelos supersimétricos extendidos para incluir una simetría
horizontal anómala $U(1)_H$ a la Froggatt-Nielsen (FN)
\cite{Froggatt:1978nt}, las partículas del modelo estándar y sus
supercompañeros no llevan un número cuántico de paridad R, y en su
lugar llevan una carga horizontal (carga~$H$). Para una artículo de
revisión ver \cite{Dreiner:2003hw}.  Además, este tipo de modelos
involucran nuevos campos pesados de FN y, en la realización más
simple, un supercampo singlete electrodébil $\Phi$ de carga $H$
$-1$. Términos efectivos invariantes bajo $SU(3)_c\times SU(2)_L\times
U(1)_Y\times U(1)_H$ que conservan y también que violan paridad R,
surgen una vez los grados de libertad pesados son integrados debajo la
escala de los campos FN ($M$), donde $M$ corresponde a la masa de
Planck. Estos términos involucran factores del tipo $(\Phi/M)^n$,
donde $n$ esta fijado por las cargas horizontales de los campos
involucrados, y determina si un término en particular puede o no estar
presente en el superpotencial. La holomorfía del superpotencial
prohíbe todos los términos para los cuales $n<0$, y aunque ellos se
generaran después la ruptura de supersimetría a través el potencial de
K\"ahler \cite{Giudice:1988yz} estos términos son en general mucho más
suprimidos que aquellos para los cuales $n\ge0$.  Términos con $n$
fraccionario están completamente prohibidos. Finalmente, una vez
$U(1)_H$ se rompe los términos con $n$ positivo mantienen
acoplamientos de Yukawa determinados, hasta factores de orden uno, por
$\theta^n=(\langle\phi\rangle/M)^n$. Los acoplamientos de Yukawa del
modelo estándar provienen típicamente de términos de esta clase.

En este tipo de modelos, los ángulos de mezcla de los quarks, las masa
de los leptones cargados, y las condiciones de cancelación de
anomalías restringen los posibles asignamientos de cargas $H$
\cite{Leurer:1992wg,Binetruy:1996xk}. Ya que el número de
restricciones es siempre menor que el número de cargas $H$ algunas de
ellas quedan necesariamente sin restricciones, y aparte del límite
superior en sus valores pueden considerarse como parámetros libres que
se puede determinar a partir de información fenomenológica
adicional. 


En el caso de modelos supersimétricos basados una simetría de sabor
anómala $U(1)_H$, con un solo flavón de carga $-1$, una vez se
introducen las condiciones teóricas y fenomenológicas, la solución más
óptima se puede expresar en términos de 4 cargas $H$ libres que se
puede usar para explicar el valor de los 45 parámetros del
superpotencial que violan paridad R en la ec.~(\ref{eq:5})
\cite{Mira:2000gg,Dreiner:2003hw,Dreiner:2003yr,Dreiner:2007vp,Dreiner:2006xw,Sierra:2009zq}.

De las diferentes posibilidades se pueden construir modelos: con 
\begin{enumerate}
\item el modelo con ruptura bilineal de paridad R, es decir, con
  acoplamientos $\mu_i$ que explican las masas y mezclas de los
  neutrinos \cite{Mira:2000gg,Dreiner:2003hw,Dreiner:2006xw}.
\label{item:1}
\item modelos donde la simetría discreta de conservación de
  paridad R, o una simetría equivalente, queda como remanente de la
  ruptura espontánea de la simetría $U(1)_H$
  \cite{Dreiner:2003hw,Dreiner:2003yr,Dreiner:2007vp}. Las masas y
  mezclas de neutrinos se pueden explicar con la introducción de
  neutrinos derechos de carga $H$ semientera.
\label{item:2}
\item Modelos con violación de número leptónico a través de términos
  trilineales con ruptura de paridad R. Se pueden construir modelos
  con hasta dos términos del tipo $\lambda_{ijk}$, con los tres
  índices diferentes \cite{Sierra:2009zq}.  En este caso las cargas
  $H$ se pueden escoger de manera que los acoplamientos trilineales
  que violan paridad R queden muy suprimidos, del orden de $10^{-23}$,
  de modo que si los neutralinos son la partícula supersimétrica más
  livina (LSP de sus siglas en inglés), éstos pueden ser candidatos a
  materia oscura inestable con un tiempo de vida media del orden de
  $10^{26}\,$~sec. Dichos modelos se pueden usar para explicar las
  anomalías en rayos cósmicos \cite{Sierra:2009zq} recientemente
  detectadas por experimentos como PAMELA \cite{Adriani:2008zr}, ATIC
  \cite{:2008zzr} y Fermi~LAT \cite{Abdo:2009zk}. Estos datos sugieren
  que el decaimiento del neutralino sea principalmente leptónico, lo
  cual hace especialmente interesante que la única posibilidad
  consistente con simetrías horizontales de lugar precisamente a
  decaimientos leptónicos.
\label{item:3}
\item Modelos con violación de número bariónico a través de términos
  trilineales que violan $R$--paridad del tipo $\lambda''_{ijk}$. La
  fenomenología de este tipo de modelos aún no ha sido estudiada en la
  literatura
\label{item:4}
\end{enumerate}
En este proyecto pretendemos estudiar de manera sistemática la
fenomenología de los modelos \ref{item:1}. y \ref{item:3}. El caso
\ref{item:3}. sin embargo, nos enfocaremos en la posibilidad en la que
el LSP es el gravitino, donde creemos que también podemos explicar las
masas de los neutrinos, y además como en este caso los acoplamientos
trilineales pueden ser mucho más grandes, tenemos la posibilidad de
comprobar el modelo en el LHC.

\subsection{Justificación del problema}

En el Grupo nos hemos enfocado en extensiones del modelo estándar que
dan cuenta de las masas y mezclas de neutrinos. Hemos estudiado
exhaustivamente las predicciones del modelo con ruptura bilineal de
paridad R que incluye sólo los tres términos con $\mu_i$ (caso
\ref{item:1}.), tanto para el Tevatron como para el LHC, asumiendo que
el neutralino es la LSP
\cite{Magro:2003zb,deCampos:2005ri,deCampos:2007bn,deCampos:2008ic,deCampos:2008re,DeCampos:2010yu}. En
el último trabajo al respecto hemos determinado el nivel de precisión
con el que se puede llegar a medir en el LHC la correlación entre
decaimientos de neutralinos a muón y tau con el ángulo de mezcla
atmosférico de neutrinos: una predicción muy concreta que de no
observarse en el LHC en los próximos años descartaría completamente el
modelo como el mecanismo de generación de masas para neutrinos.
\begin{proyecto}
  En este proyecto pretendemos seguir explorando más correlaciones de
  de observables en el LHC con física de neutrinos para determinar con
  que nivel de precisión se podrían llegar a medir en el LHC. Cuando
  el neutralino es la LSP, los decaimientos a tres cuerpos mediados
  por sfermiones con muones y electrones en los estados finales, están
  correlaciones con el ángulo de mezcla solar, y la longitud de
  decaimiento del neutralino está correlacionada con la diferencia de
  masa atmosférica.
\end{proyecto}

Cuando el gravitino es la LSP, se constituye en un candidato viable de
materia oscura inestable. Éste caso ha sido estudiado exhaustivamente
en la literatura por sus implicaciones en experimentos de rayos
cósmicos, especialmente en el caso en que las masas de neutrinos son
explicadas a través del mecanismo de seesaw. En tal caso el modelo
también explica bariogénisis a través de leptogénesis. De hecho, como
hemos mostramos en \cite{Choi:2010jt}, y ratificado por otro grupo con
datos más recientes en \cite{Garny:2010eg}, los últimos datos de
líneas de rayos gamma publicado por FERMI-LAT excluyen la posibilidad
de que las masas de neutrinos puedan ser generadas por términos
bilineales de ruptura de paridad R si la temperatura de
recalentamiento está sobre $10^9\ $GeV como sugiere leptogénesis. En
esta región del espacio de parámetros el gravitino es mayor de unos 10
GeV y los acoplamientos bilineales son tan pequeños que el decaimiento
de la partícula siguiente a la LSP, la NLSP (de sus siglas en inglés),
ocurre fuera del detector en el LHC, por lo que el modelo es
básicamente indistinguible del MSSM. En éste caso los datos de
detección indirecta de materia oscura a través de rayos cósmicos
serían la única forma de diferenciar el modelo del MSSM (donde el
neutralino es el candidato de materia oscura). Sin embargo, si
consideremos masas de gravitinos más pequeños, aunque ya no podríamos
explicar bariognesis a través de leptogénesis, recuperamos la
posibilidad explicar las masas y mezclas de los neutrinos a través de
los términos bilineales que rompen paridad R~\cite{Hirsch:2005ag}, un
mecanismo, que a diferencia del seesaw, si se puede verificar en el
LHC.

\begin{proyecto}
  En este proyecto realizaremos las simulaciones de las señales del
  NSLP para el LHC  en las regiones del
  espacio de parámetros donde el gravitino es un buen candidato a
  materia oscura y los términos bilineales explican los datos de
  oscilaciones de neutrinos. 
\end{proyecto}

\begin{darkmatter}
  La metodología desarrollada en el estudio exhaustivo del modelo con
  ruptura bilineal de paridad R la hemos logrado aplicar a otros
  modelos de generación radiativa de masas de neutrinos con nuevas
  partículas a la escala del TeV asequibles en el LHC
  \cite{Sierra:2008wj,AristizabalSierra:2006ri}. Recientemente hemos
  comenzado a explorar modelos que puedan dar cuenta simultáneamente
  de masas de neutrinos y candidatos de materia oscura
  \cite{Hirsch:2005ag,Choi:2010jt,Sierra:2008wj}. En el caso del
  seesaw radiativo por ejemplo, se implementa una simetría $Z_2$ de
  manera que la masa de neutrinos sólo pueda ser generada
  radiativamente con los neutrinos derechos a la escala del TeV
  garantizando que el modelo pueda comprobarse en el LHC. La misma
  simetría garantiza que el escalar o a un neutrino derecho puedan ser
  buenos candidatos de materia oscura. Finalmente, la estructura
  escalar del modelo permite tener el Higgs mucho más pesado que en el
  modelo estándar de forma compatible con las correcciones
  electrodébiles.
% Las regiones del modelo en la cual la partícula de materia oscura es
% escalar ha sido bastante estudiada en la literatura en el contexto
% del modelo conocido como doblete inerte \cite{1011.1411}. Sin
% embargo, las regiones en las cuales el neutrino derecho es el
% candidato de materia oscura todavía requieren de


\begin{proyecto}
  Todo esto hace muy llamativo el estudio de las señales del seesaw
  radiativo para el LHC, y en los experimentos de detección directa de
  materia oscura. En este proyecto pretendemos establecer todas las
  regiones del espacio de parámetros donde se puede tener la densidad
  de reliquia de materia oscura y las masas y mezclas de neutrinos
  adecuadas. En cada región se establecerán las señales que se esperan
  en el LHC y en experimentos de detección directa de materia oscura,
  y para las señales más representativas hacer simulaciones para el
  detector ATLAS del LHC.
\end{proyecto}
\end{darkmatter}

La otra posibilidad de tener términos de violación de número leptónico
en el superpotencial compatible con la simetría horizontal, es el caso
\ref{item:3}. donde tenemos un modelo con ruptura trilineal de paridad
R a través de términos $\lambda$. Esta posibilidad es muy llamativa
porque da lugar a un candidato de materia oscura que decae sólo
leptónicamente. Nuestro grupo \cite{Nardi:2008ix} fue uno de los
primeros en proponer una explicación en términos de materia oscura
inestable para explicar el exceso de positrones observado por el
satelite PAMELA en el 2008 \cite{Adriani:2008zr}. Luego hemos
construido un modelo basado en supersimetría con ruptura de paridad R
a través de términos trilineales del tipo $\lambda$, caso
\ref{item:3}, para explicar la preferencia por decaimientos leptónicos
de la partícula de materia oscura, que en este caso es el
neutralino \cite{Sierra:2009zq}.
\begin{proyecto}
  En este proyecto queremos explorar la posibilidad de explicar
  también las masas de neutrinos, y el gravitino como materia oscura
  en el marco del modelo con ruptura de paridad R a través de términos
  trilineales del tipo $\lambda$.
\end{proyecto}
En un proyecto en marcha estamos explorando la posibilidad opuesta
\ref{item:4}., de un candidato de materia oscura que decae sólo
hadrónicamente. La simetría horizontal garantiza que sólo los
acoplamientos trilineales de quarks derechos estén permitidos. En
dicho proyecto hemos extendido el modelo para incluir masas de
neutrinos. Ésta experiencia nos servirá para hacer las modificaciones
necesarias al modelo con acoplamientos trilineales leptónicos del tipo
$\lambda$ para dar cuenta de masas y mezclas para los neutrinos.


\begin{proyecto}
  En términos generales, en este proyecto queremos explorar modelos
  motivados por falencias fenomenológicas del modelo estándar,
  especialmente aquellos modelos que no sólo tienen implicaciones en
  el LHC, sino también en experimentos de detección de rayos cósmicos,
  y experimentos de detección directa de materia oscura.
\end{proyecto}



\begin{proyecto}
  Con base en lo planteado anteriormente, lo que proponemos en este
  proyecto es tratar de responder la siguiente pregunta: ¿Cuáles
  serían las restricciones impuestas por los recientes y futuros datos
  de los experimentos de física de partículas sobre modelos que
  presenten una partícula candidata a materia oscura o que generen
  masas para los neutrinos?
\end{proyecto}

%%% Local Variables: 
%%% mode: latex
%%% TeX-master: "Ficha-2011_bsm"
%%% End: 

\section{ Objetivos del Proyecto }

Proponer modelos que tengan consecuencias medibles en experimentos de aceleradores, de rayos cósmicos, y de detección directa de materia oscura, así como encontrar modelos que expliquen los resultados en cualquiera de ellos.

En modelos específicos para la generación radiativa de masas de neutrinos, establecer si la matriz de masa de neutrinos resultante puede llegar a ser reconstruida en los aceleradores futuros. Se considerarán modelos de supersimetría con Violación Bilineal de Paridad R (VBPR), el modelo de Zee para generación radiactiva de masas
de neutrinos a un loop, el modelo de Babu para generación radiativa de
masas de neutrinos a dos loops, y modelos 3-3-1 con neutrinos
derechos.

Analizar las restricciones astrofísicas y de aceleradores sobre los modelos supersimétricos
con violación de paridad R extendidos con una simetría $U(1)_H$ que presentan un neutralino inestable.


%%% Local Variables: 
%%% mode: latex
%%% TeX-master: Ficha-2011_bsm
%%% End: 


\section{Metodología Propuesta}
\begin{instrucciones}
  Esta debe reflejar la estructura lógica y el rigor científico del
  proceso de investigación, empezando por la elección de un enfoque
  metodológico específico y finalizando con la forma como se van a
  analizar, interpretar y presentar los resultados.
\end{instrucciones}
%%%

%Final
A lo largo de nuestra experiencia con el modelo con violación de
paridad R, hemos logrado implementar todos los pasos de la metodología
usada en física de partículas para estudiar modelos de física más allá
del modelo estándar.  Ésta parte desde la formulación misma del modelo
hasta su estudio real en los aceleradores de partículas por parte de
los grupos experimentales. Esta metodología también la hemos logrado
aplicar con éxito a otros modelos. A continuación se muestran los
pasos a seguir usando como paradigma el modelo con ruptura bilineal de
paridad R y ejemplificando cada paso con referencias concretas.
\begin{enumerate}
\item Se construye un modelo que solucione un problema fenomenológico
  del modelo estándar, como el problema de las masas de neutrinos
  \cite{Hirsch:2000ef}, o la materia oscura (o ambos
  \cite{Hirsch:2005ag}), mostrando bajo que mecanismo específico se
  soluciona el problema en cuestión. La solución del problema requiere
  introducir partículas adicionales las cuales pueden buscarse en
  detectores de partículas.
  \label{item:5}
\item Se determina el espacio de parámetros compatible con los datos
  experimentales, se calculan los branchings de decaimiento y las
  secciones eficaces y se establecen correlaciones entre estos
  observables en aceleradores con otros observables. En el caso de
  modelos con mecanismos de generación radiativa de masas de neutrinos
  las correlaciones se buscan con los datos de oscilaciones de
  neutrinos \cite{Diaz:2003as}, y en el caso de materia oscura con
  experimentos de detección directa e indirecta \cite{Choi:2010xn}.
  \label{item:6}
\item Se desarrolla un programa computacional que dado los parámetros
  de entrada del modelo entregue las secciones eficaces y los
  branching de decaimiento en un formato adecuado (el SLHA
  \cite{Allanach:2008qq}) para ser usado luego por programas de Montecarlo
  de generación de eventos como PYTHIA \cite{Sjostrand:2006za}. El
  modelo con ruptura bilineal de paridad R ha sido implementado en el
  programa computacional Spheno~\cite{Porod:2003um} (el cual tiene
  implementado el formato de salida SLHA version 2).
  \label{item:7}
\item Se divide el espacio de n-parámetros establecido en el paso
  \ref{item:6} en una malla n-dimensional. Para cada punto se corre el
  programa computacional desarrollado en el paso \ref{item:7} y los
  correspondiente datos de salida se pasan a PYTHIA usando la
  interfase SLHA. En cada punto se realiza con PYTHIA una simulación
  que consiste en generar aleatoriamente eventos de acuerdo a la
  geometría y características de detectores específicos, para
  determinar la factibilidad de descubrir las señales en aceleradores
  \cite{Magro:2003zb,deCampos:2005ri,deCampos:2007bn,deCampos:2008ic,deCampos:2008re},
  o en experimentos de detección directa o indirecta
  \cite{Choi:2010xn} de materia oscura. Este paso requiere de
  herramientas de computación de alto rendimiento en clusters de
  computadores.
  \label{item:8}
\item Como modelos muy diferentes pueden dar lugar a las mismas
  señales en detectores, se debe hacer también simulaciones en PYTHIA
  del nivel de precisión con el que se pueden determinar observables
  en aceleradores que se puedan correlacionar con observables de otros
  experimentos.
 % \begin{soloproyecto}
 %    A partir del valor experimental del ángulo de mezcla
 %  atmosférico, en el modelo con ruptura bilineal de paridad R por
 %  ejemplo, se predice que el decaimiento del neutralino a muon W, debe
 %  ser muy similar al decaimiento del neutralino a tau W. En
 %  \cite{DeCampos:2010yu} hemos determinado el nivel de precisión con
 %  el que se pude medir esta correlación en el detector ATLAS del LHC.
 %   \end{soloproyecto}
  \label{item:9}
\end{enumerate}
Las simulaciones se realizan con el fin de desarrollar todas las
herramientas necesarias para que los grupos experimentales de los
aceleradores puedan comparar los datos obtenidos con modelos
específicos y puedan descubrir o poner cotas sobre las nuevas
partículas propuestas. En el caso de detección directa o indirecta de
materia oscura donde las señales están preestablecidas, la
simulaciones se realizan para poder comparar con los datos obtenidos o
que se pueden llegar a obtener. En el caso del modelo con ruptura
bilineal de paridad R, ya se han hecho búsquedas en
Tevatron~\cite{Brigliadori:2008vf} y se están implementando búsquedas
con datos reales en el LHC en grupos de trabajo de ATLAS y
LHCb. Además se han establecido la posibles señales que se pueden
observar en experimentos de detección indirecta de materia oscura a
través de rayos cósmicos \cite{Choi:2010jt}.

\begin{proyecto}
  En el marco de este proyecto se simularán dos correlaciones
  adicionales que existen en el modelo con ruptura bilineal de paridad R:
  Cuando el neutralino es la LSP, los decaimientos a tres cuerpos
  mediados por sfermiones con muones y electrones en los estados
  finales, están correlaciones con el ángulo de mezcla solar, y la
  longitud de decaimiento del neutralino está correlacionada con la
  diferencia de masa atmosférica.
\end{proyecto}

\begin{proyecto}
  En el modelo donde la ruptura bilineal de paridad R puede explicar
  tanto las masas de neutrinos como la densidad de reliquia de materia
  oscura cuando la LSP es el gravitino se realizaran los pasos
  \ref{item:7} y \ref{item:8} de la metodología.
\end{proyecto}

\begin{darkmatter}
  \begin{proyecto}
  Para el seesaw radiativo, en la literatura básicamente sólo se ha
  realizado el paso \ref{item:5}, aunque para algunas regiones del
  espacio de parámetros donde la partícula más liviana de paridad
  impar (LOP de sus siglas en inglés) es escalar se ha llegado hasta
  el paso \ref{item:8}~\cite{Bergman:2007pm}. En este proyecto haremos un
  estudio sistemático del modelo incluyendo todos los pasos de la
  metodología. Cómo el modelo está basado en una simetría $Z_2$,
  implementaremos el modelo en MicrOMEGAS~\cite{Belanger:2010gh} donde
  además de calcular la densidad de reliquia, y la sección eficaz
  WIMP-nucleon, se pueden obtener todas las secciones eficaces y
  amplitudes de decaimiento con el formato SLHA, lo que facilitará las
  simulaciones para el LHC.
\end{proyecto}
\end{darkmatter}

\begin{proyecto}
  Para el modelo con con ruptura de paridad R a través de términos
  trilineales del tipo $\lambda$, sólo hemos realizado el paso
  \ref{item:5}, como un modelo que contiene un candidato inestable de
  materia oscura que explica el exceso de positrones observado por
  PAMELA. En este proyecto queremos adicionar al modelo la posibilidad
  de explicar las masas y mezclas de neutrinos y avanzar en los demás
  pasos de la metodología.
\end{proyecto}

A través de todo el proyecto se requiere de una infraestructura de
computación de alto rendimiento adecuada, que garantice una
disponibilidad permanente de poder de computo para las diferentes
simulaciones y programas computacionales que requiere el proyecto. El
grupo ha venido consolidando sus herramientas de computación con la
adquisición de un servidor Blade con capacidad para 8 nodos, aunque de
momento sólo tiene un nodo instalado. Toda esta experiencia le ha
permitido consolidar un grupo de desarrolladores y administradores de
software científico que ahora conforman La división de ciencias de la
computación del grupo la cual es ahora la encargada de administrar el
Centro Regional de Simulación y Cálculo Avanzado (CRESCA) y ofrece
servicio de computación científica.  Una de las metas prioritarias es
la unificar el poder de computo de los Grupos de Investigación de la
Universidad en un Grid Institucional que la vez éste conectado con
Grid-Colombia. En el proyecto solicitamos la renovación del contrato
al Administrador de Sistemas, que dedicaría parte de su tiempo para la
administración de CRESCA y la implementación del Grid Institucional de
computación de alto rendimiento.

\begin{ideas}
  

Para un modelo que incluya mecanismos de generación radiativa de masas de neutrinos se determina el espacio de parámetros compatible con los datos experimentales sobre física de neutrinos, se calculan los branchings de decaimiento y las secciones eficaces y se establecen correlaciones entre estos observables y los datos de física de neutrinos [7]. Para el caso de VBPR todo esto ya ha sido implementado en el programa computacional SPheno [8]. Finalmente se hace una simulación con PYTHIA [9] de la factibilidad de descubrir el modelo en aceleradores, como hemos venido haciendo con VBPR en [1], y de la viabilidad de medir las correlaciones establecidas, que es uno de los objetivos del proyecto.

Como hemos ya hecho en [3], obtendremos resultados numéricos para los flujos de rayos cósmicos  producidos en el decaimiento del CMOI usando PYTHIA. A partir de los resultados se analizará el tiempo de vida del CMOI  en términos de su masa con el fin de obtener y analizar cuáles serían las regiones del espacio de parámetros restringidas o excluidas por los experimentos de rayos cósmicos.

En este proyecto haremos una implementación del modelo de seesaw radiativo en Micromegas  con el cual se pueden hacer los cálculos de densidad de reliquia, secciones eficaces, amplitudes de decaimiento, etc. Una vez establecidas las regiones del espacio de parámetros relevantes, se procederá con el método para mecanismos de generación radiativa de masas de neutrinos descrito anteriormente.
\end{ideas}

\begin{ideas}
Los elementos teóricos y conceptuales fundamentales del proyecto que queremos desarrollar
son los siguientes:

\begin{enumerate}
\item Se asume supersimetría, y una simetría Abeliana de sabor $U(1)_H$ con uno ó varios flavones $Phi$ de cargas $H$.

\item Para un determinado término del superpotencial (\ref{eq:5}) con carga horizontal $n$ tenemos que, para $\theta=\langle\Phi\rangle$
  \begin{enumerate}
  \item Si $n\ge 0$ el acoplamiento se genera con una supresión $\theta^n$.
  \item Si $n<0$ el acoplamiento se genera con una supresión $(m_{3/2}/M)\theta^{|n|}\sim 10^{-17}\theta^{|n|}$ para un operador trilineal, y con una supresión $m_{3/2}\theta^{|n|}$ para un operador bilineal.
  \item Si $n$ es fraccionario el operador queda prohibido.
  \end{enumerate}
Usando estás posibilidades debemos fijar las cargas $H$ libres para obtener el modelo supersimétrico con las propiedades adecuadas. %Las condiciones para obtener modelos con solo términos trilineales de ruptura de paridad $R$ con violación de número leptónico ya han sido establecidas en \cite{Sierra:2009zq}.
En el caso de $n<0$ nos interesa analizar las restricciones que los experimentos sobre rayos cósmicos actuales y futuros imponen sobre dicho modelo, mientras que en el caso en que los operadores con acoplamientos $\lambda$ tengan cargas con $n\ge 0$, nos interesa explorar las predicciones que dicho modelo tenga para el LHC.
\end{enumerate}

Para un modelo que incluya mecanismos de generación radiativa de masas de neutrinos se determina el espacio de parámetros compatible con los datos experimentales sobre física de neutrinos, se calculan los branchings de decaimiento y las secciones eficaces y se establecen correlaciones entre estos observables y los datos de física de neutrinos [7]. Finalmente se hace una simulación con PYTHIA [9] de la factibilidad de descubrir el modelo en aceleradores, como hemos venido haciendo con VBPR en [1], y de la viabilidad de medir las correlaciones establecidas, que es uno de los objetivos del proyecto.

 A partir de los resultados se analizará el tiempo de vida del CMOI  en términos de su masa con el fin de obtener y analizar cuáles serían las regiones del espacio de parámetros restringidas o excluidas por los experimentos de rayos cósmicos. 

%Una vez establecidas las regiones del espacio de parámetros relevantes, se procederá con el método para mecanismos de generación radiativa de masas de neutrinos descrito anteriormente.

%El modelo de seesaw raditivo se estudiara a traves del paquete computacinal micrOMEGAs, con el cual se hacen actualmente a nivel mundial los calculos de densidad de reliquia, secciones eficaces, amplitudes de decaimiento, entre otros en modelos genéricos que pretendan tener canddatos para materia oscura. Además, desarrollaremos scripts que nos permitan hacer el estudio del comportamiento del sector de neutrinos del modelo, para poder compararlo con las observaciones.

Para el caso de modelos con ruptura de paridad $R$ del tipo $\lambda$ y carga $n\ge 0$ debemos encontrar una forma analítica que nos permita predecir el tamaño de los acoplamientos, y entonces establecer observables para el LHC que sean independientes del valor específico de $n$ y que nos permitan hacer predicciones que sean compatibles con las hipótesis de la simetría horizontal usada. Así mismo deberemos determinar la carga fraccionaria más óptima de los nuevos flavones que nos permita explicar de la mejor forma posible los resultados experimentales de oscilaciones de neutrinos y de materia oscura.
\end{ideas}

%%% Local Variables: 
%%% mode: latex
%%% TeX-master: "Ficha-2011_bsm"
%%% End: 


\section{ Resultados Esperados }
\begin{instrucciones}
CODI:  Impacto y relevancia:
¿El proyecto permite la generación de conocimiento científico o aporta a la resolución de problemas concretos de la realidad? ¿Son suficientes y adecuados los mecanismos de comunicación y socialización de resultados? 

 COLCI: Formule los resultados directos verificables que se
alcanzarán con el desarrollo de los objetivos específicos del proyecto. Estos deben ser coherentes
con los objetivos y con la metodología planteada.



  \begin{enumerate}
  \item \textbf{Relacionados con la generación de conocimiento y/o nuevos desarrollos
 tecnológicos:} Incluye resultados/productos que corresponden a nuevo
 conocimiento científico o tecnológico o a nuevos desarrollos o adaptaciones de
 tecnología que puedan verificarse a través de publicaciones científicas,
 productos o procesos tecnológicos, patentes, normas, mapas, bases de datos,
 colecciones de referencia, secuencias de macromoléculas en bases de datos de
 referencia, registros de nuevas variedades vegetales, etc.
\item \textbf{Conducentes al fortalecimiento de la capacidad científica
  nacional:} Incluye resultados/productos tales como formación de
  recurso humano a nivel profesional o de posgrado (trabajos de grado
  o tesis de maestría o doctorado sustentadas y aprobadas),
  realización de cursos relacionados con las temáticas de los
  proyectos (deberá anexarse documentación soporte que certifique su
  realización), formación y consolidación de redes de investigación
  (anexar documentación de soporte y verificación) y la construcción
  de cooperación científica internacional (anexar documentación de
  soporte y verificación).
\item \textbf{Dirigidos a la apropiación social del conocimiento:}
  Incluye aquellos resultados/productos que son estrategias o medios
  para divulgar o transferir el conocimiento o tecnologías generadas
  en el proyecto a los beneficiarios potenciales y a la sociedad en
  general. Incluye tanto las acciones conjuntas entre investigadores y
  beneficiarios como artículos o libros divulgativos, cartillas,
  videos, programas de radio, presentación de ponencias en eventos,
  entre otros.
  \end{enumerate}

  Para cada uno de los resultados/productos esperados identifique (en
  los cuadros a continuación) indicadores de verificación (ej:
  publicaciones, patentes, registros, videos, certificaciones, etc.)
  as\'\i{} como las instituciones, gremios y comunidades beneficiarias,
  nacionales o internacionales, que podrán utilizar los resultados de
  la investigación para el desarrollo de sus objetivos, políticas,
  planes o programas:

\end{instrucciones}

\begin{instrucciones}
  Los impactos no necesariamente se logran al finalizar el proyecto, ni
con la sola consecución de los resultados/productos. Los impactos
esperados son una descripción de la posible incidencia del uso de los
resultados del proyecto en función de la solución de los asuntos o
problemas estratégicos, nacionales o globales, abordados. Generalmente
se logran en el mediano y largo plazo, como resultado de la aplicación
de los conocimientos o tecnologías generadas a través del desarrollo
de una o varias líneas de investigación en las cuales se inscribe el
proyecto. Los impactos pueden agruparse, entre otras, en las
siguientes categorías: sociales, económicos, ambientales, de
productividad y competitividad. Para cada uno de los impactos
esperados se deben identificar indicadores cualitativos o
cuantitativos verificables as\'\i:
\end{instrucciones}


Nos encontramos en una época única en la que se esperan grandes
descubrimientos en el área de física de altas energías. El LHC ha
comenzado a explorar la escala del TEV desde el 2010. Hasta el 2012
estará acumulando datos a 7\,TeV de energía de centro de masas, y a
partir del 2014 comenzará a funcionar a su energía de diseño de 14
TeV. El satélite Planck fue lanzado en el 2009 y se espera que
entregue datos definitivos sobre los parámetros astronómicos en el
2012. La misión complementará y mejorará las observaciones hechas por
el WMAP. Éste año XENON100 ha entregado sus primeros datos que
comienzan a restringir las regiones del espacio de parámetros de
algunos modelos de materia oscura, y para el 2015 se espera que los
datos de XENON1T cubran completamente la región de materia oscura
predicha por el MSSM restringido. AMS-02 es un detector de rayos
cósmicos que será llevado a la estación espacial internacional a bordo
de la última misión del transbordador Endeavor el 29 de abril de
2011. Se espera que compruebe y mejore las medidas de PAMELA sobre
exceso en el flujo de positrones de rayos cósmicos. Hemos entrado en
una nueva era de la física que, de obtener los resultados esperados,
combinaría los descubrimientos de nuevas partículas, a los cada vez
mejor establecidos resultados de física de neutrinos y observaciones
cosmológicas sobre materia oscura. El resultado de este proyecto es
aportar a esta área de la ciencia con nuevas propuestas de señales
para ser buscadas en estos detectores (subterráneos y en el espacio) y
con la interpretación de los resultados que surjan de ellos en
términos de los modelos propuestos, los cuales añaden partículas
nuevas al modelo estándar de la partículas elementales. Estos se
reflejarán en la publicación de al menos dos artículos científicos en
el área, y en la presentación de los resultados en al menos dos
conferencias internacionales.


De especial importancia es el nuevo paradigma científico que surgirá
de la combinación de todos esos resultados experimentales. Por ejemplo
el descubrimiento en el LHC de la partícula escalar elemental predicha
por el Modelo Estándar, establecería finalmente las teorías gauge con
rompimiento espontáneo de simetría como el principio fundamental para
describir las interacciones subatómicas, redondeando décadas de
desarrollo científico.  El estudio detallado de sus propiedades
acompañado posiblemente de señales de nueva física marcará el camino
para encontrar el mecanismo de generación de masas y mezclas de
neutrinos, así como la determinación de la partícula que compone la
materia oscura del Universo. A más tardar al finalizar ésta década se
espera tener respuesta a estos interrogantes. Todo esto tendrá un
impacto en la enseñanza de la física a todos los niveles. Además el
descubrimiento del Higgs, que sería la primera partícula escalar
elemental, daría un mejor fundamento teórico a los modelos
inflacionarios en cosmología y a la interpretación de la energía
oscura como la causante de la expansión acelerada del Universo. El
otro escenario posible en el que no se encuentre al Higgs del modelo
estándar, debe dar lugar a datos experimentales suficientes para
dilucidar cual es el realmente el mecanismo de ruptura de la simetría
electrodébil. Nuestro grupo es el llamado a difundir estos avances en
nuestra sociedad, como lo ha venido haciendo a través de conferencias
y cursos de extensión en los últimos años. Es importante que nuestro
país siga participando en el desarrollo de la física fundamental, no
sólo con la participación de grupos teóricos como el nuestro, sino
también con grupos experimentales de física de altas energías como lo
viene haciendo en las colaboraciones ATLAS y CMS del LHC con grupos de
la Universidad Antonio Nariño y de los Andes respectivamente. Con
ellos, y con los otros grupos teóricos del país hemos venido
colaborando y organizando congresos en el área en los últimos años
para consolidar ésta área de investigación en el país.

El principal aporte un Grupo como el nuestro al desarrollo del país es
la formación de talento humano con capacidad de hacer investigación
científica al más alto nivel. Para ello es prioritario que nuestro
Grupo siga produciendo productos de gran impacto en la comunidad
mundial de física de altas energías con participación de nuestros
estudiantes de pregrado y posgrado.  Aunque de momento los doctores
que formamos son rápidamente reabsorbidos en el ámbito académico, 
esperamos que a futuro, como pasa en otros países donde profesionales
de este tipo son muy apreciados en empresas de innovación tecnológica,
los nuestros puedan llegar a hacer aportes significativos a otros
sectores de la sociedad.


De ser aprobado, éste proyecto nos permitiría participar en esta
excitante era del desarrollo científico que coincide con
los primeros años de funcionamiento del LHC.  Se espera que entre los
resultados de los próximos años LHC, no solo esté el del
descubrimiento del Higgs, sino también de alguna señal de física más
allá del modelo estándar que explique los problemas fenomenológicos y
teóricos del Modelo Estándar. En los próximos años también se espera
que los experimentos de detección directa, o indirecta a través de
rayos cósmicos, entreguen una evidencia definitiva de materia oscura.



%Final
Publicar al menos 3 artículos internacionales con algunos de los siguientes resultados:

Determinación de la factibilidad del LHC para determinar las correlaciones de VBPR entre el cociente de branchings de neutralino a W mu y W tau con el ángulo de mezcla atmosférico de neutrinos, así como de la correlación entre la longitud de decaimiento del neutralino y la diferencia de masa al cuadrado atmosférica.

Determinación de acoplamientos, vida media y masa del gravitino como CMOI en modelos de ruptura trilineal de paridad R (RTPR) con violación de número leptónico, que pueda explicar el exceso de positrones en rayos cósmicos. Predicciones de flujo de positrones y rayos gama del modelo, para experimentos futuros de rayos cósmicos. 

Determinación de las restricciones que los diferentes experimentos de rayos cósmicos 
presentes y futuros imponen sobre el neutralino como CMOI. 

Determinación de las señales en el LHC para 1) Regiones del espacio de parámetros compatible con física de neutrinos y densidad de materia oscura del seesaw radiativo. 

Formación 1 estudiante de maestría y 1 de doctorado.

\begin{ideas}
  
%Determinación de la factibilidad del LHC para determinar las correlaciones de VBPR entre el cociente de branchings de neutralino a W mu y W tau con el ángulo de mezcla atmosférico de neutrinos, así como de la correlación entre la longitud de decaimiento del neutralino y la diferencia de masa al cuadrado atmosférica.

Determinación del flujo de rayos gamas y su confrontación con lo observado en experimentos de rayos cósmicos en modelos supersimétricos donde el gravitino es el CMOI.

Determinación de acoplamientos, vida media y masa del gravitino como CMOI en modelos de ruptura trilineal de paridad R (RTPR) con violación de número leptónico, que pueda explicar el exceso de positrones en rayos cósmicos. Predicciones de flujo de positrones y rayos gama del modelo para experimentos futuros de rayos cósmicos. 

%Determinación de las restricciones que los diferentes experimentos de rayos cósmicos 
%presentes y futuros que se imponen sobre el neutralino como CMOI. 

Determinación de las señales en el LHC para 1) Modelos de RTPR con violación de número bariónico y mecanismo see-saw para masas de neutrinos, inducida por una simetría Abeliana anómala U(1). 2) Regiones del espacio de parámetros compatible con física de neutrinos y densidad de materia oscura del seesaw radiativo.

% Obtener algunos canales visibles que involucren materia oscura en el LHC, y predecir el flujo de rayos gamma en la galaxia producto de la aniquilacion de materia oscura.

%De otro lado, esperamos construir el primer modelo estándar supersimétrico con una simetría horizontal anómala $U(1)_H$ que se pueda constrastar directamente con resultados experimentales de aceleradores. Así mismo como formular el primer modelo con violación de número leptónico a través de términos $\lambda$ que a la vez de cuenta de las masas y mezclas de neutrinos.

Esperemos obtener predicciones específicas para modelos supersimétricos con contenido mínimo de partículas y una simetría horizontal anómala $U(1)_H$ que puedan comprobarse directamente en experimentos de rayos cósmicos o de aceleradores de partículas.

Formación 1 estudiante de maestría y 1 de doctorado.
\end{ideas}

%%% Local Variables: 
%%% mode: latex
%%% TeX-master: "Ficha-2011_bsm"
%%% End: 

\section{ Cronograma}



\begin{itemize}

\item {\bf meses 1-3 :} Se revisará la literatura existente sobre
  simetrías horizontales, además de revisar la literatura de  las diferentes colaboraciones de experimentos de rayos cósmicos.
  %  En esta fase también se formará el
  % estudiante de Maestría en los tópicos específicos
  % relativos a la investigación, así como permitirle tomar parte activa
  % en todas las fases de desarrollo del proyecto.

\item {\bf meses 4-11 : }   Se establecerá el background de rayos cósmicos, asumiendo que las anomalías actuales se pueden explicar a partir de fenómenos astrofísicos. Se hará una programa de simulación en \texttt{Pythia} \cite{Sjostrand:2006za} para los modelos con ruptura trilineal de $R$--paridad.
%a través de términos $\lambda$, así como para el modelo con ruptura de paridad $R$ a través de términos $\lambda''$.
Con este programa se calculará el flujo de positrones, electrones, y antiprotones, en cada modelo y se comparará con el background astrofísico para establecer las regiones de tiempo de vida media y masas de neutralinos que se pueden excluir a partir del los datos actuales. También se evaluará el impacto de las futuras medidas de positrones y electrones por parte PAMELA~\cite{Adriani:2008zr}, Fermi--LAT\cite{Abdo:2009zk} y AMS-02~\cite{ams:2009} en los modelos estudiados. Así mismo se evaluara el background de rayos gamma y se hará un programa de simulación en \texttt{Pythia} para el espectro de rayos gamas provenientes de desintegración de neutralinos en taus y hadrones.
%los cuales generan fotones en sus cadenas de decaimiento.
Estos resultados se compararán con las medidas preliminares\footnote{T. Porter (Fermi-LAT) (2009), Talk given at TeV Particle Astrophysics (TeVPA), July 13-17, 2009.} y con las medidas que serán reportadas por la colaboración Femi--LAT en los próximos meses.

Paralelamente se establecerán las condiciones sobre los cuatro cargas $H$ libres del modelo estándar supersimétrico con una simetría $U(1)_H$ con violación de número bariónico a través de términos trilineales $\lambda''$. Usando las restricciones experimentales sobre los acoplamientos individuales $\lambda''$~\cite{Barbier:2004ez}, se establecerán los cocientes entre acoplamientos $\lambda''$ que predice el modelo. A partir de dichos cocientes se establecerán las posibles señales de decaimiento de la partícula supersimétrica más liviana,  que permitan comprobar el modelo propuesto en el LHC. Adicionalmente se intentará implementar el mecanismo seesaw para explicar las masas y mezclas de neutrinos, introduciendo un nuevo flavón de carga fraccionaria.

\item {\bf meses 11-18 : } Se analizarán los resultados importantes y se escribirán dos artículos, uno con las restricciones de experimentos de rayos cósmicos en modelos supersimétricos de ruptura de $R$--paridad con materia oscura inestable, y otro sobre las predicciones para el LHC del modelo estándar supersimétrico con una simetría $U(1)_H$ con violación de número bariónico a través de términos trilineales $\lambda''$.

\end{itemize}

%%% Local Variables: 
%%% mode: latex
%%% TeX-master: "Ficha-2011_bsm"
%%% End: 


\section{ Compromisos y estrategia de comunicación }
\begin{instrucciones}
  Formule los resultados directos verificables que se
alcanzarán con el desarrollo de los objetivos específicos del proyecto. Estos deben ser coherentes
con los objetivos y con la metodología planteada.


  \begin{enumerate}
  \item \textbf{Relacionados con la generación de conocimiento y/o nuevos desarrollos
 tecnológicos:} Incluye resultados/productos que corresponden a nuevo
 conocimiento científico o tecnológico o a nuevos desarrollos o adaptaciones de
 tecnología que puedan verificarse a través de publicaciones científicas,
 productos o procesos tecnológicos, patentes, normas, mapas, bases de datos,
 colecciones de referencia, secuencias de macromoléculas en bases de datos de
 referencia, registros de nuevas variedades vegetales, etc.
\item \textbf{Conducentes al fortalecimiento de la capacidad científica
  nacional:} Incluye resultados/productos tales como formación de
  recurso humano a nivel profesional o de posgrado (trabajos de grado
  o tesis de maestría o doctorado sustentadas y aprobadas),
  realización de cursos relacionados con las temáticas de los
  proyectos (deberá anexarse documentación soporte que certifique su
  realización), formación y consolidación de redes de investigación
  (anexar documentación de soporte y verificación) y la construcción
  de cooperación científica internacional (anexar documentación de
  soporte y verificación).
\item \textbf{Dirigidos a la apropiación social del conocimiento:}
  Incluye aquellos resultados/productos que son estrategias o medios
  para divulgar o transferir el conocimiento o tecnologías generadas
  en el proyecto a los beneficiarios potenciales y a la sociedad en
  general. Incluye tanto las acciones conjuntas entre investigadores y
  beneficiarios como artículos o libros divulgativos, cartillas,
  videos, programas de radio, presentación de ponencias en eventos,
  entre otros.
  \end{enumerate}

  Para cada uno de los resultados/productos esperados identifique (en
  los cuadros a continuación) indicadores de verificación (ej:
  publicaciones, patentes, registros, videos, certificaciones, etc.)
  as\'\i{} como las instituciones, gremios y comunidades beneficiarias,
  nacionales o internacionales, que podrán utilizar los resultados de
  la investigación para el desarrollo de sus objetivos, políticas,
  planes o programas:

\end{instrucciones}

%Final
%\begin{itemize}
 Artículo en revistas ISI (Web of Science) o Scopus. 
 Otro producto de nuevo conocimiento, de autoría o en coautoría
  del estudiante: artículo en revista ISI (Web of Science) o Scopus.
 Un artículo en revista de divulgación.
 Proyecto de investigación presentado a una entidad o
  convocatoria externa.
 Pasantía de investigación para el estudiante de Doctorado.
 Participaciòn con ponencia en evento académico nacional o
  internacional.
%\end{itemize}

\begin{ideas}
  


Dirección de un trabajo de investigación para optar al título de Magíster.

Opcional: Participación con ponencia en evento académico nacional :
    13° Encuentro Colombiano de Matemática Educativa 2012 - ASCOLMEPublicar al menos 4 artículos internacionales con algunos de los siguientes resultados:

Determinación de la factibilidad del LHC para determinar las correlaciones de VBPR entre el cociente de branchings de neutralino a W mu y W tau con el ángulo de mezcla atmosférico de neutrinos, así como de la correlación entre la longitud de decaimiento del neutralino y la diferencia de masa al cuadrado atmosférica.

Determinación del flujo de rayos gamas y su confrontación con lo observado en experimentos de rayos cósmicos en modelos supersimétricos donde el gravitino es el CMOI con masa de hasta 80 GeV. 

Determinación de acoplamientos, vida media y masa del gravitino como CMOI en modelos de ruptura trilineal de paridad R (RTPR) con violación de número leptónico, que pueda explicar el exceso de positrones en rayos cósmicos. Predicciones de flujo de positrones y rayos gama del modelo para experimentos futuros de rayos cósmicos. 

Determinación de las restricciones que los diferentes experimentos de rayos cósmicos 
presentes y futuros imponen sobre el neutralino como CMOI. 

Determinación de las señales en el LHC para 1) Modelos de RTPR con violación de número bariónico y mecanismo see-saw para masas de neutrinos, inducida por una simetría Abeliana anómala U(1). 2) Regiones del espacio de parámetros compatible con física de neutrinos y densidad de materia oscura del seesaw radiativo. 3) Modelos 3-3-1 con cuatro familias

Formación 1 estudiante de maestría y 1 de doctorado.
\end{ideas}

\begin{ideas}

Se asume los siguientes compromisos:

\begin{itemize}

\item Por lo menos tres artículos en revista internacional indexada
  A1. Y un artículo divulgativo.

\item Ponencia en evento internacional donde se presentaran los resultados de le
  investigación.

\item
Por lo menos un seminario en el Instituto de Física de la Universidad
de Antioquia, entregado por parte del estudiante de Doctorado.

\end{itemize}
\end{ideas}


%%% Local Variables: 
%%% mode: latex
%%% TeX-master: t
%%% End: 

\section{ Funciones de los Estudiantes en Formación }



En el proyecto se incluye un estudiante de doctorado que se encuentra realizando
el segundo semestre.

Además de participar en todos los pasos del desarrollo de la
investigación, el estudiante  tendrán las siguientes
tareas especificas:

\begin{itemize}
\item Establecer los diferentes conjuntos de parámetros $\lambda$ para valores diferentes del la carga $H$ de los operadores correspondientes. En caso de que alguno de ellos resulte con carga $n$ negativa, calcular el acoplamiento diferente de cero que surge después de hacer la rotación del potencial de K\"ahler.
\item Encontrar la carga fraccionaria óptima de los nuevos flavones para explicar las masas y mezclas de los neutrinos a través del mecanismo seesaw en el modelo estándar supersimétrico con la simetría anómala $U(1)_H$ y violación de número leptónico a través de términos trilineales $\lambda$.
\item Fijamdo el gravitino como la LSP del modelo, determinar los perfiles de background para electrones, positrones.
\item Elaborar los programas en \texttt{Pythia} para calculo del flujo de electrones, positrones.
\item Establecer las regiones de tiempo de vida media y masa del gravitino excluidas para los modelos bajo consideración.
\item Participar activamente en la elaboración del artpículo al respecto.
\item Participar en todas la actividades del estudiante de maestría
\item Extender los programas computacionales para la simulación de señales supersimetrícas con ruptura bilineal de paridad R para analisar las nuevas correlaciones propuestas en el proyecto.
\item Participar activamente en el análisis de resultados y en la
  publicación del artículo al respecto.
\end{itemize}

%%% Local Variables: 
%%% mode: latex
%%% TeX-master: "Ficha-2011_bsm"
%%% End: 


\begin{instrucciones}
  Establecer la importancia y aporte de la investigación
propuesta en función de la generación de conocimiento, el desarrollo tecnológico, la innovación y
la solución a problemas nacionales.
\end{instrucciones}


\begin{instrucciones}
  Estos deben ser coherentes con los objetivos específicos y con la
  metodología planteada.\\
  Los resultados/productos pueden clasificarse en tres categorías:
  \begin{enumerate}
  \item \textbf{Relacionados con la generación de conocimiento y/o nuevos desarrollos
 tecnológicos:} Incluye resultados/productos que corresponden a nuevo
 conocimiento científico o tecnológico o a nuevos desarrollos o adaptaciones de
 tecnología que puedan verificarse a través de publicaciones científicas,
 productos o procesos tecnológicos, patentes, normas, mapas, bases de datos,
 colecciones de referencia, secuencias de macromoléculas en bases de datos de
 referencia, registros de nuevas variedades vegetales, etc.
\item \textbf{Conducentes al fortalecimiento de la capacidad científica
  nacional:} Incluye resultados/productos tales como formación de
  recurso humano a nivel profesional o de posgrado (trabajos de grado
  o tesis de maestría o doctorado sustentadas y aprobadas),
  realización de cursos relacionados con las temáticas de los
  proyectos (deberá anexarse documentación soporte que certifique su
  realización), formación y consolidación de redes de investigación
  (anexar documentación de soporte y verificación) y la construcción
  de cooperación científica internacional (anexar documentación de
  soporte y verificación).
\item \textbf{Dirigidos a la apropiación social del conocimiento:}
  Incluye aquellos resultados/productos que son estrategias o medios
  para divulgar o transferir el conocimiento o tecnologías generadas
  en el proyecto a los beneficiarios potenciales y a la sociedad en
  general. Incluye tanto las acciones conjuntas entre investigadores y
  beneficiarios como artículos o libros divulgativos, cartillas,
  videos, programas de radio, presentación de ponencias en eventos,
  entre otros.
  \end{enumerate}

  Para cada uno de los resultados/productos esperados identifique (en
  los cuadros a continuación) indicadores de verificación (ej:
  publicaciones, patentes, registros, videos, certificaciones, etc.)
  as\'\i{} como las instituciones, gremios y comunidades beneficiarias,
  nacionales o internacionales, que podrán utilizar los resultados de
  la investigación para el desarrollo de sus objetivos, políticas,
  planes o programas:

\end{instrucciones}

%Final
Las evidencias más sólidas de nueva física son las medidas de masas y mezclas de neutrinos y de materia oscura.

La enorme actividad experimental actual en física de partículas intenta determinar los modelos que incorporan esa nueva física. Esto requiere un esfuerzo teórico enorme al que esperamos aportar con este proyecto. En los últimos meses se ha reportando un exceso de rayos cósmicos de electrones y positrones.  Esto ha dado lugar a un sinnúmero de publicaciones tratando de explicar su origen [2]. Los diferentes experimentos de detección de materia oscura han comenzado a obtener resultados prometedores. El LHC ya ha comenzado a producir colisiones a una energía de 7 TeV.

El presente proyecto se centra en proponer modelos que tengan consecuencias medibles en todos estos experimentos, y encontrar modelos que expliquen los resultados de cualquiera de ellos. Pretendemos establecer observables que no sólo permitan descubrir [1] sino también  comprobar si el VBPR es el mecanismo de generación de masas de neutrinos, así como estudiar las implicaciones de modelos con un candidato de materia oscura. 

%%% Local Variables: 
%%% mode: latex
%%% TeX-master: Ficha-2011_bsm
%%% End: 


\section{ Presupuesto}
\begin{instrucciones}
  ¿Los rubros son pertinentes? ¿Los montos son los adecuados para cumplir los objetivos del proyecto? ¿El número de investigadores y el tiempo de dedicación son adecuados?
\end{instrucciones}
Ver documento adjunto 

%\newpage
\bibliographystyle{apsrev4-1long}
\bibliography{susy}%%,soko,snova,nu-rev06,parke-ref}

\end{document}


%%% Local Variables: 
%%% mode: latex
%%% TeX-master: t
%%% End: 
