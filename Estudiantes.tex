\section{ Funciones de los Estudiantes en Formación }



En el proyecto se incluye un estudiante de doctorado que se encuentra realizando
el segundo semestre.

Además de participar en todos los pasos del desarrollo de la
investigación, el estudiante  tendrán las siguientes
tareas especificas:

\begin{itemize}
\item Establecer los diferentes conjuntos de parámetros $\lambda$ para valores diferentes del la carga $H$ de los operadores correspondientes. En caso de que alguno de ellos resulte con carga $n$ negativa, calcular el acoplamiento diferente de cero que surge después de hacer la rotación del potencial de K\"ahler.
\item Encontrar la carga fraccionaria óptima de los nuevos flavones para explicar las masas y mezclas de los neutrinos a través del mecanismo seesaw en el modelo estándar supersimétrico con la simetría anómala $U(1)_H$ y violación de número leptónico a través de términos trilineales $\lambda$.
\item Fijamdo el gravitino como la LSP del modelo, determinar los perfiles de background para electrones, positrones.
\item Elaborar los programas en \texttt{Pythia} para calculo del flujo de electrones, positrones.
\item Establecer las regiones de tiempo de vida media y masa del gravitino excluidas para los modelos bajo consideración.
\item Participar activamente en la elaboración del artpículo al respecto.
\item Participar en todas la actividades del estudiante de maestría
\item Extender los programas computacionales para la simulación de señales supersimetrícas con ruptura bilineal de paridad R para analisar las nuevas correlaciones propuestas en el proyecto.
\item Participar activamente en el análisis de resultados y en la
  publicación del artículo al respecto.
\end{itemize}

%%% Local Variables: 
%%% mode: latex
%%% TeX-master: "Ficha-2011_bsm"
%%% End: 

