\section{ Metodología Propuesta }

Los elementos teóricos fundamentales del proyecto que queremos desarrollar
son los siguientes:

\begin{enumerate}
\item Se asume supersimetría,   y una simetría Abeliana de sabor $U(1)_H$ con un sólo flavón, $\Phi$, de carga $H$ $-1$.

\item Para un determinado término del superpotencial (\ref{eq:5}) con carga horizontal $n$ tenemos que, para $\theta=\langle\Phi\rangle$
  \begin{enumerate}
  \item Si $n\ge 0$ el acoplamiento se genera con una supresión $\theta^n$.
  \item Si $n<0$ el acoplamiento se genera con una supresión $(m_{3/2}/M)\theta^{|n|}\sim 10^{-17}\theta^{|n|}$ para un operador trilineal, y con una supresión $m_{3/2}\theta^{|n|}$ para un operador bilineal.
  \item Si $n$ es fraccionario el operador queda prohibido.
  \end{enumerate}
Usando estás posibilidades debemos fijar las cargas $H$ libres para obtener el modelo supersimétrico con las propiedades adecuadas. Las condiciones para obtener modelos con solo términos trilineales de ruptura de paridad $R$ con violación de número leptónico ya han sido establecidas en \cite{Sierra:2009zq}.

En este proyecto debemos establecer las condiciones para obtener modelos con únicamente acoplamientos de $\lambda''$. En el caso de $n<0$ nos interesa analizar las restricciones que los experimentos sobre rayos cósmicos actuales y futuros imponen sobre dicho modelo, mientras que en el caso en que los operadores con acoplamientos $\lambda''$ tengan cargas con $n\ge 0$, nos interesa explorar las predicciones que dicho modelo tenga para el LHC.
\end{enumerate}

Para el caso de modelos con un candidato a materia oscura inestable la metodología es la siguiente:
En este escenario, asumiremos que el flujo de positrones y rayos gamma astrofísicos constituirían un nuevo background para las posibles señales de detección de materia oscura \cite{Choi:2009qc}. En colaboración con el Dr. Carlos Yaguna, uno de los autores de la referencia anterior, se compararán los perfiles de dicho background  con las simulaciones hechas en \texttt{Pythia} \cite{Sjostrand:2006za} del flujo rayos cósmicos para modelos con ruptura trilineal de $R$--paridad donde los neutralinos son candidatos de materia oscura inestable. A partir de los flujos que superen de forma significativa el nuevo background se obtendrán las regiones de exclusión del  tiempo de vida media, y de masa para la partícula de materia oscura. Con el nuevo background  astrofísico, y con una metodología similar a la usada en \cite{Choi:2009qc} para  el caso de aniquilación de materia oscura, también se estudiará la posibilidad de detectar positrones que se originen a partir de materia oscura inestable, en futuros experimentos de medidas de flujos de positrones tales como el experimento AMS-02~\cite{ams:2009}. AMS-02 tomará datos durante tres años y espera medir el espectro de positrones hasta los $300\,$GeV. De esta comparación esperamos obtener el rango de tiempos de vida media y masas que se podrán explorar en AMS-02, para modelos con materia oscura inestable.


Para el caso de modelos con ruptura de paridad $R$ del tipo $\lambda''$ y carga $n\ge 0$ debemos encontrar una forma analítica que nos permita predecir el tamaño de los acoplamientos, y entonces establecer observables para el LHC que sean independientes del valor específico de $n$ y que nos permitan hacer predicciones que sean compatibles con las hipótesis de la simetría horizontal usada. Así mismo deberemos determinar la carga fraccionaria más óptima de un nuevo flavón que nos permita explicar de la mejor forma posible los resultados experimentales de oscilaciones de neutrinos.


%%% Local Variables: 
%%% mode: latex
%%% TeX-master: Ficha-2011_bsm
%%% End: 

