\section{Metodología Propuesta}
\begin{instrucciones}
  Esta debe reflejar la estructura lógica y el rigor científico del
  proceso de investigación, empezando por la elección de un enfoque
  metodológico específico y finalizando con la forma como se van a
  analizar, interpretar y presentar los resultados.
\end{instrucciones}
%%%

%Final
A lo largo de nuestra experiencia con el modelo con violación de
paridad R, hemos logrado implementar todos los pasos de la metodología
usada en física de partículas para estudiar modelos de física más allá
del modelo estándar.  Ésta parte desde la formulación misma del modelo
hasta su estudio real en los aceleradores de partículas por parte de
los grupos experimentales. Esta metodología también la hemos logrado
aplicar con éxito a otros modelos. A continuación se muestran los
pasos a seguir usando como paradigma el modelo con ruptura bilineal de
paridad R y ejemplificando cada paso con referencias concretas.
\begin{enumerate}
\item Se construye un modelo que solucione un problema fenomenológico
  del modelo estándar, como el problema de las masas de neutrinos
  \cite{Hirsch:2000ef}, o la materia oscura (o ambos
  \cite{Hirsch:2005ag}), mostrando bajo que mecanismo específico se
  soluciona el problema en cuestión. La solución del problema requiere
  introducir partículas adicionales las cuales pueden buscarse en
  detectores de partículas.
  \label{item:5}
\item Se determina el espacio de parámetros compatible con los datos
  experimentales, se calculan los branchings de decaimiento y las
  secciones eficaces y se establecen correlaciones entre estos
  observables en aceleradores con otros observables. En el caso de
  modelos con mecanismos de generación radiativa de masas de neutrinos
  las correlaciones se buscan con los datos de oscilaciones de
  neutrinos \cite{Diaz:2003as}, y en el caso de materia oscura con
  experimentos de detección directa e indirecta \cite{Choi:2010xn}.
  \label{item:6}
\item Se desarrolla un programa computacional que dado los parámetros
  de entrada del modelo entregue las secciones eficaces y los
  branching de decaimiento en un formato adecuado (el SLHA
  \cite{0801.0045}) para ser usado luego por programas de Montecarlo
  de generación de eventos como PYTHIA \cite{hep-ph/0603175}. El
  modelo con ruptura bilineal de paridad R ha sido implementado en el
  programa computacional Spheno~\cite{Porod:2003um} (el cual tiene
  implementado el formato de salida SLHA version 2).
  \label{item:7}
\item Se divide el espacio de n-parámetros establecido en el paso
  \ref{item:6} en una malla n-dimensional. Para cada punto se corre el
  programa computacional desarrollado en el paso \ref{item:7} y los
  correspondiente datos de salida se pasan a PYTHIA usando la
  interfase SLHA. Con PYTHIA en cada punto se realiza una simulación
  que consiste en generar aleatoriamente eventos de acuerdo a la
  geometría y características de detectores específicos, para
  determinar la factibilidad de descubrir las señales en aceleradores
  \cite{Magro:2003zb,deCampos:2005ri,deCampos:2007bn,deCampos:2008ic,deCampos:2008re},
  o en experimentos de detección directa o indirecta
  \cite{Choi:2010xn} de materia oscura. Este paso requiere de
  herramientas de computación de alto rendimiento en clusters de
  computadores.
  \label{item:8}
\item Como modelos muy diferentes pueden dar lugar a las mismas
  señales en detectores, se debe hacer también simulaciones en PYTHIA
  del nivel de precisión con el que se pueden determinar observables
  en aceleradores que se puedan correlacionar con observables de otros
  experimentos. A partir del valor experimental del ángulo de mezcla
  atmosférico, en el modelo con ruptura bilineal de paridad R por
  ejemplo, se predice que el decaimiento del neutralino a muon debe
  ser muy similar al decaimiento del neutralino a tau. En
  \cite{DeCampos:2010yu} hemos determinado el nivel de precisión con
  el que se pude medir esta correlación en el detector ATLAS del LHC.
  \label{item:9}
\end{enumerate}
Las simulaciones se realizan con el fin de desarrollar todas las
herramientas necesarias para que los grupos experimentales de los
aceleradores puedan comparar los datos obtenidos con modelos
específicos y puedan descubrir o poner cotas sobre las nuevas
partículas propuestas, o en el caso de detección directa o indirecta
de materia oscura donde las señales están preestablecidas, para poder
comparar con los datos obtenidos o que se pueden obtener. En el caso
del modelo con ruptura bilineal de paridad R, ya se han hecho
búsquedas en Tevatron~\cite{0802.3887} y se están implementando 
búsquedas en el LHC en grupos de trabajo de ATLAS y LHCb.

\begin{proyecto}
  En el marco de este proyecto se simularán dos correlaciones
  adicionales que existen el modelo con ruptura bilineal de paridad R:
  Cuando el neutralino es la LSP, los decaimientos a tres cuerpos
  mediados por sfermiones con muones y electrones en los estados
  finales, están correlaciones con el ángulo de mezcla solar, y la
  longitud de decaimiento del neutralino está correlacionada con la
  diferencia de masa atmosférica.
\end{proyecto}

\begin{proyecto}
  En un proyecto de menor cuantía presentado por el investigador
  asociado al Grupo, Oscar Zapata, se realizará el paso \ref{item:6}
  de determinar las regiones del espacio de parámetros donde el modelo
  con ruptura bilineal de paridad R puede explicar tanto las masas de
  neutrinos como la densidad de reliquia de materia oscura cuando la
  LSP es el gravitino. Para las regiones establecidas, en este proyecto realizaremos los pasos \ref{item:7} y \ref{item:8}.
\end{proyecto}


\begin{proyecto}
  Para el seesaw radiativo, en la literatura básicamente sólo se ha
  realizado el paso \ref{item:5}, aunque para algunas regiones del
  espacio de parámetros donde la partícula más liviana de paridad
  impar (LOP de sus siglas en inglés) es escalar se ha llegado hasta
  el paso \ref{item:8}~\cite{0708.2839}. En este proyecto haremos un
  estudio sistemático del modelo incluyendo todos los pasos de la
  metodología. Cómo el modelo está basado en una simetría $Z_2$,
  implementaremos el modelo en MicrOMEGAS~\cite{1004.1092} donde
  además de calcular la densidad de reliquia, y la sección eficaz
  WIMP-nucleon, se pueden obtener todas las secciones eficaces y
  amplitudes de decaimiento con el formato SLHA, lo que facilitará las
  simulaciones para el LHC.
\end{proyecto}

\begin{proyecto}
  Para el modelo con con ruptura de paridad R a través de términos
  trilineales del tipo $\lambda$, sólo hemos realizado el paso
  \ref{item:5}, como un modelo que contiene un candidato inestable de
  materia oscura que explica el exceso de positrones observado por
  PAMELA. En este proyecto queremos adicionar al modelo la posibilidad
  de explicar las masas y mezclas de neutrinos y avanzar en los demás
  pasos de la metodología.
\end{proyecto}





\begin{ideas}
  

Para un modelo que incluya mecanismos de generación radiativa de masas de neutrinos se determina el espacio de parámetros compatible con los datos experimentales sobre física de neutrinos, se calculan los branchings de decaimiento y las secciones eficaces y se establecen correlaciones entre estos observables y los datos de física de neutrinos [7]. Para el caso de VBPR todo esto ya ha sido implementado en el programa computacional SPheno [8]. Finalmente se hace una simulación con PYTHIA [9] de la factibilidad de descubrir el modelo en aceleradores, como hemos venido haciendo con VBPR en [1], y de la viabilidad de medir las correlaciones establecidas, que es uno de los objetivos del proyecto.

Como hemos ya hecho en [3], obtendremos resultados numéricos para los flujos de rayos cósmicos  producidos en el decaimiento del CMOI usando PYTHIA. A partir de los resultados se analizará el tiempo de vida del CMOI  en términos de su masa con el fin de obtener y analizar cuáles serían las regiones del espacio de parámetros restringidas o excluidas por los experimentos de rayos cósmicos.

En este proyecto haremos una implementación del modelo de seesaw radiativo en Micromegas  con el cual se pueden hacer los cálculos de densidad de reliquia, secciones eficaces, amplitudes de decaimiento, etc. Una vez establecidas las regiones del espacio de parámetros relevantes, se procederá con el método para mecanismos de generación radiativa de masas de neutrinos descrito anteriormente.
\end{ideas}

\begin{ideas}
Los elementos teóricos y conceptuales fundamentales del proyecto que queremos desarrollar
son los siguientes:

\begin{enumerate}
\item Se asume supersimetría, y una simetría Abeliana de sabor $U(1)_H$ con uno ó varios flavones $Phi$ de cargas $H$.

\item Para un determinado término del superpotencial (\ref{eq:5}) con carga horizontal $n$ tenemos que, para $\theta=\langle\Phi\rangle$
  \begin{enumerate}
  \item Si $n\ge 0$ el acoplamiento se genera con una supresión $\theta^n$.
  \item Si $n<0$ el acoplamiento se genera con una supresión $(m_{3/2}/M)\theta^{|n|}\sim 10^{-17}\theta^{|n|}$ para un operador trilineal, y con una supresión $m_{3/2}\theta^{|n|}$ para un operador bilineal.
  \item Si $n$ es fraccionario el operador queda prohibido.
  \end{enumerate}
Usando estás posibilidades debemos fijar las cargas $H$ libres para obtener el modelo supersimétrico con las propiedades adecuadas. %Las condiciones para obtener modelos con solo términos trilineales de ruptura de paridad $R$ con violación de número leptónico ya han sido establecidas en \cite{Sierra:2009zq}.
En el caso de $n<0$ nos interesa analizar las restricciones que los experimentos sobre rayos cósmicos actuales y futuros imponen sobre dicho modelo, mientras que en el caso en que los operadores con acoplamientos $\lambda$ tengan cargas con $n\ge 0$, nos interesa explorar las predicciones que dicho modelo tenga para el LHC.
\end{enumerate}

Para un modelo que incluya mecanismos de generación radiativa de masas de neutrinos se determina el espacio de parámetros compatible con los datos experimentales sobre física de neutrinos, se calculan los branchings de decaimiento y las secciones eficaces y se establecen correlaciones entre estos observables y los datos de física de neutrinos [7]. Finalmente se hace una simulación con PYTHIA [9] de la factibilidad de descubrir el modelo en aceleradores, como hemos venido haciendo con VBPR en [1], y de la viabilidad de medir las correlaciones establecidas, que es uno de los objetivos del proyecto.

 A partir de los resultados se analizará el tiempo de vida del CMOI  en términos de su masa con el fin de obtener y analizar cuáles serían las regiones del espacio de parámetros restringidas o excluidas por los experimentos de rayos cósmicos. 

%Una vez establecidas las regiones del espacio de parámetros relevantes, se procederá con el método para mecanismos de generación radiativa de masas de neutrinos descrito anteriormente.

%El modelo de seesaw raditivo se estudiara a traves del paquete computacinal micrOMEGAs, con el cual se hacen actualmente a nivel mundial los calculos de densidad de reliquia, secciones eficaces, amplitudes de decaimiento, entre otros en modelos genéricos que pretendan tener canddatos para materia oscura. Además, desarrollaremos scripts que nos permitan hacer el estudio del comportamiento del sector de neutrinos del modelo, para poder compararlo con las observaciones.

Para el caso de modelos con ruptura de paridad $R$ del tipo $\lambda$ y carga $n\ge 0$ debemos encontrar una forma analítica que nos permita predecir el tamaño de los acoplamientos, y entonces establecer observables para el LHC que sean independientes del valor específico de $n$ y que nos permitan hacer predicciones que sean compatibles con las hipótesis de la simetría horizontal usada. Así mismo deberemos determinar la carga fraccionaria más óptima de los nuevos flavones que nos permita explicar de la mejor forma posible los resultados experimentales de oscilaciones de neutrinos y de materia oscura.
\end{ideas}

%%% Local Variables: 
%%% mode: latex
%%% TeX-master: "Ficha-2011_bsm"
%%% End: 

