\section{ Planteamiento del Problema }

\subsection{Marco teórico}

Con la entrada en operación del LHC, se espera descubrir señales de materia oscura y de modelos que tengan algún mecanismo radiactivo para generar masas de neutrinos.

Los futuros experimentos de rayos cósmicos, tales como el experimento AMS-02, Planck, IceCube, etc, esperan mejorar la sensibilidad a este tipo de señales. Así mismo, los experimentos de detección directa de materia oscura como DAMA, CDMS, CoGeNT, XENON han creado grandes expectativas por los resultados presentes y los que se espera obtener en los próximos meses. Los experimentos de física de neutrinos y de radiación cósmica de fondo han entrado ya en una era de medidas de precisión.
Lo que proponemos en este proyecto es tratar de responder las siguiente pregunta: ¿Cuáles serán las restricciones impuestas por los recientes y futuros datos de todos estos experimentos sobre  modelos que presenten una partícula candidata a materia oscura o que generen masas para los neutrinos?
Los recientes datos reportados por las colaboraciones PAMELA y Fermi han impuesto restricciones muy fuertes sobre el flujo de rayos cósmicos provenientes de partículas de materia oscura. Estas restricciones afectarán directamente el espacio de parámetros permitido por los experimentos actuales. Adicional a estas nuevas restricciones, también se debe tener en cuenta la sensibilidad de los futuros experimentos de rayos cósmicos, tales como el experimento AMS-02, Planck, IceCube, etc.

Por otro lado, con la entrada en operación del LHC junto con el tevatrón, se esperan descubrir señales del bosón de Higgs, materia oscura, dimensiones extra, supersimetría, etc. En algunos modelos más allá del Modelo Estándar se tienen predicciones concretas sobre los posibles productos de aniquilación y desintegración del candidato a materia oscura que podrán ser detectados.

Actualmente se encuentra funcionado el acelerador Tevatron en Fermilab en Estados Unidos. Este acelera protones y antiprotones a una energía de 1 TeV, y se espera que siga funcionado por 2 años más. Ya ha entrado en funcionamiento el acelerador LHC en el CERN en la frontera entre Francia y Suiza. Aunque de momento se encuentra en estado de pruebas funcionando a energias que alcanzarán los 7TeV, en un futuro cercano este  acelerará protones en ambos sentidos a una energía que alcanzará hasta los 14 TeV. La mayoría de los experimentos conducidos son hechos en grandes instrumentos llamados detectores. Por ejemplo el LHC tiene 4 detectores. En dos de ellos, ATLAS y CMS, hay grupos colombianos participando. En los últimos años el Grupo ha tenido una estrecha vinculación con el Grupo de Marta Losada que esta participando en la colaboración ATLAS.

A nivel de detección de radiación cósmica tenemos el satélite PAMELA para detección de rayos cósmicos de electrones, positrones, antiprotones, y núcleos livianos. En agosto del año pasado PAMELA reporto una abundancia anómala de positrones %\cite{pamela},
que ha dado lugar a un sinnúmero de publicaciones tratando de explicar su origen.  Fermi-LAT es otro satélite para la detección de rayos cósmicos que también estará funcionando  durante los próximos años. En sus primeras medidas %\cite{fermi}
también ha reportado una abundancia anómala de positrones y electrones. El detector de rayos cósmicos AMS-02~%\cite{ams:2009},
que será instalado en la Estación Espacial  internacional, medirá el espectro de antiprotones y positrones en un rango de energía mucho más amplio y con una estadística mucho mejor. En cuanto a detección directa de materia oscura existen grandes expectivas por los resultados presentes y los que se espera obtener en los próximos meses %... \url{http://resonaances.blogspot.com/2010/03/another-experiment-sees-dark-matter.html}

En los próximos años entrará el satélite Planck que fue lanzado en mayo, y desde julio  se encuentra en pleno funcionamiento. Con los datos obtenidos de este experimento se espera medir la radiación cósmica de fondo con mucha más precisión que su antecesor WMAP. Además esta el experimento Auger en Argentina que detecta rayos cósmicos de ultra alta energía que alcanzan a llegar a la superficie terrestre.

También hay varios experimentos de física de neutrinos en marcha que esperan mejorar los datos actuales de oscilaciones de neutrinos, determinar el ángulo de mezcla $\theta_{13}$, del cual actualmente solo hay cotas, y si este resulta ser diferente de cero, buscar evidencia de violación de CP en el sector de neutrinos, un ingrediente crucial para generar bariogénesis a través de leptogénesis.

Esto representa un escenario muy dinámico donde el Grupo debe estar preparado para interpretar cualquier evidencia que resulte del Higgs o nueva física a la luz de los modelos que viene desarrollando.
	
La pregunta que queremos abordar en el presento proyecto es si:
¿se podrá establecer en los aceleradores futuros cual es el modelo que
explica la generación de masas y mezclas para los neutrinos?

%%Jose david
Por medio del estudio del modelo de seesaw radiactivo se predicen algunos canales de decaimiento para la materia oscura que podrán ser observados en el LHC. Si se llegan a detectar dichos canales sabremos en que modelo la generación de masas de los neutrinos se da, y a través de que mecanismo propuesto.

Aunque se ha encontrado que los neutrinos tienen masas a la escala de
sub-eV y se mezclan %\cite{hep-ph/0405172},
con experimentos sólo sobre
física de neutrinos no se puede establece cual es el modelo que contiene
el mecanismo apropiado de generación de masas y mezcla de neutrinos.
En particular, será muy difícil establecer la naturaleza de los
neutrinos, es decir si tienen masa de Dirac o Majorana.

Los aceleradores de partículas tienen la posibilidad de
establecer cual es el mecanismo de generación de masas y mezclas de
neutrinos y de forma indirecta establecer entonces la naturaleza de
los mismos. El mecanismo seesaw %\cite{Gell-Mann:vs}
ha sido la
explicación más popular para la pequeñez de las masas de neutrinos. Es
elegante y simple, y depende sólo de análisis dimensional para la
nueva física que se requiere. Los datos actuales de neutrinos
apuntan a una escala de seesaw del orden $10^{10}$ - $10^{15}$ GeV,
donde la violación de número leptónico ocurre a través de masas de
Majorana de los neutrinos derechos. Con una escala tan alta, los
efectos de violación de sabor leptónico en procesos diferentes a la
oscilación de neutrinos son extremadamente pequeños. Por ejemplo el
decaimiento del muon a electrón fotón en extensiones seesaw del Modelo
Estándar, es decenas de ordenes de magnitud menor a la cota actual.

Una alternativa al mecanismo seesaw que también explica naturalmente
la pequeñez de la masa de neutrinos son los mecanismos de generación
radiativa, ver por ejemplo los siguientes artículos recientes y
sus referencias
%\cite{hep-ph/0503059,AristizabalSierra:2006ri,Babu:2002uu,Dong:2006vk}.
En esta aproximación las masas de neutrinos son cero a nivel árbol y
son inducidas únicamente como correcciones radiativas finitas.  Estas
correcciones radiativas son proporcionales a la raíz cuadrada de
las masas del leptón cargado (o quark) divididas por la escala de nueva
física $\Lambda$. Las masas de los neutrinos pueden estar en el rango
de sub-eV aún cuando la escala de nueva física sea del orden de
TeV. En este caso, los procesos de violación de número leptónico en
otros procesos diferentes a la física de neutrinos pueden llegar
a ser accesibles a los aceleradores.

En este proyecto proponemos analizar en detalle las consecuencias
experimentales de modelos específicos donde las masas de
neutrinos se generan radiativamente. En particular estamos interesados
en calcular los decaimientos de las partículas del modelo que
están controlados por los mismos parámetros que inducen las masas y
las mezclas de los neutrinos. Estos decaimientos podrán no sólo
ayudar a determinar el mecanismo de generación de masas de neutrinos
con aceleradores de partículas, sino también ayudar a determinar
experimentalmente los parámetros del modelo que aparecen en la matriz
de masa de neutrinos.

Los recientes datos reportados por las colaboraciones PAMELA y Fermi-LAT
han señalado un exceso en el flujo de positrones y en el flujo total de
electrones y positrones respectivamente, mientras para el flujo de antiprotones no se
ha observado exceso alguno. El exceso en el flujo de antimateria puede ser
interpretado como evidencia de nuevas fuentes astrofísicas de rayos cósmicos de
alta energía en nuestra galaxia, tales como pulsares o remanentes de supernovas.
En este escenario, el flujo de positrones y rayos gamma astrofísicos constituirán
un nuevo background para las posibles señales de detección de materia oscura o para las posibles zonas
de exclusión de los diferentes modelos que predicen la existencia de materia oscura. En particular,
los background de los flujos de positrones, antiprotones y rayos gamma afectarán directamente
el espacio de parámetros permitido por los experimentos hasta ahora de los modelos
supesimétricos con violación de paridad R extendidos con una simetría horizontal
$U(1)_H$ donde el neutralino es candidato a materia oscura inestable. Adicional a
estas nuevas restricciones, también se debe tener en cuenta la sensibilidad de
los futuros experimentos de rayos cósmicos, tales como el experimento AMS-02.

Por otro lado, con la entrada en operación del LHC junto con el tevatrón,
se esperan descubrir señales del bosón de Higgs, materia oscura, dimensiones extra,
supersimetría, entre muchas otras señales.  En modelos supersimétricos con violación de paridad R
extendidos con una simetría horizontal $U(1)_H$ donde el neutralino decae rápidamente vía operadores
que violan número bariónico, se tendrán predicciones concretas sobre los posibles productos de
desintegración del neutralino que podrán ser detectados en el LHC o tevatrón.

Con base en lo planteado anteriormente, lo que proponemos en este proyecto es
tratar de responder la siguiente pregunta: ¿Cuáles serán las restricciones
impuestas por los recientes y futuros datos astrofísicos y de aceleradores sobre
los modelos supersimétricos con violación de paridad R extendidos con una simetría $U(1)_H$ que presentan un neutralino inestable?

\subsection{Planteamiento del problema}


En el caso de modelos supersimétricos basados una simetría de sabor anómala $U(1)_H$, con un solo flavón de carga $-1$, una vez se introducen las condiciones teóricas y fenomenológicas, la solución más óptima se puede expresar en términos de 4 cargas $H$ libres que se puede usar para explicar el valor de los 45 parámetros del superpotencia que violan $R$--paridad \cite{Mira:2000gg,Dreiner:2003hw,Dreiner:2003yr,Dreiner:2007vp,Dreiner:2006xw,Sierra:2009zq}.
\begin{equation}
  \label{eq:5}
  W_{\cancel{R_p}} = \mu_i\widehat{L}_i\widehat{H}_u +
  \lambda_{i j k}\widehat{L}_i\widehat{L}_j\widehat{l}_k +
  \lambda'_{i j k}\widehat{L}_i\widehat{Q}_j\widehat{d}_k +
  \lambda''_{ijk}\widehat{u}_i\widehat{d}_j\widehat{d}_k\,
\end{equation}
De las diferentes posibilidades se pueden construir modelos con
\begin{enumerate}
\item solamente términos bilineales que violan $R$--paridad, con acoplamientos $\mu_i$ que explican las masas y mezclas de los neutrinos \cite{Mira:2000gg,Dreiner:2003hw,Dreiner:2006xw}
\label{item:1}
\item modelos donde la simetría discreta de conservación de $R$--paridad, o una simetría equivalente, queda como remanente de la ruptura espontánea de la simetría $U(1)_H$ \cite{Dreiner:2003hw,Dreiner:2003yr,Dreiner:2007vp}. Las masas y mezclas de neutrinos se pueden explicar con la introducción de neutrinos derechos de carga $H$ semientera.
\label{item:2}
\item Modelos con violación de número leptónico a través de términos trilineales con ruptura de $R$--paridad. Se pueden construir modelos con hasta dos términos del tipo $\lambda_{ijk}$, con los tres índices diferentes \cite{Sierra:2009zq}.  En este caso las cargas $H$ se pueden escoger de manera que los acoplamientos trilineales que violan $R$--paridad queden muy suprimidos, del orden de $10^{-23}$, de modo que los neutralinos pueden ser  candidatos a materia oscura inestables con un tiempo de vida media del orden de $10^{26}\,$~sec. Dichos modelos se pueden usar para explicar las anomalías en rayos cósmicos \cite{Sierra:2009zq} recientemente detectadas por experimentos como PAMELA \cite{Adriani:2008zr}, ATIC \cite{:2008zzr} y Fermi~LAT \cite{Abdo:2009zk}. Estos datos sugieren que el decaimiento del neutralino sea principalmente leptónico, lo cual hace especialmente interesante que la única posibilidad consistente con simetrías horizontales de lugar precisamente a decaimientos leptónicos.
\label{item:3}
\item Modelos con violación de número bariónico a través de términos trilineales que violan $R$--paridad del tipo $\lambda''_{ijk}$. La fenomenología de este tipo de modelos aún no ha sido estudiada en la literatura
\label{item:4}
\end{enumerate}
En este proyecto pretendemos estudiar de manera sistemática la fenomenología de los modelos \ref{item:3}., \ref{item:4}. a luz de los resultados experimentales presentes y futuros.

En el caso \ref{item:3}., en lugar de asumir que los términos trilineales pueden explicar las anomalías de rayos cósmicos, se puede asumir que estas anomalías tienen un origen astrofísico, e intentar usar los datos presentes para restringir los posibles tiempo de vida media y masa del neutralino. En esta línea, también se planea estudiar las predicciones de modelos específicos de materia oscura inestable para los experimentos de rayos cósmicos en marcha, y los que comenzarán a funcionar en un futuro cercano.  En particular queremos  investigar el flujo de los rayos gamma que generan en modelos donde la materia oscura decae leptónica  o hadrónicamente (caso \ref{item:4}) y determinar las regiones del espacio de parámetros que puedan ser confrontadas con las medidas de rayos gamma que se esperan de Fermi--LAT en los próximos meses.

En el caso \ref{item:4}, en este proyecto pretendemos investigar posibles predicciones de señales para el LHC en modelos donde la simetría horizontal se use para formular modelos consistentes de violación de número bariónico a través de términos trilineales de ruputura $R$--paridad de tipo $\lambda''$. Después de tener en cuenta las restricciones experimentales sobre acoplamientos individuales
%, de las cuales la más importante es \cite{Barbier:2004ez}
%\begin{align}
%   \lambda_{11k}''<  10^{-9}   \left( \frac{m_{\tilde g}}{100\,\text{GeV}}  \right)^{1/2}
%\left( \frac{m_{\tilde q_k}}{100\,\text{GeV}}  \right)^2
%\end{align}
esperamos tener una predicción muy concreta para la jerarquía en los acoplamientos $\lambda''$, la cual podría dar lugar a señales muy específicas en el LHC. De este modo, en caso de que se descubriera en el LHC un modelo de ruptura de $R$--paridada través términos $\lambda''$, se podría determinar si la simetría Abeliana  $U(1)_H$ con un sólo flavón es la simetría que explica la jerarquía en la masas de los fermiones.

En el modelo usual para violación de número bariónico a través de términos $\lambda''$, donde se impone  una simetría discreta del tipo paridad leptónica, los términos no renormalizables asociados a la masa de neutrino quedan prohibidos, por lo que el mecanismo seesaw no puede ser implementado. En el caso de que la simetría horizontal se use para formular un modelo de este tipo, se puede introducir un nuevo flavón con carga $H$ fraccionaria [del tipo usado en \cite{Chen:2008tc}] que puede dar lugar a un mecanismo de seesaw apropiado para explicar los datos de oscilaciones de neutrinos. En este proyecto pretendemos explorar esta posibilidad.


%%% Local Variables: 
%%% mode: latex
%%% TeX-master: Ficha-2011_bsm
%%% End: 

