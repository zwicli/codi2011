\section{Planteamiento del Problema }
\begin{instrucciones}
  CODI: ¿Está bien definido el problema que se quiere investigar? ¿Es clara su justificación desde el punto de vista académico, científico, tecnológico, social, económico y legal? (15).

  COLCI: en este ítem usted deberá describir de forma precisa y completa la
  naturaleza y magnitud del problema de investigación que se quiere
  abordar. Formule claramente las preguntas concretas a las cuales se
  quiere responder en el contexto del problema planteado.
\end{instrucciones}
%Tesis
El modelo estándar de las interacciones fundamentales ha sido muy
exitoso durante las últimas décadas para explicar la mayoría de los
resultados experimentales de física de altas energías. Sin embargo,
simultáneamente se ha venido acumulando evidencia experimental que
requiere extender el modelo estándar con nuevas partículas. En la
actualidad hay un gran esfuerzo experimental para explorar en detalle
todas esas posibilidades. Estamos ahora en una época de efervescencia
experimental con detectores de partículas instalados desde las alturas
de satelites artificiales, hasta las profundidades de laboratorios
subterráneos a kilometros de profundidad. 

El Large Hadron Collider (LHC), en un tunel a 100 metros de profundidad y
con una circunferencia de 27 Km, es el primer acelerador
protón--protón que puede probar la escala del TeV.  Posee cuatro
detectores, dos de los cuales están especialmente diseñados para
encontrar señales de nueva física. El LHC finalmente ha comenzado a
tomar datos en el 2010, aunque hasta el 2012 lo hará a la mitad de la
energía para la cual fue diseñado.

Desde el 2008 varios experimentos sobre rayos cósmicos como ATIC
\cite{:2008zzr} y los satélites PAMELA \cite{Adriani:2008zr} y
Fermi--LAT \cite{Abdo:2009zk}, han venido reportando un exceso en el
flujo de electrones y positrones en rayos cósmicos. Estos resultados
han dado lugar a un sinnúmero de publicaciones tratando de explicar su
origen. Ya se han comenzado a reportar las primeras medidas de rayos
gamma por parte de Fermi--LAT, las cuales pueden ayudar a discernir si
el origen de las anomalías detectadas en electrones y positrones es
debida a fuentes astrofísicas como pulsares cercanos, o a la
aniquilación o el decaimiento de materia oscura. El detector de rayos
cósmicos AMS-02~\cite{ams:2009}, que será instalado en la estación
espacial internacional, medirá el espectro de antiprotones y
positrones en un rango de energía mucho más amplio y con una
estadística mucho mejor.  

El satélite Planck fue lanzado en mayo, y
desde julio se encuentra en pleno funcionamiento.  Con los datos
obtenidos de este experimento se espera medir la radiación cósmica de
fondo con mucha más precisión que su antecesor WMAP. 

%hablar sobre XENON basado en la discusión de 1104.3572
Los experimentos de detección directa de materia oscura instalados en
laboratorios subterráneos como XENON100 \cite{1104.3121}, han comenzado a explorar las
regiones predichas por algunos de los modelos más estudiados de
materia oscura. Para los próximos años se espera cubrir todo el espacio de parámetros para materia oscura del modelo estándar supersimétrico mínimo restringido (Constrained MSSM de sus siglas en Inglés)

Finalmente hay varios experimentos de física de neutrinos en marcha
que esperan mejorar los datos actuales de oscilaciones de neutrinos.


En éste proyecto se explorarán diferentes extensiones del modelo
estándar que explican estos datos y se obtendrán predicciones
concretas en los experimentos presentes y futuros mencionados
anteriormente.

%evidencia 1
La principal evidencia de necesidad de física más allá del modelo
estándar se halla en las medidas de las oscilaciones de neutrinos,
que a través de muchos experimentos diferentes han logrado establecer
que los neutrinos tienen masa y se mezclan entre sí. Si la masa de los
neutrinos es total o parcialmente explicada por métodos radiativos, no
sólo es posible dar cuenta de la pequeñez de sus masas con respecto a
la de los otros fermiones, sino también, hacer predicciones muy
concretas en aceleradores de partículas, como el LHC. Aún en el caso
de modelos tan estudiados de este tipo, como es el caso del modelo
supersimétrico con violación bilineal de paridad R, quedan muchos
aspectos que explorar.

%evidencia 2
En los últimos tiempos, sobre todos después de las medidas de
precisión de WMAP, el satelite de la NASA que ha medido con mucha
precisión la radiación cósmica de fondo, se ha venido estableciendo la
necesidad de una forma de materia compuesta por partículas débilmente
interactuantes (WIMPs de sus siglas en inglés) que se conoce como
materia oscura. Esta forma el 22\% de la materia del Universo. Sin
embargo, aún no se ha encontrado evidencia directa de la existencia de
dicha partícula. Hay mucha expectativa de lo que pueda encontrar el
LHC o los experimentos de detección directa a corto y mediano plazo.

%evidencia 3
La materia oscura puede ser estable o con un tiempo de vida mucho
mayor que la edad del Universo. En el caso de materia oscura inestable
se esperan señales fuertes en los detectores de rayos cósmicos
instalados en satélites artificiales que orbitan la tierra como
Fermi-LAT y PAMELA, y en los detectores futuros como
AMS-02~\cite{ams:2009}. De hecho, recientemente se ha encontrado un
exceso en positrones \cite{Adriani:2008zr} que se podría explicar con
una partícula de materia oscura que decae principalmente a leptones
con una vida media de unos $10^{26}$ seg.

%evidencia 4
Las observaciones astronómicas sugieren que el Universo está compuesto
en su mayor parte de materia. En el contexto del big-bang, esto
implica que en algún momento grandes cantidades de materia y
antimateria se aniquilaron dejando el pequeño exceso de materia que
constituye el Universo observable actual. El problema de explicar el
exceso inicial de materia sobre antimateria se conoce con el nombre de
bariogénesis. Con el modelo estándar no es posible explicar
bariogénesis.

%conclusion
Un modelo ideal sería uno que de cuenta de las masas y mezclas de
neutrinos, tenga un candidato a materia oscura que sirva para explicar
el exceso de positrones en experimentos de rayos cósmicos y a la vez
contenga los ingredientes para explicar bariogenesis. En este proyecto
pretendemos formular un modelo de éstas características, además
queremos continuar explorando otros posibles modelos que puedan dar
cuenta de alguna de las observaciones que requieren una extensión del
modelo estándar.
 

\subsection{Marco teórico}

\begin{instrucciones}
 CODI: ¿Es la teoría actualizada y acertada con respecto al problema que se va a estudiar? ¿Su formulación es coherente? ¿Es clara la perspectiva teórica desde donde se ubica el problema?
\end{instrucciones}

%comentar sobre una simetría Z_2
La partícula de materia oscura puede ser estable, o inestable con un
tiempo de vida mucho más grande que la edad del Universo. En el primer
caso usualmente se impone una simetría adicional $Z_2$ bajo la cual
las partículas del modelo estándar son pares, mientras que las
partículas nuevas son impares. Este tipo de simetría garantiza que la
partícula impar más liviana es estable y, de ser neutra, constituye un
buen candidato de materia oscura. En el programa computacional
microMEGAS~\cite{Belanger:2006is} se puede implementar cualquier
extensión del modelo estándar que posea una simetría $Z_2$ para
calcular numéricamente la densidad de reliquia de materia oscura y la
sección eficaz WIMP--nucleón, relevante para los experimentos de
búsquedas directas. Cuando la partícula de materia oscura es inestable
se requiere de alguna simetría para garantizar que su acoplamiento a
las partículas del modelo estándar esté suficientemente suprimido. 

En el caso de supersimetría por ejemplo, todas las interacciones del
modelo están especificadas por la simetría gauge y el superpotencial,
que debe ser una función holomórfica de los campos escalares del
modelo. En el modelo estándar supersimétrico, además de los términos
del superpotencial que constituyen la semilla del potencial escalar y
del Lagrangiano de Yukawa, la invarianza gauge permite los siguiente
términos
\begin{equation}
  \label{eq:5}
  W_{\cancel{R_p}} = \mu_i\widehat{L}_i\widehat{H}_u + 
  \lambda_{i j k}\widehat{L}_i\widehat{L}_j\widehat{l}_k +
  \lambda'_{i j k}\widehat{L}_i\widehat{Q}_j\widehat{d}_k + 
  \lambda''_{ijk}\widehat{u}_i\widehat{d}_j\widehat{d}_k\,
\end{equation}
La simetría $Z_2$, conocida en éste caso como paridad R, $R_p$,
prohíbe todos estos términos, y cuando la partícula impar bajo paridad
R es el neutralino o el gravitino, ésta puede ser una candidato de
materia oscura. Aunque se rompa paridad R y alguno de estos términos
queden permitidos, el gravitino puede seguir siendo un candidato de
materia oscura pues sus decaimientos están suprimidos por la escala de
Planck. Para que el neutralino pueda ser materia oscura inestable
cuando se rompe paridad R, se requiere que los acoplamientos sean
extremadamente pequeños: del orden de $10^{-24}$ para los
acoplamientos trilineales ${\lambda^{(\ }}'\;''{}^)$.


\subsection{Justificación}

En el Grupo nos hemos enfocado en extensiones del modelo estándar que
dan cuenta de las masas y mezclas de neutrinos. Hemos estudiado
exhaustivamente las predicciones del modelo bilineal con ruptura de
paridad R que incluye sólo los tres términos con $\mu_i$ del
superpotencial en \eqref{eq:5}, tanto para el Tevatron como para el
LHC asumiendo que el neutralino es la partícula supersimétrica más
liviana (LSP de sus siglas en inglés)
\cite{Magro:2003zb,deCampos:2005ri,deCampos:2008ic,deCampos:2008re,DeCampos:2010yu,deCampos:2007bn}. En
el último trabajo al respecto hemos determinado el nivel de precisión
con el que se puede llegar a medir en el LHC la correlación entre
decaimientos de neutralinos a muón y tau con el ángulo de mezcla
atmosférico de neutrinos: una predicción muy concreta que de no observarse en el LHC en los próximos años descartaría completamente el modelo como el mecanismo de generación de masas para neutrinos. 
\begin{proyecto}
  En este proyecto pretendemos seguir explorando más correlaciones de
  éste tipo.
\end{proyecto}

Cuando el gravitino es la LSP además constituye un candidato viable de
materia oscura inestable. Éste caso ha sido estudiado exhaustivamente
en la literatura por sus implicaciones en experimentos de rayos
cósmicos, especialmente en el caso en que las masas de neutrinos son
explicadas a través del mecanismo de seesaw y el modelo también
explica bariogénisis a través de leptogénesis. De hecho, como hemos
mostramos en \cite{Choi:2010jt}, y ratificado por otro grupo con datos
más recientes en \cite{Garny:2010eg}, los últimos datos de líneas de
rayos gamma publicado por FERMI-LAT excluyen la posibilidad de que las
masas de neutrinos puedan ser generadas por términos bilineales de
ruptura de paridad R si la temperatura de recalentamiento es sobre
$10^9\ $GeV como sugiere leptogénesis. En esta región del espacio de
parámetros los acoplamientos bilineales son tan pequeños que el
decaimiento del partícula siguiente a la LSP, la NLSP, ocurre fuera
del detector en el LHC, por lo que el modelo es básicamente
indistinguible del MSSM. En éste caso los datos de detección indirecta
de materia oscura a través de rayos cósmicos serían la única forma de
diferenciar el modelo del de supersimetría usual con conservación de
paridad R y el neutralino como materia oscura, del modelo con ruptura
bilineal de paridad R con el gravitino como materia oscura.  
\begin{proyecto}
  En esté proyecto queremos explorar las implicaciones que los datos
  de líneas de rayos gammas tiene sobre el modelo de ruptura bilineal
  de paridad R. Permitiendo temperaturas de recalentamiento de hasta 1
  TeV, y relajando la hipótesis de universalidad en las masas de los
  gauginos, usualmente hecha en estos análisis, queremos establecer
  las regiones del espacio de parámetros donde es posible explicar los
  datos de oscilaciones de neutrinos con los parámetros bilineales de
  ruptura de paridad R. En éste caso las señales generadas por la
  partícula próxima al LSP en el LHC podrían predecir señales muy
  específicas en los experimentos de rayos cósmicos.
\end{proyecto}


La metodología desarrollada en el estudio exhaustivo del modelo con
ruptura bilineal de paridad R la hemos logrado aplicar a otros modelos
de generación radiativa de masas de neutrinos con nuevas partículas a
la escala del TeV asequibles en el LHC
\cite{Sierra:2008wj,AristizabalSierra:2006ri}. Recientemente hemos
comenzado a explorar modelos que puedan dar cuenta simultáneamente de
masas de neutrinos y candidatos de materia oscura
\cite{Hirsch:2005ag,Choi:2010jt,Sierra:2008wj}. En el caso del seesaw
radiativo por ejemplo, se implementa una simetría $Z_2$ de manera que
un escalar o a un neutrino derecho puedan ser buenos candidatos de
materia oscura. 
\begin{proyecto}
  En este proyecto queremos explorar más en detalle las implicaciones
  fenomenológicas de éstos modelos.
\end{proyecto}

Nuestro grupo \cite{Nardi:2008ix} fue uno de los primeros en proponer
una explicación en términos de materia oscura inestable para explicar
el exceso de positrones observado por el satelite PAMELA en el
2008\cite{Adriani:2008zr}. Luego hemos construido un modelo basado en
supersimetría con ruptura de paridad R a través de términos
trilineales del tipo $\lambda$ en la ec.~\eqref{eq:5}, para explicar
la preferencia por decaimientos leptónicos de la partícula de materia
oscura, que en este caso es el neutralino. Para ello se ha
implementado una simetría horizontal que permite sólo la presencia de
acoplamientos trilineales leptónicos \cite{Sierra:2009zq}. 
\begin{proyecto}
  En este proyecto queremos explorar la posibilidad de explicar
  también las masas de neutrinos, y el gravitino como materia oscura
  dentro de este modelo.
\end{proyecto}
 En un proyecto en marcha estamos explorando la posibilidad
opuesta de un candidato de materia oscura que decae sólo
hadrónicamente. La simetría horizontal garantiza que sólo los
acoplamientos trilineales de quarks derechos estén
permitidos. Finalmente hemos extendido el modelo para incluir masas de
neutrinos. Ésta experiencia nos servirá para hacer las modificaciones
necesarias al modelo con acoplamientos trilineales leptónicos del tipo
$\lambda$ para dar cuenta de masas y mezclas para los neutrinos. 

\begin{proyecto}
  En términos generales, en este proyecto queremos explorar modelos
  motivados por falencias fenomenológicas del modelo estándar,
  especialmente aquellos modelos que no sólo tienen implicaciones en
  el LHC, sino también en experimentos de detección de rayos cósmicos,
  y experimentos de detección directa de materia oscura.
\end{proyecto}


%resumiendo tenemos
%1. Correlaciones bilineal en el LHC
%2  Restricciones en masa de neutrinos con gravitino LSP en el modelo bilineal.
%3. Materia oscura con el seesaw radiativo.
%4. modelo con lambda: gravitino y neutrinos derechos.


\begin{proyecto}
  Con base en lo planteado anteriormente, lo que proponemos en este
  proyecto es tratar de responder la siguiente pregunta: ¿Cuáles
  serían las restricciones impuestas por los recientes y futuros datos
  de todos estos experimentos sobre modelos que presenten una
  partícula candidata a materia oscura o que generen masas para los
  neutrinos?
\end{proyecto}







 



Con la entrada en operación del LHC, se espera descubrir señales de materia oscura y de modelos que tengan algún mecanismo radiativo para generar masas de neutrinos.

Por otro lado, los recientes datos reportados por las colaboraciones PAMELA y Fermi han impuesto restricciones muy fuertes sobre el flujo de rayos cósmicos provenientes de partículas de materia oscura. Los futuros experimentos de rayos cósmicos, tales como el experimento AMS-02, Planck, IceCube, etc, esperan mejorar la sensibilidad a este tipo de señales. Así mismo, los experimentos de detección directa de materia oscura como DAMA, CDMS, CoGeNT, XENON han creado grandes expectativas por los resultados presentes y los que se espera obtener en los próximos meses. Los experimentos de física de neutrinos y de radiación cósmica de fondo han entrado ya en una era de medidas de precisión.

Con base en lo planteado anteriormente, lo que proponemos en este proyecto es tratar de responder las siguiente pregunta: ¿Cuáles serían las restricciones impuestas por los recientes y futuros datos de todos estos experimentos sobre  modelos que presenten una partícula candidata a materia oscura o que generen masas para los neutrinos?


\subsection{Marco teórico}


\begin{ideas}
Se ha calculado el flujo de rayos gamas y confrontarlo con lo observado en experimentos de rayos cósmicos en modelos supersimétricos donde el gravitino es el candidato de materia oscura inestable (CMOI) en la región de masa de hasta 80 GeV. Estudios recientes  muestran que en está región son importantes canales de decaimiento que no se habían considerado previamente [4]. 



  El problema que queremos abordar trata acerca de la fenomenología más alla del modelo estándar.
%Con la entrada en operación del LHC, se espera descubrir señales de materia oscura y de modelos que tengan algún mecanismo radiactivo para generar masas de neutrinos.
%Por otro lado, los recientes datos reportados por las colaboraciones PAMELA y Fermi han impuesto restricciones muy fuertes sobre el flujo de rayos cósmicos provenientes de partículas de materia oscura. Los futuros experimentos de rayos cósmicos, tales como el experimento AMS-02, Planck, IceCube, etc, esperan mejorar la sensibilidad a este tipo de señales. Así mismo, los experimentos de detección directa de materia oscura como DAMA, CDMS, CoGeNT, XENON han creado grandes expectativas por los resultados presentes y los que se esperan obtener en los próximos meses. Los experimentos de física de neutrinos y de radiación cósmica de fondo han entrado ya en una era de medidas de precisión.
Ya que con la entrada en operación del LHC más el tevatrón, se esperan descubrir señales del bosón de Higgs, de la materia oscura, dimensiones extra, supersimetría, etc. En algunos modelos más allá del Modelo Estándar se tienen predicciones concretas sobre los posibles productos de aniquilación y desintegración del candidato a materia oscura que podrían ser detectados en estos aceleradores.La mayoría de los experimentos conducidos son hechos en grandes instrumentos llamados detectores. Por ejemplo el LHC tiene 4 detectores. En dos de ellos ATLAS y CMS hay grupos colombianos participando. En los últimos años el Grupo ha tenido una estrecha vinculación con el Grupo de Marta Losada que esta participando en la colaboración ATLAS.

 Además se tienen otros experimentos como son las colaboraciones PAMELA y Fermi que han impuesto restricciones muy fuertes sobre el flujo de rayos cósmicos provenientes de partículas de materia oscura. Los futuros experimentos de rayos cósmicos, tales como el experimento AMS-02, Planck, IceCube, etc, esperan mejorar la sensibilidad a este tipo de señales. Así mismo, los experimentos de detección directa de materia oscura como DAMA, CDMS, CoGeNT, XENON han creado grandes expectativas por los resultados presentes y los que se esperan obtener en los próximos meses.

%Actualmente se encuentra funcionando el acelerador Tevatron en Fermilab en los Estados Unidos. Este acelera protones y antiprotones a una energía de 1 TeV, y se espera que siga funcionado por 2 años más. Ya ha entrado en funcionamiento el acelerador LHC en el CERN en la frontera entre Francia y Suiza. Aunque de momento se encuentra en estado de pruebas funcionando a energías que alcanzarán los 7TeV, en un futuro cercano éste acelerará protones en ambos sentidos a una energía que alcanzará hasta los 14 TeV. La mayoría de los experimentos conducidos son hechos en grandes instrumentos llamados detectores. Por ejemplo el LHC tiene 4 detectores. En dos de ellos ATLAS y CMS hay grupos colombianos participando. En los últimos años el Grupo ha tenido una estrecha vinculación con el Grupo de Marta Losada que esta participando en la colaboración ATLAS.

%A nivel de detección de radiación cósmica tenemos el satélite PAMELA para detección de rayos cósmicos de electrones, positrones, antiprotones, y núcleos livianos. En agosto del año pasado PAMELA reporto una abundancia anómala de positrones %\cite{pamela}, 
%que ha dado lugar a un sinnúmero de publicaciones tratando de explicar su origen.  Fermi-LAT es otro satélite para la detección de rayos cósmicos que también estará funcionando durante los próximos años. En sus primeras medidas %\cite{fermi} 
%también ha reportado una abundancia anómala de positrones y electrones. El detector de rayos cósmicos AMS-02~%\cite{ams:2009},  
%que será instalado en la Estación Espacial internacional, medirá el espectro de antiprotones y positrones en un rango de energía mucho más amplio y con una estadística mucho mejor. Análogamente está el satélite Planck que fue lanzado en mayo, y desde julio se encuentra en pleno funcionamiento. Con los datos obtenidos de este experimento se espera medir la radiación cósmica de fondo con mucha más precisión que su antecesor WMAP. Además esta el experimento Auger en Argentina que detecta rayos cósmicos de ultra alta energía que alcanzan a llegar a la superficie terrestre.

Igualmente hay varios experimentos de física de neutrinos en marcha que esperan mejorar los datos actuales de oscilaciones de neutrinos, determinar el ángulo de mezcla $\theta_{13}$, del cual actualmente solo hay cotas, y si este resulta ser diferente de cero, buscar evidencia de violación de CP en el sector de neutrinos, un ingrediente crucial para generar bariogénesis a través de leptogénesis.

 Aunque se ha encontrado que los neutrinos tienen masas a la escala de
sub-eV, no se ha podido establecer cual es el modelo que contiene
el mecanismo apropiado de generación de masas y mezcla de neutrinos.
En particular, será muy difícil establecer la naturaleza de los
neutrinos, es decir si tienen masa de Dirac o Majorana. El mecanismo seesaw %\cite{Gell-Mann:vs} 
ha sido la
explicación más popular para la pequeñez de las masas de neutrinos. Es
elegante y simple, y depende sólo de análisis dimensional para la
nueva física que se requiere. Los datos actuales de neutrinos
apuntan a una escala de seesaw del orden $10^{10}$ - $10^{15}$ GeV,
donde la violación de número leptónico ocurre a través de masas de
Majorana de los neutrinos derechos. Con una escala tan alta, los
efectos de violación de sabor leptónico en procesos diferentes a la
oscilación de neutrinos son extremadamente pequeños. Una alternativa al mecanismo seesaw que también explica naturalmente
la pequeñez de la masa de neutrinos son los mecanismos de generación
radiativa.
%\cite{Hirsch:2005ag,AristizabalSierra:2006ri,Babu:2002uu,Dong:2006vk}.
%En este aproximación las masas de neutrinos son cero a nivel árbol y
%son inducidas únicamente como correcciones radiativas finitas.
% Estas correcciones radiativas son proporcionales a la raíz cuadrada de
%las masas del leptón cargado (o quark) dividas por la escala de nueva
%física $\Lambda$. Las masas de los neutrinos pueden estar en el rango
%de sub-eV aún cuando la escala de nueva física sea del orden de
%TeV. En este caso, los procesos de violación de número leptónico en
%otros procesos diferentes a la física de neutrinos pueden llegar
%a ser accesibles a los aceleradores.
En este proyecto proponemos analizar en detalle las consecuencias
experimentales de modelos específicos donde las masas de
neutrinos se generan radiativamente.
% En particular estamos interesados
%en calcular los decaimientos de las partículas del modelo que
%estén controlados por los mismos parámetros que inducen las masas y
%las mezclas de los neutrinos. Estos decaimientos podrían no sólo
%ayudar a determinar el mecanismo de generación de masas de neutrinos
%con aceleradores de partículas, sino también ayudar a determinar
%experimentalmente los parámetros del modelo que aparecen en la matriz
%de masa de neutrinos.

Con lo expuesto anteriormente, las preguntas que queremos abordar en el presente proyecto son: 
¿se podrá establecer en los aceleradores futuros cual es el modelo que
explica la generación de masas y mezclas para los neutrinos? ¿Cuáles serían las restricciones 
impuestas por los recientes y futuros datos astrofísicos y de aceleradores sobre los modelos supersimétricos con violación de paridad R extendidos con una simetría $U(1)_H$ que presentan un neutralino ó gravitino inestable? Y ¿Cuáles serían las restricciones impuestas por los recientes y futuros datos de todos estos experimentos sobre  modelos que presenten una partícula candidata a materia oscura?
Todo esto representa un escenario muy dinámico donde el Grupo debe estar preparado para interpretar cualquier evidencia que resulte del Higgs o nueva física a la luz de los modelos que viene desarrollando.



\subsection{Planteamiento del problema}
\begin{instrucciones}
  ¿Están vinculados con el problema planteado? ¿Son viables, claros, concretos y factibles, de acuerdo con el estudio y los métodos?
\end{instrucciones}

En el caso de modelos supersimétricos basados en una simetría de sabor anómala $U(1)_H$, con un solo flavón de carga $-1$, una vez se introducen las condiciones teóricas y fenomenológicas, la solución más óptima se puede expresar en términos de 4 cargas $H$ libres que se puede usar para explicar el valor de los 45 parámetros del superpotencial que violan $R$--paridad \cite{Mira:2000gg,Dreiner:2003hw,Dreiner:2003yr,Dreiner:2007vp,Dreiner:2006xw,Sierra:2009zq}. 
\begin{equation}
  \label{eq:5}
  W_{\cancel{R_p}} = \mu_i\widehat{L}_i\widehat{H}_u +
  \lambda_{i j k}\widehat{L}_i\widehat{L}_j\widehat{l}_k +
  \lambda'_{i j k}\widehat{L}_i\widehat{Q}_j\widehat{d}_k +
  \lambda''_{ijk}\widehat{u}_i\widehat{d}_j\widehat{d}_k\,
\end{equation}
De las diferentes posibilidades se pueden construir modelos con
\begin{enumerate}
\item Solamente términos bilineales que violan $R$--paridad, con acoplamientos $\mu_i$ que explican las masas y mezclas de los neutrinos \cite{Mira:2000gg,Dreiner:2003hw,Dreiner:2006xw}
\label{item:1}
\item Modelos donde la simetría discreta de conservación de $R$--paridad, o una simetría equivalente, queda como remanente de la ruptura espontánea de la simetría $U(1)_H$ \cite{Dreiner:2003hw,Dreiner:2003yr,Dreiner:2007vp}. Las masas y mezclas de neutrinos se pueden explicar con la introducción de neutrinos derechos de carga $H$ fraccionaria.
\label{item:2}
\item Modelos con violación de número leptónico a través de términos trilineales con ruptura de $R$--paridad. Se pueden construir modelos con hasta dos términos del tipo $\lambda_{ijk}$, con los tres índices diferentes \cite{Sierra:2009zq}.  En este caso las cargas $H$ se pueden escoger de manera que los acoplamientos trilineales que violan $R$--paridad queden muy suprimidos, del orden de $10^{-23}$, de modo que los neutralinos o gravitinos pueden ser candidatos a materia oscura inestables con un tiempo de vida media del orden de $10^{26}\,$~sec. Dichos modelos se pueden usar para explicar las anomalías en rayos cósmicos \cite{Sierra:2009zq} recientemente detectadas por experimentos como PAMELA \cite{Adriani:2008zr}, ATIC \cite{:2008zzr} y Fermi~LAT \cite{Abdo:2009zk}. Estos datos sugieren que el decaimiento del neutralino sea principalmente leptónico, lo cual hace especialmente interesante que la única posibilidad consistente con simetrías horizontales de lugar precisamente a decaimientos leptónicos.
\label{item:3}
\item Modelos con violación de número bariónico a través de términos trilineales que violan $R$--paridad del tipo $\lambda''_{ijk}$. La fenomenología de este tipo de modelos aún no ha sido estudiada en la literatura
\label{item:4}
\end{enumerate}
%En este proyecto pretendemos estudiar de manera sistemática la fenomenología de los modelos \ref{item:3}., \ref{item:4}. a luz de los resultados experimentales presentes y futuros.

%En el caso \ref{item:3}., en lugar de asumir que los términos trilineales pueden explicar las anomalías de rayos cósmicos, se puede asumir que estas anomalías tienen un origen astrofísico, e intentar usar los datos presentes para restringir los posibles tiempo de vida media y masa del neutralino o gravitino. En esta línea, también se planea estudiar las predicciones de modelos específicos de materia oscura inestable para los experimentos de rayos cósmicos en marcha, y los que comenzarán a funcionar en un futuro cercano. En particular queremos  investigar el flujo de los rayos gamma que generan en modelos donde la materia oscura decae leptónica o hadrónicamente (caso \ref{item:3}) y determinar las regiones del espacio de parámetros que puedan ser confrontadas con las medidas de rayos gamma que se esperan de Fermi--LAT en los próximos meses.

%En el caso \ref{item:3}, en este proyecto pretendemos investigar posibles predicciones de señales para el LHC en modelos donde la simetría horizontal se use para formular modelos consistentes de violación de número leptónico a través de términos trilineales de ruputura $R$--paridad de tipo $\lambda$. Después de tener en cuenta las restricciones experimentales sobre acoplamientos individuales
%, de las cuales la más importante es \cite{Barbier:2004ez}
%\begin{align}
%   \lambda_{11k}''<  10^{-9}   \left( \frac{m_{\tilde g}}{100\,\text{GeV}}  \right)^{1/2}
%\left( \frac{m_{\tilde q_k}}{100\,\text{GeV}}  \right)^2
%\end{align}
%esperamos tener una predicción muy concreta para la jerarquía en los acoplamientos $\lambda$, la cual podría dar lugar a señales muy específicas en el LHC. De este modo, en caso de que se descubriera en el LHC un modelo de ruptura de $R$--paridad a través términos $\lambda$, se podría determinar si la simetría Abeliana  $U(1)_H$ con uno ó más flavones es la simetría que explica la jerarquía en la masas de los fermiones.
En el modelo usual para violación de número leptónico a través de términos $\lambda$, donde se impone una simetría discreta del tipo paridad leptónica, los términos no renormalizables asociados a la masa del neutrino quedan prohibidos, por lo que el mecanismo seesaw no puede ser implementado. En el caso de que la simetría horizontal se use para formular un modelo de este tipo, se puede introducir un nuevo flavón con carga $H$ fraccionaria [del tipo usado en \cite{Chen:2008tc}] que puede dar lugar a un mecanismo de seesaw apropiado para explicar los datos de oscilaciones de neutrinos. En este proyecto pretendemos explorar la posibilidad de obtener una ttextura para las masa de los neutrinos en base a un modelo supersimétrico a la Froggatt-Nielsen, y además obtener un candidato consistente para materia oscura.

\end{ideas}


%%% Local Variables: 
%%% mode: latex
%%% TeX-master: "Ficha-2011_bsm"
%%% End: 
