\section{ Resultados Esperados }
\begin{instrucciones}
CODI:  Impacto y relevancia:
¿El proyecto permite la generación de conocimiento científico o aporta a la resolución de problemas concretos de la realidad? ¿Son suficientes y adecuados los mecanismos de comunicación y socialización de resultados? 

 COLCI: Formule los resultados directos verificables que se
alcanzarán con el desarrollo de los objetivos específicos del proyecto. Estos deben ser coherentes
con los objetivos y con la metodología planteada.



  \begin{enumerate}
  \item \textbf{Relacionados con la generación de conocimiento y/o nuevos desarrollos
 tecnológicos:} Incluye resultados/productos que corresponden a nuevo
 conocimiento científico o tecnológico o a nuevos desarrollos o adaptaciones de
 tecnología que puedan verificarse a través de publicaciones científicas,
 productos o procesos tecnológicos, patentes, normas, mapas, bases de datos,
 colecciones de referencia, secuencias de macromoléculas en bases de datos de
 referencia, registros de nuevas variedades vegetales, etc.
\item \textbf{Conducentes al fortalecimiento de la capacidad científica
  nacional:} Incluye resultados/productos tales como formación de
  recurso humano a nivel profesional o de posgrado (trabajos de grado
  o tesis de maestría o doctorado sustentadas y aprobadas),
  realización de cursos relacionados con las temáticas de los
  proyectos (deberá anexarse documentación soporte que certifique su
  realización), formación y consolidación de redes de investigación
  (anexar documentación de soporte y verificación) y la construcción
  de cooperación científica internacional (anexar documentación de
  soporte y verificación).
\item \textbf{Dirigidos a la apropiación social del conocimiento:}
  Incluye aquellos resultados/productos que son estrategias o medios
  para divulgar o transferir el conocimiento o tecnologías generadas
  en el proyecto a los beneficiarios potenciales y a la sociedad en
  general. Incluye tanto las acciones conjuntas entre investigadores y
  beneficiarios como artículos o libros divulgativos, cartillas,
  videos, programas de radio, presentación de ponencias en eventos,
  entre otros.
  \end{enumerate}

  Para cada uno de los resultados/productos esperados identifique (en
  los cuadros a continuación) indicadores de verificación (ej:
  publicaciones, patentes, registros, videos, certificaciones, etc.)
  as\'\i{} como las instituciones, gremios y comunidades beneficiarias,
  nacionales o internacionales, que podrán utilizar los resultados de
  la investigación para el desarrollo de sus objetivos, políticas,
  planes o programas:

\end{instrucciones}

\begin{instrucciones}
  Los impactos no necesariamente se logran al finalizar el proyecto, ni
con la sola consecución de los resultados/productos. Los impactos
esperados son una descripción de la posible incidencia del uso de los
resultados del proyecto en función de la solución de los asuntos o
problemas estratégicos, nacionales o globales, abordados. Generalmente
se logran en el mediano y largo plazo, como resultado de la aplicación
de los conocimientos o tecnologías generadas a través del desarrollo
de una o varias líneas de investigación en las cuales se inscribe el
proyecto. Los impactos pueden agruparse, entre otras, en las
siguientes categorías: sociales, económicos, ambientales, de
productividad y competitividad. Para cada uno de los impactos
esperados se deben identificar indicadores cualitativos o
cuantitativos verificables as\'\i:
\end{instrucciones}


Nos encontramos en una época única en la que se esperan grandes
descubrimientos en el área de física de altas energías. El LHC ha
comenzado a explorar la escala del TEV desde el 2010. Hasta el 2012
estará acumulando datos a 7\,TeV de energía de centro de masas, y a
partir del 2014 comenzará a funcionar a su energía de diseño de 14
TeV. El satélite Planck fue lanzado en el 2009 y se espera que
entregue datos definitivos sobre los parámetros astronómicos en el
2012. La misión complementará y mejorará las observaciones hechas por
el WMAP. Éste año XENON100 ha entregado sus primeros datos que
comienzan a restringir las regiones del espacio de parámetros de
algunos modelos de materia oscura, y para el 2015 se espera que los
datos de XENON1T cubran completamente la región de materia oscura
predicha por el MSSM restringido. AMS-02 es un detector de rayos
cósmicos que será llevado a la estación espacial internacional a bordo
de la última misión del transbordador Endeavor el 29 de abril de
2011. Se espera que compruebe y mejore las medidas de PAMELA sobre
exceso en el flujo de positrones de rayos cósmicos. Hemos entrado en
una nueva era de la física que, de obtener los resultados esperados,
combinaría los descubrimientos de nuevas partículas, a los cada vez
mejor establecidos resultados de física de neutrinos y observaciones
cosmológicas sobre materia oscura. El resultado de este proyecto es
aportar a esta área de la ciencia con nuevas propuestas de señales
para ser buscadas en estos detectores (subterráneos y en el espacio) y
con la interpretación de los resultados que surjan de ellos en
términos de los modelos propuestos, los cuales añaden partículas
nuevas al modelo estándar de la partículas elementales. Estos se
reflejarán en la publicación de al menos dos artículos científicos en
el área, y en la presentación de los resultados en al menos dos
conferencias internacionales.


De especial importancia es el nuevo paradigma científico que surgirá
de la combinación de todos esos resultados experimentales. Por ejemplo
el descubrimiento en el LHC de la partícula escalar elemental predicha
por el Modelo Estándar, establecería finalmente las teorías gauge con
rompimiento espontáneo de simetría como el principio fundamental para
describir las interacciones subatómicas, redondeando décadas de
desarrollo científico.  El estudio detallado de sus propiedades
acompañado posiblemente de señales de nueva física marcará el camino
para encontrar el mecanismo de generación de masas y mezclas de
neutrinos, así como la determinación de la partícula que compone la
materia oscura del Universo. A más tardar al finalizar ésta década se
espera tener respuesta a estos interrogantes. Todo esto tendrá un
impacto en la enseñanza de la física a todos los niveles. Además el
descubrimiento del Higgs, que sería la primera partícula escalar
elemental, daría un mejor fundamento teórico a los modelos
inflacionarios en cosmología y a la interpretación de la energía
oscura como la causante de la expansión acelerada del Universo. El
otro escenario posible en el que no se encuentre al Higgs del modelo
estándar, debe dar lugar a datos experimentales suficientes para
dilucidar cual es el realmente el mecanismo de ruptura de la simetría
electrodébil. Nuestro grupo es el llamado a difundir estos avances en
nuestra sociedad, como lo ha venido haciendo a través de conferencias
y cursos de extensión en los últimos años. Es importante que nuestro
país siga participando en el desarrollo de la física fundamental, no
sólo con la participación de grupos teóricos como el nuestro, sino
también con grupos experimentales de física de altas energías como lo
viene haciendo en las colaboraciones ATLAS y CMS del LHC con grupos de
la Universidad Antonio Nariño y de los Andes respectivamente. Con
ellos, y con los otros grupos teóricos del país hemos venido
colaborando y organizando congresos en el área en los últimos años
para consolidar ésta área de investigación en el país.

El principal aporte un Grupo como el nuestro al desarrollo del país es
la formación de talento humano con capacidad de hacer investigación
científica al más alto nivel. Para ello es prioritario que nuestro
Grupo siga produciendo productos de gran impacto en la comunidad
mundial de física de altas energías con participación de nuestros
estudiantes de pregrado y posgrado.  Aunque de momento los doctores
que formamos son rápidamente reabsorbidos en el ámbito académico, 
esperamos que a futuro, como pasa en otros países donde profesionales
de este tipo son muy apreciados en empresas de innovación tecnológica,
los nuestros puedan llegar a hacer aportes significativos a otros
sectores de la sociedad.


De ser aprobado, éste proyecto nos permitiría participar en esta
excitante era del desarrollo científico que coincide con
los primeros años de funcionamiento del LHC.  Se espera que entre los
resultados de los próximos años LHC, no solo esté el del
descubrimiento del Higgs, sino también de alguna señal de física más
allá del modelo estándar que explique los problemas fenomenológicos y
teóricos del Modelo Estándar. En los próximos años también se espera
que los experimentos de detección directa, o indirecta a través de
rayos cósmicos, entreguen una evidencia definitiva de materia oscura.



%Final
Publicar al menos 3 artículos internacionales con algunos de los siguientes resultados:

Determinación de la factibilidad del LHC para determinar las correlaciones de VBPR entre el cociente de branchings de neutralino a W mu y W tau con el ángulo de mezcla atmosférico de neutrinos, así como de la correlación entre la longitud de decaimiento del neutralino y la diferencia de masa al cuadrado atmosférica.

Determinación de acoplamientos, vida media y masa del gravitino como CMOI en modelos de ruptura trilineal de paridad R (RTPR) con violación de número leptónico, que pueda explicar el exceso de positrones en rayos cósmicos. Predicciones de flujo de positrones y rayos gama del modelo, para experimentos futuros de rayos cósmicos. 

Determinación de las restricciones que los diferentes experimentos de rayos cósmicos 
presentes y futuros imponen sobre el neutralino como CMOI. 

Determinación de las señales en el LHC para 1) Regiones del espacio de parámetros compatible con física de neutrinos y densidad de materia oscura del seesaw radiativo. 

Formación 1 estudiante de maestría y 1 de doctorado.

\begin{ideas}
  
%Determinación de la factibilidad del LHC para determinar las correlaciones de VBPR entre el cociente de branchings de neutralino a W mu y W tau con el ángulo de mezcla atmosférico de neutrinos, así como de la correlación entre la longitud de decaimiento del neutralino y la diferencia de masa al cuadrado atmosférica.

Determinación del flujo de rayos gamas y su confrontación con lo observado en experimentos de rayos cósmicos en modelos supersimétricos donde el gravitino es el CMOI.

Determinación de acoplamientos, vida media y masa del gravitino como CMOI en modelos de ruptura trilineal de paridad R (RTPR) con violación de número leptónico, que pueda explicar el exceso de positrones en rayos cósmicos. Predicciones de flujo de positrones y rayos gama del modelo para experimentos futuros de rayos cósmicos. 

%Determinación de las restricciones que los diferentes experimentos de rayos cósmicos 
%presentes y futuros que se imponen sobre el neutralino como CMOI. 

Determinación de las señales en el LHC para 1) Modelos de RTPR con violación de número bariónico y mecanismo see-saw para masas de neutrinos, inducida por una simetría Abeliana anómala U(1). 2) Regiones del espacio de parámetros compatible con física de neutrinos y densidad de materia oscura del seesaw radiativo.

% Obtener algunos canales visibles que involucren materia oscura en el LHC, y predecir el flujo de rayos gamma en la galaxia producto de la aniquilacion de materia oscura.

%De otro lado, esperamos construir el primer modelo estándar supersimétrico con una simetría horizontal anómala $U(1)_H$ que se pueda constrastar directamente con resultados experimentales de aceleradores. Así mismo como formular el primer modelo con violación de número leptónico a través de términos $\lambda$ que a la vez de cuenta de las masas y mezclas de neutrinos.

Esperemos obtener predicciones específicas para modelos supersimétricos con contenido mínimo de partículas y una simetría horizontal anómala $U(1)_H$ que puedan comprobarse directamente en experimentos de rayos cósmicos o de aceleradores de partículas.

Formación 1 estudiante de maestría y 1 de doctorado.
\end{ideas}

%%% Local Variables: 
%%% mode: latex
%%% TeX-master: "Ficha-2011_bsm"
%%% End: 
