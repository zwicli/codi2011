\section{ Resultados Esperados }

Se espera que los resultados de está investigación clarifiquen hasta que punto se pueden acotar los valores de los parámetros trilineales que violan $R$--paridad en el caso de que las anomalías actuales en rayos cósmicos puedan ser explicadas por fuentes astrofísicas, como por ejemplo pulsares cercanos \cite[y referencias internas]{Chowdhury:2009jd}. Se espera mejorar las cotas existentes \cite{Barbieri:1988mw,Berezinsky:1991sp,Diwan:1996gw,Berezinsky:1996pb,Baltz:1997ar,Gupta:2004dz,Huber:2005iz} en varios ordenes de magnitud. También esperamos obtener señales específicas que se puedan contrastar con los resultados experimentales de rayos gamma que se esperan en los próximos meses.
%En particular del flujo de rayos gamma que se espera para un neutralino decayendo a $\tau^\pm\,\mu^\mp\,\nu_e$ a través de un acoplamiento trilineal del tipo $\lambda_{123}$, o de un neutralino decayendo hadrónicamente a través de acoplamientos $\lambda''$.
Así mismo se espera determinar el rango de tiempos de vida y media del neutralino como materia oscura inestable que se podrá excluir en experimentos futuros como el AMS-02 \cite{ams:2009}

De otro lado, esperamos construir el primer modelo estándar supersimétrico con una simetría horizontal anómala $U(1)_H$ que se pueda constrastar directamente con resultados experimentales de aceleradores. Así mismo como formular el primer modelo con violación de número bariónico a través de términos $\lambda''$ que a la vez de cuenta de las masas y mezclas de neutrinos.

En conclusión esperemos obtener predicciones específicas para modelos supersimétricos con contenido mínimo de partículas y una simetría horizontal anómala $U(1)_H$ que puedan comprobarse directamente en experimentos de rayos cósmicos o de aceleradores de partículas


%%% Local Variables: 
%%% mode: latex
%%% TeX-master: Ficha-2011_bsm
%%% End: 
