\section{Justificación}
\begin{instrucciones}
  Establecer la importancia y aporte de la investigación
propuesta en función de la generación de conocimiento, el desarrollo tecnológico, la innovación y
la solución a problemas nacionales.
\end{instrucciones}


\begin{instrucciones}
  Estos deben ser coherentes con los objetivos específicos y con la
  metodología planteada.\\
  Los resultados/productos pueden clasificarse en tres categorías:
  \begin{enumerate}
  \item \textbf{Relacionados con la generación de conocimiento y/o nuevos desarrollos
 tecnológicos:} Incluye resultados/productos que corresponden a nuevo
 conocimiento científico o tecnológico o a nuevos desarrollos o adaptaciones de
 tecnología que puedan verificarse a través de publicaciones científicas,
 productos o procesos tecnológicos, patentes, normas, mapas, bases de datos,
 colecciones de referencia, secuencias de macromoléculas en bases de datos de
 referencia, registros de nuevas variedades vegetales, etc.
\item \textbf{Conducentes al fortalecimiento de la capacidad científica
  nacional:} Incluye resultados/productos tales como formación de
  recurso humano a nivel profesional o de posgrado (trabajos de grado
  o tesis de maestría o doctorado sustentadas y aprobadas),
  realización de cursos relacionados con las temáticas de los
  proyectos (deberá anexarse documentación soporte que certifique su
  realización), formación y consolidación de redes de investigación
  (anexar documentación de soporte y verificación) y la construcción
  de cooperación científica internacional (anexar documentación de
  soporte y verificación).
\item \textbf{Dirigidos a la apropiación social del conocimiento:}
  Incluye aquellos resultados/productos que son estrategias o medios
  para divulgar o transferir el conocimiento o tecnologías generadas
  en el proyecto a los beneficiarios potenciales y a la sociedad en
  general. Incluye tanto las acciones conjuntas entre investigadores y
  beneficiarios como artículos o libros divulgativos, cartillas,
  videos, programas de radio, presentación de ponencias en eventos,
  entre otros.
  \end{enumerate}

  Para cada uno de los resultados/productos esperados identifique (en
  los cuadros a continuación) indicadores de verificación (ej:
  publicaciones, patentes, registros, videos, certificaciones, etc.)
  as\'\i{} como las instituciones, gremios y comunidades beneficiarias,
  nacionales o internacionales, que podrán utilizar los resultados de
  la investigación para el desarrollo de sus objetivos, políticas,
  planes o programas:

\end{instrucciones}

%Final
Las evidencias más sólidas de nueva física son las medidas de masas y mezclas de neutrinos y de materia oscura.

La enorme actividad experimental actual en física de partículas intenta determinar los modelos que incorporan esa nueva física. Esto requiere un esfuerzo teórico enorme al que esperamos aportar con este proyecto. En los últimos meses se ha reportando un exceso de rayos cósmicos de electrones y positrones.  Esto ha dado lugar a un sinnúmero de publicaciones tratando de explicar su origen [2]. Los diferentes experimentos de detección de materia oscura han comenzado a obtener resultados prometedores. El LHC ya ha comenzado a producir colisiones a una energía de 7 TeV.

El presente proyecto se centra en proponer modelos que tengan consecuencias medibles en todos estos experimentos, y encontrar modelos que expliquen los resultados de cualquiera de ellos. Pretendemos establecer observables que no sólo permitan descubrir [1] sino también  comprobar si el VBPR es el mecanismo de generación de masas de neutrinos, así como estudiar las implicaciones de modelos con un candidato de materia oscura. 

\newpage{}
%Oscar
En los últimos meses varias colaboraciones han publicado los resultados de las medidas del flujo de rayos cósmicos, reportando un exceso de electrones y positrones, y ningún exceso en el flujo de antiprotones y fotones. 
La comunidad científica ha dedicado un esfuerzo enorme en tratar de dar explicación al exceso en el flujo de rayos cósmicos. Básicamente se han manejado dos propuestas: interpretar el exceso como el producto de la desintegración o aniquilación de partículas de materia oscura en nuestra galaxia; o interpretar el exceso en términos de la producción de positrones y electrones por fuentes astrofísicas exóticas. 
Hay gran expectativa sobre esta línea de investigación pues se espera que los nuevos resultados experimentales al respecto comiencen a dar luz sobre las propiedades de las partículas de materia oscura. Además, se tiene la  búsqueda directa en aceleradores de partículas de  los candidatos a materia oscura.
Creemos que una vez realizados con éxito los objetivos trazados en este proyecto,  las diferentes colaboraciones podrán iniciar la búsqueda de las propuestas sobre las señales del materia oscura propuestas en los modelos descritos anteriormente.





%%Jose David
Es necesario el estudio de modelos como el de seesaw radiativo pues apuntan a solucionar dos de los más importantes problemas en fisica de particulas como en astrofisica, como son materia oscura y masas de neutrinos. Modelos como el de seesaw raditivo son una de las alternativas al hacer extensiones del modelo estandar en las cuales se podrian dar una explicacion a ambos fenomenos. Por eso es importante estudiar los regimenes validos de este modelo con las observaicones actuales, bien sea para aprobarlo como una opcion o para descartarlo.


4 artículos en revistas internacionales, formación de estudiante de
maestría, formación de estudiante de doctorado.

\textbf{Tabla 3.5.1 Generación de nuevo conocimiento}\\
\begin{tabular}{|c|c|c|}\hline
   \textbf{Resultado/Producto}&\textbf{Indicador} & \textbf{Beneficiario}\\
   \textbf{esperado}& & \\\hline
   \parbox[t]{4cm}{Desarrollar todos los detalles de un procedimiento de simulación de se nales de modelo con VBpR en el LHC y el ILC, para extraer con la mayor precisión posible la correlación que se puede llegar a obtener entre experimentos en aceleradores y experimentos de neutrinos}& \parbox[t]{4cm}{Publicaciones en revistas internacionales (mínimo tres artículos), por lo menos una conferencia internacional, seminarios en institutos del país y el exterior}& \parbox[t]{4cm}{Comunidad científica de físicos de partículas elementales y experimentales de altas energías.}\\\hline 
   & & \\\hline
 \end{tabular}

\textbf{Tabla 3.5.2 Fortalecimiento de la comunidad científica}\\
\begin{tabular}{|c|c|c|}\hline
   \textbf{Resultado/Producto}&\textbf{Indicador} & \textbf{Beneficiario}\\
   \textbf{esperado}& & \\\hline
 \parbox[t]{4cm}{Formación de estudiante a nivel de maestría} &\parbox[t]{4cm}{Defensa exitosa de una tesis de Maestría. Publicaciones internacionales firmada por el estudiante} & \\\hline 
\parbox[t]{4cm}{Formación de estudiante a nivel de Doctorado}  & \parbox[t]{4cm}{Defensa exitosa de una tesis de Doctorado a corto plazo, iniciando su formación doctoral en el marco de este proyecto}& \parbox[t]{4cm}{Estudiantes, el grupo de investigación, el Instituto de Física.} \\\hline
\parbox[t]{4cm}{}  & \parbox[t]{4cm}{}& \parbox[t]{4cm}{} \\\hline
 \end{tabular}

\textbf{Tabla 3.5.3 Apropiación social del conocimiento}\\
\begin{tabular}{|c|c|c|}\hline
   \textbf{Resultado/Producto}&\textbf{Indicador} & \textbf{Beneficiario}\\
   \textbf{esperado}& & \\\hline
\parbox[t]{4cm}{}  & \parbox[t]{4cm}{}& \parbox[t]{4cm}{} \\\hline
   & & \\\hline
 \end{tabular}
\\
%%%%%%%%%%%%        Inicio  OSCAR
Inicio Oscar.\\
En los últimos meses varias colaboraciones han publicado los resultados de sus experimentos, 
reportando un exceso de electrones y positrones  en el espectro de energía de los rayos cósmicos. 
Con respecto al flujo de positrones y antiprotones predicho por los modelos teóricos de producción y 
propagación de rayos cósmicos [3, 4, 5], los datos de la colaboración PAMELA muestran un exceso de 
positrones en el espectro de energía de los rayos cósmicos entre 10 y 100 GeV [1], pero no se 
observa un exceso en el flujo de antiprotones [2]. Estos resultados son compatibles con los 
experimentos anteriores HEAT y AMS01  [6, 7], pero con una precisión y estadística mucho mayor. 
Poco después de los datos de PAMELA, los experimentos con globos en el polo sur, ATIC y PPB-BETS, 
publicaron sus resultados, señalando un exceso en el flujo de electrones más positrones para energías 
entre 300 y 800 GeV [8, 9]. En mayo de 2009, la colaboración Fermi-LAT publicó los datos mostrando, 
al igual que ATIC, un exceso en el flujo de electrones más positrones. Sin embargo, 
el espectro medido por Fermi-LAT es más suave que el que medido por ATIC.

La comunidad científica ha dedicado un esfuerzo enorme en tratar de dar explicación
 al exceso en el flujo de rayos cósmicos. Básicamente se han manejado dos propuestas: 
interpretar el exceso como el producto de la desintegración [10,15,16] o aniquilación [11] de partículas 
de materia oscura en nuestra galaxia; o interpretar el exceso en términos de la producción de 
positrones y electrones por fuentes astrofísicas exóticas [12] tales como pulsares, 
remanentes de supernova, etc. Vale la pena mencionar que la idea de explicar exceso 
de electrones y positrones a través del producto de desintegraciones o aniquilaciones de 
partículas de materia oscura ha generado mucho entusiasmo en la comunidad científica debido 
a que precisamente dicho exceso sería una evidencia experimental de la existencia de 
materia oscura, la cual representa cerca del $23\%$ de la energía de nuestro universo [13]. 

Para tratar de explicar por medio de desintegraciones de la materia oscura el exceso de 
electrones y positrones reportados por PAMELA y Fermi–LAT, se supone un candidato a 
materia oscura que decaiga con un tiempo de vida media de $10^26$ sec. Entre los desarrollos 
al respecto se encuentra la posibilidad de usar al neutralino como materia oscura que decae, 
en el contexto de modelos supersimétricos con violación de paridad R [10, 14, 15, 16]. 
Esta posibilidad requiere acoplamientos que violan paridad R extremadamente pequeños. 
En [16] hemos mostrado que una simetría horizontal abeliana puede explicar la pequeñez 
de esos acoplamientos. Hay gran expectativa sobre esta línea de investigación pues se 
espera que los nuevos resultados experimentales al respecto comiencen a dar luz sobre 
el mecanismo que da lugar al exceso de positrones y antiprotones en rayos cósmicos.

En modelos supersimétricos extendidos para incluir una simetría horizontal anómala $U(1_)H$ 
a la Froggatt-Nielsen (FN) [17], las partículas del modelo estándar y sus supercompañeros no 
llevan un número cuántico de paridad R, y en su lugar llevan una carga horizontal (carga $H$). 
Para un artículo de revisión ver [18]. Además, este tipo de modelos involucran nuevos campos 
pesados FN y, en la realización más simple, un supercampo singlete electrodébil $\phi$ de carga $H = -1$. 
Términos efectivos invariantes bajo $SU(3)_C\times SU(2)_L\times U(1)_Y\times U(1)_H$  que conservan y también 
que violan paridad R, surgen una vez los grados de libertad pesados son integrados debajo la 
escala de los campos FN ($M$). Estos términos involucran factores del tipo $(\phi/M)^n$, donde $n$ está 
fijado  por las cargas horizontales de los campos involucrados, y determina si un término en 
particular puede o no estar presente en el superpotencial. La holomorfía del superpotencial 
prohíbe todos los términos para los cuales $n < 0$, y aunque ellos se generarán después la 
ruptura de simetría vía el potencial de Kahler [19] estos términos son en general mucho 
más suprimidos que aquellos para los cuales $n \geq 0$. Términos con $n$ fraccionario están 
también prohibidos y en contraste a aquellos con $n < 0$ no existe un mecanismo mediante 
el cual ellos pueden ser generados. Finalmente, una vez $U(1)_H$ se rompe los términos 
con $n$ positivo mantienen acoplamientos de Yukawa determinados, hasta factores de orden uno, 
por $\theta=(<\phi>/M)^n$. Los acoplamientos de Yukawa del modelo estándar provienen típicamente 
de términos de esta clase. Así mismo, modelos supersimétricos basados en factor abeliano 
$U(1)_H$ quedan completamente especificados en términos de las cargas $H$.

En el caso de modelos supersimétricos basado en una simetría de sabor anómala $U(1)_H$ 
las masas de los quarks, los ángulos de mezcla de los quarks, las masa de los leptones cargados, 
y las condiciones de cancelación de anomalías restringen las posible asignaciones de cargas $H$ [20, 21]. 
Ya que el número de restricciones es siempre menor que el número de cargas $H$ alguna de ellas 
quedan necesariamente sin restricciones, y aparte del límite superior en sus valores pueden 
considerarse como parámetros libres que se puede determinar a partir de información 
fenomenológica adicional. Para este propósito los datos experimentales de neutrinos 
se ha usado dando lugar a modelos en los cuales se explican las masas de los 
neutrinos [18, 22, 23, 24, 25, 26]. En [16] hemos adoptado otra aproximación 
asumiendo un neutralino que decae como candidato de materia oscura. El único modelo 
viable que resulta con acoplamientos que violan número leptónico es un modelo con 
hasta dos términos trilineales del tipo $\lambda_{ijk}$ con todos los índices diferentes. 
Resulta que este tipo de modelo es justamente el adecuado para explicar las anomalías 
reportadas flujos de rayos cósmicos de electrones y positrones.

También se pueden obtener modelos con violación de número bariónico a través de 
términos trilineales que violan paridad R del tipo $\lambda_{ijk}''$. Teniendo en cuenta las 
restricciones experimentales sobre acoplamientos individuales, de las cuales la más 
importante es [27] $\lambda''< 10^{-7}$ se espera tener una predicción muy concreta para la 
jerarquía en los acoplamientos $\lambda''$, la cual podría dar lugar a señales muy específicas 
en el LHC. De este modo, en caso de que se descubriera en el LHC un modelo de violación 
trilineal de paridad R con términos $\lambda''$, se podría determinar si la simetría abeliana 
$U(1)_H$ con un sólo flavón es la simetría que explica la jerarquía en la masas de los fermiones.


Creemos que una vez realizados con éxito los objetivos trazados en este proyecto, 
las diferentes colaboraciones experimentales encargadas de medir las propiedades 
de los rayos cósmicos y las colaboraciones ATLAS y CMS del LHC podrán iniciar 
la búsqueda de las propuestas sobre las señales del neutralino inestable en modelos 
supersimétricos extendidos con una simetría abeliana U(1)H. 

[1] O. Adriani et al., Nature 458, 607 (2009).

[2] O. Adriani et al., Phys. Rev. Lett. 102, 051101 (2009).

[3] A. W. Strong y I. V. Moskalenko, Astrophys. J. 493, 694 (1998).

[4] E. A. Baltz y J. Edsjo, Phys. Rev. D 59, 023511 (1999).

[5] T. Bringmann, y P. Salati, Phys. Rev D 75, 083006 (2007).

[6] S. W. Barwick et al., Astrophys. J. 482, L191 (1997).

[7] M. Aguilar et al., Phys. Lett. B 646, 145 (2007).

[8] J. Chang et al., Nature 456, 362 (2008).

[9] S. Torii et al., arXiv:0809.0760 [astro-ph].

[10] C. R. Chen, F. Takahashi y T. T. Yanagida, Phys. Lett. B 671, 71 (2009); C. R. Chen y F. Takahashi, JCAP 0902, 004 (2009); 
C. R. Chen, F. Takahashi y T. T. Yanagida, Phys. Lett. B 673, 255 (2009); A. Ibarra y D. Tran, JCAP 0902 (2009) 021; 
K. Hamaguchi, S. Shirai y T. T. Yanagida, Phys. Lett. B 673, 247 (2009); E. Nardi, F. Sannino y A. Strumia, JCAP 0901, 043 (2009);

[11] M. Cirelli, R. Franceschini y A. Strumia, Nucl. Phys. B 800, 204 (2008). M. Cirelli, M. Kadastik, M. Raidal y A. Strumia, 
Nucl. Phys. B 813, 1 (2009); P. D. Serpico, Phys. Rev. D 79, 021302 (2009); Q. H. Cao, E. Ma y G. Shaughnessy, 
Phys. Lett. B 673, 152 (2009).

[12] D. Hooper, P. Blasi y P. D. Serpico, JCAP 0901, 025 (2009); H. Yuksel, M. D. Kistler y T. Stanev,
 Phys. Rev. Lett. 103, 051101 (2009).

[13] C. Amsler et al., Phys. Lett. B 667, 1 (2008).

[14] R. Barbieri y V. Berezinsky, Phys. Lett. B205, 559–563 (1988); V. Berezinsky, A. Masiero, y J. W. F. Valle, 
Phys. Lett. B266, 382–388 (1991); V. Berezinsky, A. S. Joshipura, y J. W. F. Valle, Phys. Rev. D57, 147–151 (1998); 
E. A. Baltz y P. Gondolo, Phys. Rev. D57, 7601–7606 (1998);  S. K. Gupta, P. Konar y B. Mukhopadhyaya, 
Phys. Lett. B606, 384–390 (2005); S. J. Huber, JCAP 0602, 008 (2006); I. Gogoladze, R. Khalid, Q. Shafi y H. Yuksel, 
Phys. Rev. D79, 055019 (2009); K. Ishiwata, S. Matsumoto y T. Moroi, JHEP 05, 110 (2009); S. Shirai, F. Takahashi, 
and T. T. Yanagida, Phys. Lett. B680, 485–488 (2009).

[15] P. f. Yin, Q. Yuan, J. Liu, J. Zhang, X. j. Bi y S. h. Zhu, Phys. Rev. D 79, 023512 (2009).

[16] D. A. Sierra, D. Restrepo y O. Zapata, Phys.Rev. D 80, 055010 (2009).

[17] C. D. Froggatt y H. Bech Nielsen, Nucl. Phys. B147, 277 (1979).

[18] H. K. Dreiner y M. Thormeier, Phys. Rev. D69, 053002 (2004).

[19] G. F. Giudice y A. Masiero, Phys. Lett. B206, 480–484 (1988).

[20] M. Leurer, Y. Nir, y N. Seiberg, Nucl. Phys. B398, 319–342 (1993).

[21] P. S. Lavignac, y P. Ramond, Nucl. Phys. B477, 353–377 (1996).

[22] K. Choi, K. Hwang, y E. Jin Chun, Phys. Rev. D60, 031301 (1999);

[23] J. M. Mira, E. Nardi, D. A. Restrepo, y J. W. F. Valle, Phys. Lett. B492, 81–90 (2000).

[24] H. K. Dreiner, H. Murayama, y M. Thormeier, Nucl. Phys. B729, 278–316 (2005).

[25] H. K. Dreiner, C. Luhn, H. Murayama, y M. Thormeier, Nucl. Phys. B774, 127–167 (2007).

[26] H. K. Dreiner, C. Luhn, H. Murayama, y M. Thormeier, Nucl. Phys. B795, 172–200 (2008).

[27] R. Barbier et al., Phys. Rept. 420, 1–202 (2005). 

Fin Oscar
%%%%%%%%%%%%        Fin  OSCAR

\subsection{Impactos esperados a partir del uso de los resultados:}

\begin{instrucciones}
  Los impactos no necesariamente se logran al finalizar el proyecto, ni
con la sola consecución de los resultados/productos. Los impactos
esperados son una descripción de la posible incidencia del uso de los
resultados del proyecto en función de la solución de los asuntos o
problemas estratégicos, nacionales o globales, abordados. Generalmente
se logran en el mediano y largo plazo, como resultado de la aplicación
de los conocimientos o tecnologías generadas a través del desarrollo
de una o varias líneas de investigación en las cuales se inscribe el
proyecto. Los impactos pueden agruparse, entre otras, en las
siguientes categorías: sociales, económicos, ambientales, de
productividad y competitividad. Para cada uno de los impactos
esperados se deben identificar indicadores cualitativos o
cuantitativos verificables as\'\i:
\end{instrucciones}

\begin{tabular}{|c|c|c|c|}\hline
   \textbf{\parbox[t]{2cm}{\textbf{Impacto esperado}}}&\parbox[c]{4cm}{\textbf{Plazo (a nos) después de finalizado el proyecto: corto (1-4 ), mediano (5-9), largo (10 o más)}} & \textbf{\parbox[t]{2cm}{\textbf{Indicador verificable}}} &\textbf{Supuestos*}\\
   & & &\\\hline 
   & & &\\\hline
 \end{tabular}\\
*Los supuestos indican los acontecimientos, las condiciones o las decisiones,
necesarios para que se logre el impacto esperado.


Consultar proyecto viejo



\subsection{Conformación y trayectoria del Grupo de Investigación}
\begin{instrucciones}
  Con esta sección se pretende establecer la capacidad del grupo de
  investigación y de sus integrantes para realizar el proyecto
  propuesto. Esto significa conocer su importancia estratégica y
  logros a partir de proyectos de investigación realizados
  anteriormente o en curso, incluyendo sus productos más relevantes.
  La información suministrada deberá reflejar la capacidad del grupo
  de investigación y de sus integrantes para realizar el proyecto
  propuesto. Debe indicar el nombre del Grupo como está registrado en
  GrupLAC.
\end{instrucciones}
\begin{evaluacion}
  A tener en cuenta: 5/100
\end{evaluacion}
El Grupo de Fenomenología de las Interacciones Fundamentales (GFIF)
del Instituto de Física de la Universidad de Antioquia ha estado
trabajando en varios aspectos de la física de partículas en los
últimos 20 a nos. El Grupo ha producido muchos artículos
internacionales con resultados originales, ha desarrollado decenas de
proyectos de investigación exitosamente, y ha sido con premios de
excelencia de Conciencias y otras instituciones. El Grupo ofrece
formación a nivel doctoral desde 1997 y a la fecha a graduado a 3
doctores. El Grupo continua trabajando en el programa de Maestría
donde ha graduado a más de 20 estudiantes, varios de ellos, incluyendo
el investigador principal de este proyecto, continuando su formación
doctoral en prestigiosas universidades en el exterior.El Grupo cuenta
en la actualidad con siete profesores, todos ellos con título de
Doctor, cinco estudiantes de doctorado y ocho de maestría

El Grupo de Física Teórica de la Universidad Nacional sede Medellin,
esta liderado por Luis Alberto Sánchez. \'El realizó su doctorado con el
GFIF bajo la dirección del profesor William Ponce. Despues de
graduarse ha mantenido una colaboración muy estrecha con el Grupo de
la Universidad de Antioquia especialmente realizando trabajos
conjuntos sobre modelos 3-3-1

	

%%% Local Variables: 
%%% mode: latex
%%% TeX-master: "main"
%%% End: 
	